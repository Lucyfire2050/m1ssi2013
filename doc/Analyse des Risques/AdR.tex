\documentclass{../../res/univ-projet}

\usepackage[utf8]{inputenc}
\usepackage[T1]{fontenc}
\usepackage[francais]{babel}

\usepackage{multirow}

\logo{../../res/logo_univ.png}
\title{Analyse des Risques}
\author{Adrien \bsc{Smondack}}
\projet{M1SSI}
\projdesc{Projet de génération d'OTP}
\filiere{M1SSI}
\version{2.1}
\relecteur{Benjamin \bsc{Zigh}}
\signataire{\bsc{Bardet} Magali}
\date{Février 2014}

\histentry{2.0}{21/02/2014}{Version post état de l'art}
\histentry{1.2}{16/01/2014}{Version pour la revue de lancement.}
\histentry{1.1}{20/12/2013}{Version relue et corrigée.}
\histentry{1.0}{18/12/2013}{Version initiale.}

\begin{document}
\maketitle

%------------------------------------------------------------------------------------------------------------------------------------------------------------------
\section{\'Evaluation des risques potentiels}
\subsection{Classification des points durs selon leur importance}
\begin{tabular}{| p{9cm} | p{5cm} |} 
	\hline
	\cellcolor{gray} Description & \cellcolor{gray} Membres concernés \\ \hline
	\'Etude des RFC décrivant les systèmes / \'Etat de l'art & Toute l'équipe \\ \hline
	Démonstration de la sécurité des systèmes implémentés & Resp. qualité \\ \hline
	Planifier l'évolution du projet & Chef de projet \\ \hline
	Communication régulière avec le client & MOA \\ \hline
	Formation/suivi des équipes & Resp. tech./qual., Chef de projet \\ \hline
	Surveiller l'évolution des spécifications & MOA \\ \hline
	Emploi de technologies spécifiques (Android, Java Card, PAM) & Toute l'équipe \\ \hline
	Communication au sein de l'équipe & Chef de projet \\ \hline
	Mutualisation des connaissances et des compétences & Toute l'équipe \\ \hline
	Gestion globale du projet (documentation) & Toute l'équipe\\ \hline
	Veiller aux besoins techniques des membres de l'équipe & Chef de projet \\ \hline
\end{tabular}

%------------------------------------------------------------------------------------------------------------------------------------------------------------------
\subsection{Points durs \& risques associés}
\begin{tabular}{| p{8.7cm} | p{0.8cm} | p{4.5cm} |}
	\hline
	\cellcolor{gray} Point dur & \multicolumn{2}{p{6cm} |}{\cellcolor{gray} Risques associés} \\ \hline
	\cellcolor{lightgray} & \cellcolor{lightgray} Id & \cellcolor{lightgray} Description \\ 	\hline
	\multirow{2}{*}{Étude des RFC décrivant les systèmes / État de l'art} 
		& R\_01 & Le développement effectué ne correspond pas aux RFC \\ \cline{2-3}
		& R\_02 & État de l'art pas assez exhaustif \\ \hline
	\multirow{2}{*}{Démonstration de la sécurité des systèmes implémentés}
		& R\_03 & Faille de sécurité découverte après test \\ \hline
	\multirow{1}{*}{Planifier l'évolution du développement du projet}
		& R\_04 & Impondérables (surcharge de travail) \\ \hline
	\multirow{3}{*}{Communication régulière avec le client}
		& R\_05 & Le client change ses objectifs \\ \cline{2-3}
		& R\_06 & Passer à côté d'une demande du client \\ \cline{2-3}
		& R\_07 & Les clients font faillite \\ \hline
	\multirow{1}{*}{Formation / suivi des équipes}
		& R\_08 & Retard d'une partie de l'équipe sur la compréhension théorique ou les compétences techniques \\ \hline
	\multirow{2}{*}{Surveiller l'évolution des spécifications}
		& R\_09 & Spécification incomplète \\ \cline{2-3}
		& R\_10 & Spécification trop exigeantes (Problème pour atteindre la performance désirée) \\ \hline
	\multirow{2}{*}{Emploi de techno. spécifiques (Android, Java Card, PAM)}
		& R\_11 & Appréhension des nouvelles technologies plus longue que prévu \\ \cline{2-3}
		& R\_12 & Mauvaise appréhension des nouvelles technologies \\ \hline
	\multirow{2}{*}{Communication au sein de l'équipe}
		& R\_13 & Chef de projet incompétent \\ \cline{2-3}
		& R\_14 & Conflit/tension au sein de l'équipe \\ \hline
	\multirow{1}{*}{Mutualisation des connaissances et des compétences}
		& R\_15 & Départ d'une ou plusieurs personnes clé du projet \\ \hline
	\multirow{3}{*}{Gestion globale du projet (documentation)}
		& R\_16 & Phase de développement plus longue que prévu \\ \cline{2-3}
		& R\_17 & Le développement effectué ne correspond pas à la demande du client \\ \cline{2-3}
		& R\_18 & Sous estimation des efforts nécessaires à la réalisation du projet \\ \hline
	\multirow{2}{*}{Veiller aux besoins techniques des membres de l'équipe}
		& R\_19 & Un membre se retrouve privé de ses outils de travail \\ \cline{2-3}
		& R\_20 & Problème avec le git \\ \hline
\end{tabular}
\newpage

\subsection{Criticité}
	\begin{center}
		\begin{tabular}{  c  p{2cm} | p{1.5cm} | p{1.5cm} | p{1.5cm} | p{1.5cm} |}
	  		\cline{3-6}
	   		& \multicolumn{1}{c |}{} & \multicolumn{4}{ c |}{Probabilité} \\ \cline{3 - 6}
	   		& & \cellcolor{lightgray} très faible & \cellcolor{lightgray} faible & \cellcolor{lightgray} moyenne & \cellcolor{lightgray} forte \\ \hline
		   \multicolumn{1}{| c |}{\multirow{4}{*}{Impact}} & \cellcolor{lightgray} Acceptable &\cellcolor{green} &\cellcolor{green} &\cellcolor{green} &\cellcolor{green} \\ \cline{2-6}
		   \multicolumn{1}{| c |}{} & \cellcolor{lightgray} Préoccupant &\cellcolor{green} &\cellcolor{green} &\cellcolor{green} & \cellcolor{red}\\ \cline{2-6}
	   		\multicolumn{1}{| c |}{} & \cellcolor{lightgray} Important &\cellcolor{green} &\cellcolor{green} &\cellcolor{red} & \cellcolor{red} \\ \cline{2-6}
	   		\multicolumn{1}{| c |}{} & \cellcolor{lightgray} Critique &\cellcolor{green} &\cellcolor{red} &\cellcolor{red} &\cellcolor{red} \\ \hline
		\end{tabular}
	\end{center}

%------------------------------------------------------------------------------------------------------------------------------------------------------------------
\section{Gestion des risques les plus importants}
\subsection{Top 5}
\subsubsection{Risque 5 :}
	\begin{tabular}{| p{5cm} | p{2.5cm} | p{6.5cm} |} 
		\hline
		\multicolumn{3}{| p{14,5cm} |}{\cellcolor{gray}Identifiant : R\_08} \\ \hline
		\cellcolor{lightgray} Évènement/situation redouté(e) & \multicolumn{2}{p{9cm}|}{Retard d'une partie de l'équipe sur la compréhension théorique ou les compétences techniques.} \\ \hline
		\cellcolor{lightgray} Point dur associé & \multicolumn{2}{p{9cm}|}{Formation / suivi des équipes.} \\ \hline
		\cellcolor{lightgray} Impact & \multicolumn{2}{p{9cm}|}{Préoccupant.} \\ \hline
		\cellcolor{lightgray} Probabilité & \multicolumn{2}{p{9cm}|}{Forte.} \\ \hline
		\cellcolor{lightgray} Critique & \multicolumn{2}{p{9cm}|}{\cellcolor{red} Oui}\\ \hline
		\cellcolor{lightgray} Plan de réduction de risque & \multicolumn{2}{p{9cm}|}{Suivi régulier des équipes par le responsable technique et le chef de projet.} \\ \hline
		\cellcolor{lightgray}Plan d'action & Acteur : & Responsable technique, chef de projet \\ \cline{2-3}
		\cellcolor{lightgray}	& Déclenchement : & Demande de la part d'une équipe \\ \cline{2-3}
		\cellcolor{lightgray}	& But : & Rétablir la compéhension technique \\ \cline{2-3}
		\cellcolor{lightgray}	& Action : & Consultation du responsable technique. Remaniement de l'équipe si pas d'amélioration. \\ \hline
		\cellcolor{lightgray}Statut/Date/Responsable & \multicolumn{2}{p{9cm} |} {Créé le 29/11/13 par Adrien \bsc{Smondack}.} \\ \hline
	\end{tabular}

\subsubsection{Risque 4 :}
	\begin{tabular}{| p{5cm} | p{2.5cm} | p{6.5cm} |}
		\hline
		\multicolumn{3}{| p{14,5cm} |}{\cellcolor{gray}Identifiant : R\_11} \\ \hline
		\cellcolor{lightgray}Évènement/situation redouté(e) & \multicolumn{2}{p{9cm}|}{Appréhension des nouvelles technologies plus longue que prévu.} \\ \hline
		\cellcolor{lightgray}Point dur associé & \multicolumn{2}{p{9cm}|}{Emploi de technologies spécifiques (Android, Java Card, PAM).} \\ \hline
		\cellcolor{lightgray} Impact & \multicolumn{2}{p{9cm}|}{Préoccupant.} \\ \hline
		\cellcolor{lightgray} Probabilité & \multicolumn{2}{p{9cm}|}{Forte.} \\ \hline
		\cellcolor{lightgray} Critique & \multicolumn{2}{p{9cm}|}{\cellcolor{red} Oui}\\ \hline
		\cellcolor{lightgray} Plan de réduction de risque & \multicolumn{2}{p{9cm}|}{Suivi régulier des équipes par le responsable technique et le chef de projet. Rédaction de rapports explicatifs quant à la technologie étudiée (généralités, fonctionnalités nécessaires et/ou utiles pour le projet, particularités...).} \\ \hline	
		\cellcolor{lightgray}Plan d'action & Acteur : & Chef de projet \\ \cline{2-3}
		\cellcolor{lightgray}	& Déclenchement : & Demande de la part d'une équipe \\ \cline{2-3}
		\cellcolor{lightgray}	& But : & Accélérer l'apprentissage de l'équipe \\ \cline{2-3}
		\cellcolor{lightgray}	& Action : & Ajout d'une personne à l'équipe. Un remplacement est envisageable. \\ \hline
		\cellcolor{lightgray} Statut/Date/Responsable & \multicolumn{2}{p{9cm}|}{Créé le 29/11/13 par Adrien \bsc{Smondack}.} \\ \hline
	\end{tabular}

\subsubsection{Risque 3}
	\begin{tabular}{| p{5cm} | p{2.5cm} | p{6.5cm} |}
		\hline
		\multicolumn{3}{| p{14,5cm} |}{\cellcolor{gray}Identifiant : R\_03} \\ \hline
		\cellcolor{lightgray} Évènement/situation redouté(e) & \multicolumn{2}{p{9cm}|}{Faille de sécurité découverte après test.} \\ \hline
		\cellcolor{lightgray} Point dur associé & \multicolumn{2}{p{9cm}|}{Démonstration de la sécurité des systèmes implémentés.} \\ \hline
		\cellcolor{lightgray} Impact & \multicolumn{2}{p{9cm}|}{Important.} \\ \hline
		\cellcolor{lightgray} Probabilité & \multicolumn{2}{p{9cm}|}{Moyenne.} \\ \hline
		\cellcolor{lightgray} Critique & \multicolumn{2}{p{9cm}|}{\cellcolor{red} Oui}\\ \hline
		\cellcolor{lightgray} Plan de réduction de risque & \multicolumn{2}{p{9cm}|}{Suivi ($\Rightarrow$ tests) régulier des fonctionnalités du produit.} \\ \hline
		\cellcolor{lightgray}Plan d'action & Acteur : & Équipe de développement \\ \cline{2-3}
		\cellcolor{lightgray}	& Déclenchement : & Faille détectée \\ \cline{2-3}
		\cellcolor{lightgray}	& But : & Réparer la faille de sécurité \\ \cline{2-3}
		\cellcolor{lightgray}	& Action : & Déclencher une procédure identique à celle lancée en cas de test échoué. Si le temps manque, focaliser autant d'équipiers que nécessaire sur le problème.\\ \hline
		\cellcolor{lightgray} Statut/Date/Responsable & \multicolumn{2}{p{9cm}|}{Créé le 29/11/13 par Adrien \bsc{Smondack}.} \\ \hline
	\end{tabular}

\subsubsection{Risque 2}
	\begin{tabular}{| p{5cm} | p{2.5cm} | p{6.5cm} |}
		\hline
		\multicolumn{3}{| p{14,5cm} |}{\cellcolor{gray}Identifiant : R\_16} \\ \hline
		\cellcolor{lightgray}Évènement/situation redouté(e) & \multicolumn{2}{p{9cm}|}{Phase de développement plus longue que prévu.} \\ \hline
		\cellcolor{lightgray}Point dur associé & \multicolumn{2}{p{9cm}|}{Gestion globale du projet (documentation).} \\ \hline
		\cellcolor{lightgray}Impact & \multicolumn{2}{p{9cm}|}{Important.} \\ \hline
		\cellcolor{lightgray}Probabilité & \multicolumn{2}{p{9cm}|}{Forte.} \\ \hline
		\cellcolor{lightgray}Critique & \multicolumn{2}{p{9cm}|}{\cellcolor{red} Oui}\\ \hline
		\cellcolor{lightgray}Plan de réduction de risque & \multicolumn{2}{p{9cm}|}{Suivi régulier des équipes par le responsable technique et le chef de projet.} \\ \hline
		\cellcolor{lightgray}Plan d'action & Acteur : & Chef de projet \\ \cline{2-3}
		\cellcolor{lightgray}	& Déclenchement : & Date de fin de tache prévue dépasse celle du planning \\ \cline{2-3}
		\cellcolor{lightgray}	& But : & Éviter la propagation du retard.\\ \cline{2-3}
		\cellcolor{lightgray} & Action : & Réunion d'équipe pour analyser et adapter les démarches en vue d'accélérer la productivité. Éventuellement : Remaniement du planning en vue de ré-agencer les tâches\\ \hline
		\cellcolor{lightgray}Statut/Date/Responsable & \multicolumn{2}{p{9cm}|}{Créé le 29/11/13 par Adrien \bsc{Smondack}.} \\ \hline
	\end{tabular}

\subsubsection{Risque 1}
	\begin{tabular}{| p{5cm} | p{2.5cm} | p{6.5cm} |}
		\hline
		\multicolumn{3}{| p{14,5cm} |}{\cellcolor{gray}Identifiant : R\_04} \\ \hline
		\cellcolor{lightgray}Évènement/situation redouté(e) & \multicolumn{2}{p{9cm}|}{Impondérables (surcharge de travail).} \\ \hline
		\cellcolor{lightgray}Point dur associé & \multicolumn{2}{p{9cm}|}{Planifier l'évolution du développement du projet.} \\ \hline
		\cellcolor{lightgray}Impact & \multicolumn{2}{p{9cm}|}{Important.} \\ \hline
		\cellcolor{lightgray}Probabilité & \multicolumn{2}{p{9cm}|}{Forte.} \\ \hline
		\cellcolor{lightgray}Critique & \multicolumn{2}{p{9cm}|}{\cellcolor{red} Oui}\\ \hline
		\cellcolor{lightgray}Plan de réduction de risque & \multicolumn{2}{p{9cm}|}{Raccourcir les délais afin de garder un peu d'avance.} \\ \hline
		\cellcolor{lightgray}Plan d'action & Acteur : & Chef de projet \\ \cline{2-3}
		\cellcolor{lightgray} & Déclenchement : & - \\ \cline{2-3}
		\cellcolor{lightgray} & But : & Tenir les délais annoncés.\\ \cline{2-3}
		\cellcolor{lightgray} & Action : & Augmentation du rythme de travail. Mise en commun des ressources et des compétences de toute l'équipe. Prolonger les délais n'est pas envisageable.\\ \hline
		\cellcolor{lightgray}Statut/Date/Responsable & \multicolumn{2}{p{9cm}|}{Créé le 29/11/13 par Adrien \bsc{Smondack}.} \\ \hline
	\end{tabular}

\newpage
%------------------------------------------------------------------------------------------------------------------------------------------------------------------
\subsection{Les autres risques}
\begin{flushleft}
	\begin{tabular}{| p{1.4cm} | p{4cm} | p{4cm} | p{1.7cm} | p{1.6cm} | p{1.2cm} |}
  		\hline
  		\cellcolor{gray}Identifiant & \cellcolor{lightgray}Évènement/situation redouté(e) & \cellcolor{lightgray}Point dur associé & \cellcolor{lightgray}Impact & \cellcolor{lightgray}Probabilité & \cellcolor{lightgray}Critique \\ \hline
  
  		R\_01 & Le développement effectué ne correspond pas aux RFC. & Étude des RFC décrivant les systèmes / État de l'art & Important & Faible & \cellcolor{green} Non \\ \hline
  		R\_02 & État de l'art pas assez exhaustif. & Étude des RFC décrivant les systèmes / État de l'art. & Important & Faible. & \cellcolor{green} Non \\ \hline
  		R\_05 & Le client change ses objectifs. & Communication régulière avec le client. & Important & Faible & \cellcolor{green} Non\\ \hline
  		R\_06 & Passer à côté d'une demande du client. & Communication régulière avec le client. & Préoccupant & Faible & \cellcolor{green} Non \\ \hline
  		R\_07 & Les clients font faillite. & Communication régulière avec le client. & Critique & (Très) Faible & \cellcolor{green} Non\\ \hline
  		R\_09 & Spécification incomplète. & Surveiller l'évolution des spécifications. & Préoccupant & Moyenne & \cellcolor{green} Non\\ \hline
  		R\_10 & Spécification trop exigeantes (Problème pour atteindre la performance désirée). & Surveiller l'évolution des spécifications. & Préoccupant & Moyenne & \cellcolor{green} Non\\ \hline
  		R\_12 & Mauvaise appréhension des nouvelles technologies & Emploi de technologies spécifiques (Android, Java Card, PAM). & Préoccupant & Moyenne & \cellcolor{green} Non \\ \hline
  		R\_13 & Chef de projet incompétent. & Communication au sein de l'équipe. & Acceptable & Moyenne & \cellcolor{green} Non\\ \hline
  		R\_14 & Conflit/tension au sein de l'équipe. & Communication au sein de l'équipe. & Préoccupant & Faible & \cellcolor{green} Non\\ \hline
  		R\_15 & Départ d'une ou plusieurs personnes clé du projet. & Mutualisation des connaissances et des compétences. & Important & Faible & \cellcolor{green} Non\\ \hline
  		R\_17 & Le développement effectué ne correspond pas à la demande du client. & Gestion globale du projet (documentation). & Critique & Faible & \cellcolor{red} Oui\\ \hline
  		R\_18 & Sous estimation des efforts nécessaires à la réalisation du projet. & Gestion globale du projet (documentation). & Important & Faible & \cellcolor{green} Non\\ \hline
  		R\_19 & Un membre se retrouve privé de ses outils de travail. & Veiller aux besoins techniques des membres de l'équipe. & Acceptable & Faible & \cellcolor{green} Non\\ \hline
  		R\_20 & Problème avec le git. & Veiller aux besoins techniques des membres de l'équipe. & Préoccupant & (Très) Faible & \cellcolor{green} Non\\ \hline
	\end{tabular}
\end{flushleft}

%------------------------------------------------------------------------------------------------------------------------------------------------------------------
\end{document}