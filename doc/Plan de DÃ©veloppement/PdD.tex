\documentclass{../../res/univ-projet}

\usepackage[utf8]{inputenc}
\usepackage[T1]{fontenc}
\usepackage[francais]{babel}

\title{Plan de Développement}
\author{Adrien \bsc{Smondack}}
\projet{M1SSI}
\projdesc{Projet de génération d'OTP}
\filiere{M1SSI}
\version{1.0}
\relecteur{Benjamin \bsc{Zigh}}
\signataire{Magali \bsc{Bardet}}
\date{Décembre 2013}

\histentry{1.0}{06/12/2013}{Version initiale.}

\begin{document}
\maketitle

%----------------------------------------------------------------------------------------------------------------------------------------------
\section{Contexte du projet}
	\subsection{Origine du projet}
	Ce projet a été mis en place en réponse à un appel d'offre concernant l'implémentation de système d'authentification via OTP par Magali Bardet et Bruno Macadré.
	
	\subsection{Contexte du développement}
	\begin{description} 
		\item [Cadre :] Dans le cadre du projet annuel de la formation M1SSI;
		\item [Période :] Novembre-Avril;
		\item [Contraintes :] Documents techniques et état de l'art rendus pour Janvier.
	\end{description}
	\subsection{Acteurs}
	\begin{description}
		\item [Émetteur :] Bardet, Macadre;
		\item [Soutien technique :] Mme Bardet, Documentation.
		\item [Coach :] M Abdellah-Godard
	\end{description}
	\subsection{Objectifs poursuivis}
	\begin{itemize}
		\item Fournir un état de l'art sur les systèmes existants;
		\item Implémenter une ou plusieurs de ces méthodes;
		\item Garantir une sécurité forte.
	\end{itemize}
	\subsection{Références} 
	cf. references.pdf ??????????????????????????
\newpage

%----------------------------------------------------------------------------------------------------------------------------------------------
\section{Méthodologie de développement}
	\begin{description}
	    \item [Étape 1 :] État de l'art
		    \begin{description}
		        \item [Objectif :] Se familiariser avec les technologie actuelles et faire le points sur celles que l'on devra employer.
		        \item [Activité :] Étudier les RFC.
		        \item [Produit livrable :] Rapport.
		        \item [Responsabilité :] Toute l'équipe.
            \end{description}
	    \item [Étape 2 :] Implémentation des bibliothèques outils
		    \begin{description}
		        \item [Objectif :] Développer la boite à outils.
		        \item [Activité :] Communication client/serveur, fonctions de hachage, HMAC.
		        \item [Produit livrable :] Documentation.
		        \item [Responsabilité :] Toute l'équipe.
		    \end{description}
	    \item [Étape 3 :] Implémentation des modules OTP
		    \begin{description}
		        \item [Objectif :] Avoir des bibliothèques de calcul et de vérification d'OTP.
		        \item [Activité :] Suivre les algorithmes sélectionnés (conformément aux RFC).
		        \item [Produit livrable :] Documentation.
		        \item [Responsabilité :] Par groupes.
		    \end{description}
	    \item [Étape 4 :] Programmer les tokens (si nécessaire, selon le protocole)
		    \begin{description}
		        \item [Objectif :] Fournir un générateur d'OTP.
		        \item [Activité :] Intégration des outils et des modules de calculs d'OTP dans une IHM.
		        \item [Produit livrable :] Les tokens.
		        \item [Responsabilité :]  Par groupes.
		    \end{description}
	    \item [Étape 5 :] Développer un serveur d'authentification et un client
		    \begin{description}
		        \item [Objectif :] Permettre à un utilisateur de s'authentifier sur un service.
		        \item [Activité :] Intégration des modules de vérification d'OTP.
		        \item [Produit livrable :] Serveur d'authentification + client.
		        \item [Responsabilité :] Par groupes.
	        \end{description}
    \end{description}

%----------------------------------------------------------------------------------------------------------------------------------------------
\section{Organisation et responsabilités}
%Définir l’organisation en présentant
%	L’organigramme (schéma)
%	La description des rôles associés à l’organisation
%	La définition des responsabilités et attributions de chaque intervenant

\begin{tabular}{|p{4.8cm}|p{3.2cm}|p{4cm}|p{4cm}|}
	\hline
	{R\^ole} & {Attribution} & {Description} & {Responsabilité}\\
	\hline
    Chef de projet & Adrien \bsc{Smondack} & &\\
    &&&\\
    MOA & Benjamin \bsc{Zigh} & &\\
    Assistant MOA & Maxime \bsc{Michotte} & &\\
     &&&\\
    Responsable technique & Damien \bsc{Picard} & &\\
    Responsable technique Android & Yves \bsc{Adegoloye} & &\\
     &&&\\
    Responsable qualité & Gaetan \bsc{Ferry} & &\\
    Assist. responsable qualité & Claire \bsc{Hardouin} & &\\
    \hline
\end{tabular}

%----------------------------------------------------------------------------------------------------------------------------------------------
\section{Organigramme des t\^aches}
%A partir de l’arborescence produit et du processus de développement choisi, construire et formaliser un Organigramme des Tâches.
%Faire figurer dans l’OT
%	La liste des produits intermédiaires 
%	La liste des tâches élémentaires associées à la réalisation des produits
%	La liste des moyens matériel et/ou logiciels à mettre en place pour réaliser les produits ou exécuter les tâches

%----------------------------------------------------------------------------------------------------------------------------------------------
\section{évaluation du projet et dimensionnement des moyens}
%Présenter et justifier
%L’évaluation globale de la charge et la répartition par phase 
%Le besoin en moyens et en ressources
%pour la plate-forme de développement (matériels, logiciels et outils)
%pour la plate-forme de tests

%----------------------------------------------------------------------------------------------------------------------------------------------
\section{Planning général}
%Construire et présenter un planning général du projet avec les dates clés du projet et les livraisons prévues.

%N.B. : Un planning détaillé de la première itération sera annexé sous forme de diagramme PERT ou Gantt.

%----------------------------------------------------------------------------------------------------------------------------------------------
\section{Procédés de gestion}
\subsection{Gestion de la documentation}
%Lister les documents à produire pendant le projet en précisant les responsabilités de rédaction, de relecture et d’approbation.
%Définir brièvement  les règles de production et de gestion de la documentation.

\subsection{Gestion des configurations}
%Définir les procédés de gestion à mettre en œuvre pour assurer la maîtrise des configurations pendant le développement  (règles d’identification, organisation des espaces, sauvegardes et archivages, traitement des évolutions, etc.)

%----------------------------------------------------------------------------------------------------------------------------------------------
\section{Revues et points clef}
%Décrire les points clefs prévus dans le planning
%Objectifs
%Calendrier prévisionnel et/ou événements déclenchants
%Objet des vérifications
%Modalités de contrôle
%Intervenants

%----------------------------------------------------------------------------------------------------------------------------------------------
\section{Procédure de suivi et d'avancement}
%Définir les procédures mises en œuvre pendant le projet pour en assurer le suivi
%Suivi interne (processus, modalités, fréquence, etc.)
%Comptes-rendus d’avancement externe (forme, fréquence, destinataires, etc.)
%Réunions prévues

\end{document}
