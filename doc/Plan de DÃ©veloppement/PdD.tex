\documentclass{../../res/univ-projet}

\usepackage[latin1]{inputenc}
\usepackage[T1]{fontenc}
\usepackage[francais]{babel}

\title{Plan de D\'eveloppement}
\author{Adrien \bsc{Smondack}}
\projet{M1SSI}
\projdesc{Projet de g\'en\'eration d'OTP}
\filiere{M1SSI}
\version{1.0}
\relecteur{Benjamin \bsc{Zigh}}
\signataire{Magali \bsc{Bardet}}
\date{D\'ecembre 2013}

\histentry{1.0}{06/12/2013}{Version initiale.}

\begin{document}
\maketitle

%----------------------------------------------------------------------------------------------------------------------------------------------
\section{Contexte du projet}
	Origine du projet : r\'eponse \'a un appel d'offre concernant l'impl\'ementation de syst\`eme d'authentification via OTP par Magali Bardet et Bruno Macadre.
	
	Contexte du d\'eveloppement :
	\begin{description} 
		\item [Cadre :] Dans le cadre du projet annuel de la formation M1GMI;
		\item [P\'eriode :] Novembre-Avril;
		\item [Contrainte :] gestion et \'etat de l'art rendus pour Janvier.
	\end{description}
	Acteurs :
	\begin{description}
		\item [Emetteur :] Bardet, Macadre;
		\item [Soutien technique :] M Google, M man, Mme Bardet
		\item [Coach :] M Abdelah Godard
	\end{description}
	Objectifs poursuivi :
	\begin{itemize}
		\item Fournir un \'etat de l'art sur les syst\'eme existants;
		\item Impl\'ementer un ou plusieurs de ces m\'ethodes;
		\item Garantir une s\'ecurit\'e forte.
	\end{itemize}
	Ref : cf. references.pdf

%----------------------------------------------------------------------------------------------------------------------------------------------
\section{M\'ethodologie de dev}
	<schema de la STB>

	\begin{description}
	    \item [\'Etape 1 :] \'Etat de l'art
		    \begin{description}
		        \item [Objectif :] Se familiariser avec les technologie actuelles et faire le points sur celle qu'on devra employer
		        \item [Activit\'e :] \'etudier les RFC
		        \item [Produit livrable :] Document
		        \item [Responsabilit\'e :] Toute l'\'equipe
            \end{description}
	    \item [\'Etape 2 :] Impl\'ementation des modules d'algo
		    \begin{description}
		        \item [Objectif :] D\'evelopper la tool box
		        \item [Activit\'e :] Comm. client/serveur, fonctions de hachage, hmac
		        \item [Produit livrable :] /
		        \item [Responsabilit\'e :] Par groupes
		    \end{description}
	    \item [\'Etape 3 :] impl\'ementation des modules OTP
		    \begin{description}
		        \item [Objectif :] Avoir des lib de calcul et de vérif d'OTP
		        \item [Activit\'e :] Suivre les algos s\'electionn\'es (comform\'ement aux RFC)
		        \item [Produit livrable :] /
		        \item [Responsabilit\'e :] Par groupes
		    \end{description}
	    \item [\'Etape 4 :] Programmer les tokens (si n\'ecessaire, selon le protocole)
		    \begin{description}
		        \item [Objectif :] Fournir un g\'en\'erateur d'OTP
		        \item [Activit\'e :] Int\'egration des toolbox et des modules de calculs d'OTP dans une UI
		        \item [Produit livrable :] Les tokens
		        \item [Responsabilit\'e :]  Par groupes
		    \end{description}
	    \item [Etape 5 :] D\'evelopper un serveur d'authentification et un client
		    \begin{description}
		        \item [Objectif :] Permettre un user de s'auth. sur un service
		        \item [Activit\'e :] Int\'egration des modules de v\'erification d'OTP
		        \item [Produit livrable :] Serveur d'auth. + clients
		        \item [Responsabilit\'e :] Par groupes
	        \end{description}
    \end{description}

%----------------------------------------------------------------------------------------------------------------------------------------------
\section{Organisation et responsabilit\'es}
%Définir l’organisation en présentant
%	L’organigramme (schéma)
%	La description des rôles associés à l’organisation
%	La définition des responsabilités et attributions de chaque intervenant

\begin{tabular}{|p{4cm}|p{4cm}|p{4cm}|p{4cm}|}
	\hline
	{R\^ole} & {Attribution} & {Description} & {Responsabilit\'e}\\
	\hline
    Chef de projet & Adrien \bsc{Smondack} & \\
    MOA & Benjamin \bsc{Zigh} & \\
    Assistant MOA & Maxime \bsc{Michotte} & \\
    Responsable technique & Damien \bsc{Picard} & \\
    Responsable tech. Android & Yves \bsc{Adegoloye} & \\
    Responsable qualit\'e & Gaetan \bsc{Ferry} & \\
    Assist. responsable qualit\'e & Claire \bsc{Hardouin} & \\
    \hline
\end{tabular}

%----------------------------------------------------------------------------------------------------------------------------------------------
\section{Organigramme des t\^aches}
%A partir de l’arborescence produit et du processus de développement choisi, construire et formaliser un Organigramme des Tâches.
%Faire figurer dans l’OT
%	La liste des produits intermédiaires 
%	La liste des tâches élémentaires associées à la réalisation des produits
%	La liste des moyens matériel et/ou logiciels à mettre en place pour réaliser les produits ou exécuter les tâches

%----------------------------------------------------------------------------------------------------------------------------------------------
\section{\'Evaluation du projet et dimensionnement des moyens}
%Présenter et justifier
%L’évaluation globale de la charge et la répartition par phase 
%Le besoin en moyens et en ressources
%pour la plate-forme de développement (matériels, logiciels et outils)
%pour la plate-forme de tests

%----------------------------------------------------------------------------------------------------------------------------------------------
\section{Planning g\'en\'eral}
%Construire et présenter un planning général du projet avec les dates clés du projet et les livraisons prévues.

%N.B. : Un planning détaillé de la première itération sera annexé sous forme de diagramme PERT ou Gantt.

%----------------------------------------------------------------------------------------------------------------------------------------------
\section{Proc\'ed\'es de gestion}
\subsection{Gestion de la documentation}
%Lister les documents à produire pendant le projet en précisant les responsabilités de rédaction, de relecture et d’approbation.
%Définir brièvement  les règles de production et de gestion de la documentation.

\subsection{Gestion des configurations}
%Définir les procédés de gestion à mettre en œuvre pour assurer la maîtrise des configurations pendant le développement  (règles d’identification, organisation des espaces, sauvegardes et archivages, traitement des évolutions, etc.)

%----------------------------------------------------------------------------------------------------------------------------------------------
\section{Revues et points clef}
%Décrire les points clefs prévus dans le planning
%Objectifs
%Calendrier prévisionnel et/ou événements déclenchants
%Objet des vérifications
%Modalités de contrôle
%Intervenants

%----------------------------------------------------------------------------------------------------------------------------------------------
\section{Proc\'edure de suivi et d'avancement}
%Définir les procédures mises en œuvre pendant le projet pour en assurer le suivi
%Suivi interne (processus, modalités, fréquence, etc.)
%Comptes-rendus d’avancement externe (forme, fréquence, destinataires, etc.)
%Réunions prévues

\end{document}
