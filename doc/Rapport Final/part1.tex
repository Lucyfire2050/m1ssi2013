\chapter{L'authentification}

	Ce chapitre passe en revue le concept global d'authentification et ses 
	différentes formes. Cette partie n'a pas pour but d'être exhaustive mais de 
	rappeler le contexte du projet, pour de plus amples détails nous vous 
	renvoyons à l'état de l'art sur les systèmes d'authentification par OTP édité 
	en Janvier 2014. Celui ci est disponible sur le dépôt de source en ligne du projet.

\section{Qu'est-ce que l'authentification?}

	L'authentification est une mesure de sécurité visant à s'assurer que la 
	personne demandant un service est bien celle qu'elle prétend être. 
	L'authentification auprès d'un système informatique se fait par 
	l'intermédiaire d'une empreinte pouvant prendre plusieurs formes :
	
	\begin{description}
		\item[mémorielle :] Généralement un mot de passe;
		\item[matérielle :] Carte à puce, clé, etc.;
		\item[corporelle :] Empreinte digitale, image rétinienne, vocale, etc;
		\item[réactionnelle :] un geste que la personne peut reproduire (ex: une 
		signature).
	\end{description}

	L'ensemble de ces facteurs peut être combiné pour créer un système 
	d'authentification forte, par exemple token et mot de passe (matérielle / mémorielle)
	ou 	un mot de passe et une empreinte vocale (mémorielle / corporelle).
	
	L'authentification dite forte est basée sur la concaténation d'au moins deux 
	facteurs d'empreinte. On en dénombre trois grandes familles :
	
	\begin{itemize}
		\item One Time Password (OTP);
		\item Certificat numérique;
		\item Biométrie.
	\end{itemize}

\subsection{L'authentification classique}
	
	L'authentification par login \/ mot de passe est la méthode d'authentification
	la plus répandue sur les systèmes informatique. On la retrouve sur la plupart des
	services, qu'il soit hors ligne ou sur un réseau. Le principe repose sur une
  identification classique (echange d'un nom, d'un pseudo ou d'une adresse mail) et le
  partage d'un secret - une information uniquement connue de l'utilisateur et du service
  - généralement un mot de passe.
	
	Cette méthode d'authentification est suffisante pour les systèmes de base peu
	sensibles et ne requérant pas un haut niveau de sécurité.	En revanche le niveau de 
	sécurité offert n'est pas optimal. En effet, outre les méthodes d'attaque basées sur 
	le \emph{social engineering} (hameçonnage, écoute électronique, etc), 
	l'authentification par login et mot de passe est sensible aux attaques dites "par 
	re-jeu" durant laquelle un attaquant récupère le mot de passe d'un utilisateur en vue
	d'une	réutilisation ultérieure, compromettant la sécurité des données ou des ressources
	de l'utilisateur authentique.

\subsection{L'authentification par mot de passe jetable (OTP)}

	Ce type d'authentification est très efficace contre les attaques par re-jeu vues 
	précédemment puisque chaque mot de passe n'est utilisable qu'une seule fois. Ainsi, un 
	attaquant parvenant à observer un mot de passe lors d'une authentification ne pourra
	s'en servir plus tard pour usurper l'identité de l'utilisateur. 
	
	Le principe de cette méthode d'authentification est simple :
	\begin{itemize}
		\item Un utilisateur souhaitant se connecter sur un service via un système 
		OTP dispose d'un outil (\emph{token}, voir plus loin) lui permettant de générer 
		un mot de passe jetable;
		\item Cet outil et le serveur partagent un secret généré lors de la création 
		du compte utilisateur, servant de base pour le calcul des mots de passe jetables;
		\item L'utilisateur renseigne son identifiant et le mot de passe généré;
		\item Le serveur vérifie le mot de passe en effectuant les même calculs que le 
		\emph{token}. S'il trouve le même résultat, l'utilisateur est authentifié.
	\end{itemize}
	
	L'empreinte utilisée ici est un empreinte matérielle, appelée \emph{token}, servant à
	générer les mots de passe jetable. Utilisé tel quel, ce principe ne constitue pas une
	méthode d'authentification forte puisqu'on n'utilise qu'un unique facteur. Pour réparer,
	ce tort, un deuxième type d'empreinte doit être utilisé. On peut par exemple choisir de
	protéger le token par un mot de passe.
	
	La sécurité des mots de passe générés repose sur leur non prédictibilité. En effet,
	un attaquant potentiel doit être incapable de prédire le prochain mot de passe à
	utiliser, même si il à déjà observer plusieurs authentification d'un utilisateur sur 
	un service.

\subsection{L'authentification cryptographique}

	Les systèmes d'authentification dits "cryptographique" représente le niveau supérieur
	d'authentification pour les utilisateur. Ils sont utilisés dans toutes les situations
	ou la sécurité des authentifications est primordiale. On les retrouve donc dans les 
	protocoles sécurisés du web tels que le bien connu HTTPS, reposant sur l'utilisation de
	certificats numériques.
	
	Les système d'authentification cryptographiques doivent vérifier respecter les
	contraintes suivantes \cite{Auth} :
	\begin{itemize}
	  \item[1] si Alice et Bob sont honnêtes, l'authentification d'Alice par Bob doit
	  déboucher sur une acceptation.
	  \item[2] si Charlie écoute des authentifications d'Alice auprès de Bob, il ne doit
	  pas pouvoir se faire passer pour Alice auprès de Bob
	  \item[3] si Charlie est capable de faire exécuter à Alice le protocole 
	  d'authentification (i.e. de faire en sorte qu'Alice s'authentifie auprès de lui),
	  alors il ne peut toujours pas se faire passer pour Alice auprès de Bob.
	\end{itemize}

  Ces contraintes impliquent que les systèmes d'authentification par mot de passe jetable
  (sauf dans certain cas particuliers) ne peuvent pas être considérés comme
  cryptographiques. 
  
  
\subsection{Récapitulatif}

	Les trois méthodes d'authentifications étudiées ci-avant ne fournissent pas un niveau
	équivalent de sécurité. Pour illustrer ces différences, nous pouvons reprendre les
	contraintes nécessaires pour un système cryptographiques. 
	
	Nous avons vu que pour qu'un système soit qualifié de cryptographique il devait 
	respecter trois contraintes. Les systèmes d'authentification cryptographiques respectent
	les trois contraintes. Ce sont les systèmes les plus puissant que nous avons vu. Les
	systèmes à mots de passe jetables respectent quant a eux les deux première contraintes
	uniquement.
	
	En effet, si un attaquant parvient à forcer un utilisateur à s'authentifier auprès de
	lui, il sera en mesure de réutiliser les mots de passes ainsi obtenus pour
	s'authentifier à son tour sur le service authentique. Ces systèmes offre donc un niveau
	moindre de sécurité. Ils sont cependant plus performant que les systèmes
	d'authentification par login / mot de passe qui ne respectent que la première des trois
	contraintes. C'est en fait l'illustration du fait que ce dernier système est sensible 
	aux attaques par re-jeu.
	
	Au final, les systèmes d'authentification par mot de passe jetable que nous avons étudié
	dans le cadre de ce projet, représente un bon compromis entre la faible solution que
	constituent les mots de passe classiques et les systèmes cryptographiques. Ils sont en 
	effet plus sécurisant que les mots de passes standards et plus simple à mettre en place
	que les systèmes cryptographiques.	

\section{Étude approfondie des OTP}

	Plusieurs entités interviendront lors de l'étude des systèmes OTP:
	
	\begin{description}
		\item[L'utilisateur :] le sujet de l'authentification;
		\item[Le client :] un logiciel doté d'une interface utilisateur ou un site 
		web permettant une authentification auprès d'un serveur adapté;
		\item[Le serveur :] l'entité qui permet d'authentifier un utilisateur, 
		contient une base de données contenant des informations utilisateur (login, 
		dernier mot de passe utilisé, ...) et dispose d'une routine permettant de 
		vérifier l'authenticité de celui-ci avec le mot de passe jetable qu'il lui a 
		fourni;
		\item[Le token :] élément physique ou logiciel permettant de générer un mot 
		de passe jetable pour un utilisateur;
		\item[Un attaquant :] Hacker, malware, etc... Une entité cherchant à usurper 
		l'identité de l'utilisateur vis à vis du serveur en récupérant ses 
		informations de connexion ou son accès au service sans ces informations.
	\end{description}

\subsection{OTP aujourd'hui}
	
	Les utilisations de systèmes OTP sont variées, aussi bien dans le domaine 
	personnel que professionnel.
	
	On retrouve des cas concrets d'utilisation dans des logiciels tels que Google 
	Authenticator, OATH Toolkit, LinOTP, DropBox, Microsoft Authenticator, RedHat 
	FreeOT, etc. ainsi que des site web comme Amazon ou GitHub.

\subsection{Les différents types d'OTP}

	Le système OTP de base est maintenant obsolète et donc inutilisé. Cependant de
	nombreuses autres méthodes ont été développées. Parmi celles que nous avons 
	étudié, nous en avons retenu deux : TOTP et HOTP.

\subsection{Comparatif}

	...

\subsection{Conclusion}

	Les méthodes étudiées sont les plus courantes car elles offrent une résistance
	satisfaisante aux attaques exhaustives et par collision et ne présentent 
	aucune faille de sécurité notable du moment que les préconisations sont 
	respectées. De plus, ces méthodes sont très simple d'utilisation, un avantage
	non négligeable pour le non initiés. 
	
	Nous avons donc un bon compromis entre efficacité et simplicité ainsi qu'une 
	pérennité importante.