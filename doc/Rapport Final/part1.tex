\chapter{L'authentification}

	Ce chapitre passe en revue le concept global d'authentification et ses
	différentes formes. Cette partie n'a pas pour but d'être exhaustive mais
	de rappeler le contexte du projet, pour de plus amples détails nous vous
	renvoyons à l'état de l'art sur les systèmes d'authentification par OTP
	édité en Janvier 2014. Celui-ci est disponible sur le dépôt de source en
	ligne du projet.

%------------------------------------------------------------------------------
\section{Qu'est-ce que l'authentification?}

	L'authentification est une mesure de sécurité visant à s'assurer que la
	personne demandant un service est bien celle qu'elle prétend être.
	L'authentification auprès d'un système informatique se fait par
	l'intermédiaire d'un facteur pouvant prendre plusieurs formes :

	\begin{description}
		\item[Mémoriel :] Généralement un mot de passe;
		\item[Matériel :] Carte à puce, clé, etc.;
		\item[Corporel :] Empreinte digitale, rétinienne, vocale, etc;
		\item[Réactionnel :] Un geste que la personne peut reproduire (ex: une
		signature).
	\end{description}

	L'authentification dite forte est basée sur la concaténation d'au moins
	deux facteurs différents (deux mots de passe n'est pas considéré comme une
	authentification forte). On trouvera par exemple des systèmes demandant
	l'emploi d'un jeton et d'un mot de passe (matériel / mémoriel) ou d'un mot
	de passe et d'une empreinte vocale (mémoriel / corporel).

\subsection{L'authentification classique}

	L'authentification par login / mot de passe est la méthode
	d'authentification la plus répandue sur les systèmes informatique. On la
	retrouve sur la plupart des services, qu'ils soientt hors ligne ou sur un
	réseau. Le principe repose sur une identification classique (échange d'un
	nom, d'un pseudo ou d'une adresse mail) et le partage d'un secret - une
	information uniquement connue de l'utilisateur et du service -
	généralement un mot de passe.

	Cette méthode d'authentification est suffisante pour les systèmes de base
	peu sensibles et ne requérant pas un haut niveau de sécurité.	En
	revanche le niveau de sécurité offert n'est pas optimal. En effet, outre
	les méthodes d'attaque basées sur le \emph{social engineering} (
	hameçonnage, écoute électronique, etc), l'authentification par login et
	mot de passe est sensible aux attaques dites "par re-jeu" durant laquelle
	un attaquant récupère le mot de passe d'un utilisateur en vue d'une
	réutilisation ultérieure, compromettant la sécurité des données ou des
	ressources de l'utilisateur authentique.

\subsection{L'authentification par mot de passe jetable (OTP)}

	Ce type d'authentification est très efficace contre les attaques par re-
	jeu vues précédemment puisque chaque mot de passe n'est utilisable qu'une
	seule fois. Ainsi, un attaquant parvenant à observer un mot de passe lors
	d'une authentification ne pourra s'en servir plus tard pour usurper
	l'identité de l'utilisateur.

	Le principe de cette méthode d'authentification est simple :
	\begin{itemize}
		\item Un utilisateur souhaitant se connecter sur un service via un
		système OTP dispose d'un outil (un \emph{token} : voir plus loin) lui
		permettant de générer un mot de passe jetable;

		\item Ce \emph{token} et le serveur partagent un secret généré lors de
		la création du compte utilisateur, servant de base pour le calcul des
		mots de passe jetables;
		\item L'utilisateur renseigne son identifiant et le mot de passe
		généré;
		\item Le serveur vérifie le mot de passe en effectuant les mêmes
		calculs que le \emph{token}. S'il trouve le même résultat,
		l'utilisateur est authentifié.
	\end{itemize}

	Le facteur utilisé ici est le \emph{token}, un facteur matériel servant à
	générer les mots de passe jetables. Utilisé tel quel, ce principe ne
	constitue pas une méthode d'authentification forte puisqu'on n'utilise
	qu'un unique facteur. Pour réparer ce tort, un deuxième type de facteur
	doit être utilisé. On peut par exemple choisir de protéger le token par un
	mot de passe.

	La sécurité des mots de passe générés repose sur leur non prédictibilité.
	En effet, un attaquant potentiel doit être incapable de prédire le
	prochain mot de passe à utiliser, même s'il a déjà observé plusieurs
	authentification d'un utilisateur sur un service.

\subsection{L'authentification cryptographique}

	Les systèmes d'authentification dits "cryptographique" représentent le
	niveau supérieur d'authentification pour les utilisateurs. Ils sont
	utilisés dans toutes les situations où la sécurité des authentifications
	est primordiale. On les retrouve donc dans les protocoles sécurisés du web
	tels que le bien connu HTTPS, reposant sur l'utilisation de certificats
	numériques.

	Les systèmes d'authentification cryptographiques doivent vérifier les
	contraintes suivantes \cite{Auth} :
	\begin{itemize}
	  \item[1] Si Alice et Bob sont honnêtes, l'authentification d'Alice par
	  Bob doit déboucher sur une acceptation.
	  \item[2] Si Charlie écoute des authentifications d'Alice auprès de Bob,
	  il ne doit pas pouvoir se faire passer pour Alice auprès de Bob.
	  \item[3] Si Charlie est capable de faire exécuter à Alice le protocole
	  d'authentification (i.e. de faire en sorte qu'Alice s'authentifie auprès
	  de lui), alors il ne peut toujours pas se faire passer pour Alice auprès
	  de Bob.
	\end{itemize}

  Ces contraintes impliquent que les systèmes d'authentification par mot de
  passe jetable (sauf dans certain cas particuliers) ne peuvent pas être
  considérés comme cryptographiques.

\subsection{Récapitulatif}

	Les trois méthodes d'authentifications étudiées ci-avant ne fournissent
	pas un niveau équivalent de sécurité. Pour illustrer ces différences, nous
	pouvons reprendre les contraintes nécessaires pour un système
	cryptographiques.

	Nous avons vu que, pour qu'un système soit qualifié de cryptographique, il
	devait respecter trois contraintes. Les systèmes d'authentification
	cryptographiques respectent les trois contraintes. Ce sont les systèmes
	les plus puissants que nous avons vu. Les systèmes à mots de passe
	jetables respectent quant à eux les deux premières contraintes uniquement.

	En effet, si un attaquant parvient à forcer un utilisateur à
	s'authentifier auprès de lui, il sera en mesure de réutiliser les mots de
	passes ainsi obtenus pour s'authentifier à son tour sur le service
	authentique. Ces systèmes offrent donc un niveau moindre de sécurité. Ils
	sont cependant plus performant que les systèmes d'authentification par
	login / mot de passe qui ne respectent que la première des trois
	contraintes. C'est en fait l'illustration du fait que ce dernier système
	est sensible aux attaques par re-jeu.

	Au final, les systèmes d'authentification par mot de passe jetable que
	nous avons étudié dans le cadre de ce projet représentent un bon compromis
	entre la faible solution que constituent les mots de passe classiques et
	les systèmes cryptographiques. Ils sont en effet plus sécurisant que les
	mots de passe standards et plus simple à mettre en place que les systèmes
	cryptographiques, tout en prevenant la plupart des attaques passives.

%------------------------------------------------------------------------------
\section{Étude approfondie des OTP}

	Plusieurs entités interviendront lors de l'étude des systèmes OTP:

	\begin{description}
		\item[L'utilisateur :] le sujet de l'authentification;
		\item[Le client :] un logiciel doté d'une interface utilisateur ou un
		site web permettant une authentification auprès d'un serveur adapté;
		\item[Le serveur :] l'entité qui permet d'authentifier un utilisateur,
		contient une base de données contenant des informations utilisateur (
		login, dernier mot de passe utilisé, ...) et dispose d'une routine
		permettant de vérifier l'authenticité de celui-ci avec le mot de passe
		jetable qu'il lui a fourni;
		\item[Le token :] élément physique ou logiciel permettant de générer
		un mot de passe jetable pour un utilisateur;
		\item[Un attaquant :] hacker, malware, etc... Une entité cherchant à
		usurper l'identité de l'utilisateur vis à vis du serveur en récupérant
		ses informations de connexion ou son accès au service sans ces
		informations.
	\end{description}

\subsection{OTP aujourd'hui}

	Les utilisations de systèmes OTP sont variées, aussi bien dans le domaine
	personnel que professionnel. La popularité croissante des systèmes OTP
	s'explique par sa simplicité de mise en œuvre et son efficacité en terme
	de sécurité.

	On retrouve des cas concrets d'utilisation dans des sites web comme Amazon
	ou GitHub ainsi que dans des logiciels tels que :

	\begin{itemize}
		\item Google Authenticator;
		\item OATH Toolkit;
		\item LinOTP;
		\item DropBox;
		\item Microsoft	Authenticator;
		\item RedHat FreeOT;
		\item etc.
	\end{itemize}

\subsection{Les différents types de protocole OTP}

	Le protocole d'OTP de base date de 1995. Depuis, de nombreuses autres
	implémentations ont été développées, parmi lesquelles :

	\begin{description}
		\item[OTP :] Protocole standard basé sur l'utilisation de fonctions de
		hachage;
		\item[HOTP :] Extension du principe standard d'OTP reposant sur
		l'utilisation du mécanisme d'authentification de message HMAC, utilise
		un compteur de synchronisation lors du calcul des mots de passe
		jetables;
		\item[TOTP :] Système similaire à HOTP, emploit le temps comme
		compteur;
		\item[OTPW :] Les mots de passe sont précalculés, l'authentification
		se fait par question/réponse;
	\end{description}

\subsection{Comparatif}
\subsubsection{Fonctionnement}
	%tableau

\subsubsection{Sécurité}
	OTP : Pas de faille connue, outre les attaques actives de type espionnage
	et brute force.

	HOTP : Aucune faille connue, mis à part celles héritées d'OTP.

	TOTP : Idem, les attaques de type brute force sont fortement limité à cause du quantum et du time-out.

	OTPW : préfixe facilement identifiable.

\subsubsection{Pérennité}

	OTP standard : Système obsolète, remplacé par des protocoles plus simples
	et performants
	HOTP : Pas de faille de sécurité référencée, le système de calcul est
	pérenne.
	TOTP : perennité équivalente à celle de HOTP car basée sur ce dernier.
	OTPW : la perennité du système est équivalente à la fonction de hachage.
	Cependant, nécessite une intervention manuelle pour réinitialiser le
	système.


\subsection{Conclusion}

	-- faire un point sur les avantages et inconvenient de chacun --

	Les méthodes étudiées sont les plus courantes car elles offrent une
	résistance satisfaisante aux attaques exhaustives et ne présentent aucune
	faille de sécurité notable. De plus, ces méthodes sont très simple
	d'utilisation, un avantage non négligeable pour le non initiés.

	Nous avons donc un bon compromis entre efficacité et simplicité ainsi
	qu'une pérennité importante.

%------------------------------------------------------------------------------