\chapter{L'authentification}

	Ce chapitre passe en revue le concept global d'authentification et ses 
	différentes formes. Cette partie n'a pas pour but d'être exhaustive mais de 
	rappeler le contexte du projet, pour de plus amples détails nous vous 
	renvoyons à l'état de l'art sur les systèmes d'authentification par OTP édité 
	en Janvier 2014.

\section{Qu'est-ce que l'authentification?}

	L'authentification est une mesure de sécurité visant à s'assurer que la 
	personne demandant un service est bien celle qu'elle prétend être. 
	L'authentification auprès d'un système informatique se fait par 
	l'intermédiaire d'une empreinte pouvant prendre plusieurs formes :
	
	\begin{description}
		\item[mémorielle :] Généralement un mot de passe;
		\item[matérielle :] Carte à puce, clé, etc.;
		\item[corporelle :] digitale, rétinienne, vocale, etc;
		\item[réactionnelle :] une geste que la personne peut reproduire (ex: une 
		signature).
	\end{description}

	L'ensemble de ces facteurs peut être combiné pour créer un système 
	d'authentification forte, par exemple token et mot de passe ou un mot de passe et
	une empreinte vocale.
	
	L'authentification dite forte est basée sur la concaténation d'au moins deux 
	facteurs d'empreinte. On en dénombre trois grandes familles :
	
	\begin{itemize}
		\item One Time Password (OTP);
		\item Certificat numérique;
		\item Biométrie.
	\end{itemize}

\subsection{L'authentification classique}
	
	L'authentification par login \/ mot de passe est la méthode d'authentification
	la plus répandue sur les systèmes informatique, en particulier sur les 
	services Web. Le principe reposant sur une identification classique (je donne
	mon nom, mon pseudo ou mon adresse mail) et le partage d'un secret (une 
	information partagée entre moi et le serveur). généralement un mot de passe.
	
	Cette méthode d'authentification est relativement fiable pour les systèmes de 
	base. En revanche elle n'est pas infaillible. Outre les méthodes d'usurpation 
	d'identité basée sur le \emph{social engineering} (hameçonnage, écoute 
	électronique, etc), les attaques dites "par re-jeu" permettent de récupérer le
	mot de passe d'un utilisateur en vue d'une utilisation ultérieure. Ceci 
	compromet sensiblement la sécurité des données de l'utilisateur authentique ou
	de ses ressources.

\subsection{L'authentification par mot de passe jetable (OTP)}

	Ce type d'authentification est très efficace contre les attaques par re-jeu 
	puisque chaque mot de passe n'est utilisable qu'une seule fois. Ainsi, un 
	attaquant parvenant à récupérer un mot de passe ne pourra s'en servir plus 
	tard pour usurper l'identité de l'utilisateur.
	
	Le principe est simple :
	\begin{itemize}
		\item Un utilisateur souhaitant se connecter sur un service via un système 
		OTP dispose d'un outil lui permettant de générer un mot de passe jetable;
		\item Cet outil et le serveur partagent un secret généré lors de la création 
		du compte utilisateur, servant de base pour le calcul du mot de passe jetable;
		\item L'utilisateur renseigne son login et le mot de passe;
		\item Le serveur vérifie le mot de passe en ré-appliquant les calculs 
		précédents. S'il trouve le même résultat, l'utilisateur est authentifié.
	\end{itemize}
	
	La sécurité des mots de passe générés repose sur la non-réversibilité de la 
	fonction de hachage utilisée, rendant la prédiction des prochains mots de 
	passe extrêmement difficile.

\subsection{L'authentification cryptographique}

	Authentification par certificat...

\subsection{Récapitulatif}

	...
	
	Les systèmes OTP garantissent une protection efficace contre toute tentatives 
	d'usurpation d'identité. En revanche ils ne protègent pas l'utilisateur contre
	d'autres types d'attaque tels que le vol de session ou l'espionnage.
	
	...
	

\section{Étude approfondie des OTP}

	Plusieurs entités interviendront lors de l'étude des systèmes OTP:
	
	\begin{description}
		\item[L'utilisateur :] le sujet de l'authentification;
		\item[Le client :] un logiciel doté d'une interface utilisateur ou un site 
		web permettant une authentification auprès d'un serveur adapté;
		\item[Le serveur :] l'entité qui permet d'authentifier un utilisateur, 
		contient une base de données contenant des informations utilisateur (login, 
		dernier mot de passe utilisé, ...) et dispose d'une routine permettant de 
		vérifier l'authenticité de celui-ci avec le mot de passe jetable qu'il lui a 
		fourni;
		\item[Le token :] élément physique ou logiciel permettant de générer un mot 
		de passe jetable pour un utilisateur;
		\item[Un attaquant :] Hacker, malware, etc... Une entité cherchant à usurper 
		l'identité de l'utilisateur vis à vis du serveur en récupérant ses 
		informations de connexion ou son accès au service sans ces informations.
	\end{description}

\subsection{OTP aujourd'hui}
	
	Les utilisations de systèmes OTP sont variées, aussi bien dans le domaine 
	personnel que professionnel.
	
	On retrouve des cas concrets d'utilisation dans des logiciels tels que Google 
	Authenticator, OATH Toolkit, LinOTP, DropBox, Microsoft Authenticator, RedHat 
	FreeOT, etc. ainsi que des site web comme Amazon ou GitHub.

\subsection{Les différents types d'OTP}

	Le système OTP de base est maintenant obsolète et donc inutilisé. Cependant de
	nombreuses autres méthodes ont été développées. Parmi celles que nous avons 
	étudié, nous en avons retenu deux : TOTP et HOTP.

\subsection{Comparatif}

	...

\subsection{Conclusion}

	Les méthodes étudiées sont les plus courantes car elles offrent une résistance
	satisfaisante aux 	attaques exhaustives et par collision et ne présentent 
	aucune faille de sécurité notable du moment que les préconisations sont 
	respectées. De plus, ces méthodes sont très simple d'utilisation, un avantage
	non négligeable pour le non initiés. 
	
	Nous avons donc un bon compromis entre efficacité et simplicité ainsi qu'une 
	pérennité importante.