\chapter{Introduction}

\section{La sécurité informatique}

La sécurité est aujourd'hui au cœur des problématiques du monde de l'informatique. Au cours des dernières années, de nombreuses grandes firmes ont subi de lourdes pertes car elles avaient sous-estimé les menaces qui pesaient sur leur réseau. Sony\textregistered, par exemple, a subi 170 millions de dollar\cite{PSN} de pertes suite au scandale du Playstation Network\texttrademark en 2011, au cours duquel des millions d'utilisateurs à travers le monde ont eu leurs informations personnelles et notamment bancaires volées par des individus malveillants.

 Les différents scandales politiques récents ont également aidé à sensibiliser le grand public à l'importance de la protection de la vie privée, le spectre de la NSA plane sur le monde entier et l'utilisateur moyen a perdu une partie de la confiance placée dans les télécommunications en général. Les postes d'expert en sécurité sont désormais légion. Ce secteur en pleine expansion regroupe une multitude de spécialités, et offre un panel variés d'activités à qui ose s'y intéresser.
 
La cryptographie, l'administration système, la sécurité organisationnelle, sont autant de spécialités pour lesquelles la demande est de plus en plus forte dans toutes les grandes entreprises.

\section{L'équipe}

Nous sommes sept étudiants de l'université de Rouen, faculté des sciences et techniques. Après une licence informatique générale, nous avons fait le choix de nous spécialiser dans la sécurité de systèmes informatiques.

 Durant notre première année de Master, il nous a été demandé de réaliser le projet que nous présentons ici dans le cadre du module de gestion de projet. Pour ce faire, nous devions simuler un fonctionnement d'entreprise, avec des rôles bien définis dans l'équipe, et un client potentiellement extérieur à l'université.
 
 
\section{Le projet}


\subsection{Les clients}
Pour notre cas, les clients appartiennent tous les deux à l'université de Rouen. Magali Bardet, Enseignante-chercheuse et responsable du Master SSI, et Bruno Macadré, Ingénieur système, ont proposé conjointement un projet de module d'authentification à base de mot de passe jetable, qui nous a particulièrement intéressé.
\newpage

\subsection{Le sujet}

Le sujet comportait deux parties distinctes: 
\begin{itemize}
\item La réalisation d'un état de l'art des technologies existantes.
\item La création d'un module d'authentification prenant en compte les résultats du dit état de l'art.
\end{itemize}

L'état de l'art devait comporter au moins 3 technologies actuellement employées et l'implémentation devait être compatible avec les systèmes utilisés couramment.
\newline

Ce rapport décrira dans un premier temps les grands principes nécessaires à la compréhension du sujet. Nous présenterons ensuite la méthode appliquée par l'équipe pour rédiger le rapport de recherche, puis programmer les diverses applications. 