\chapter{Introduction}

\section{La sécurité informatique}

La sécurité est aujourd'hui au cœur des problématiques du monde de l'informatique. Au cours des dernières années, de nombreuses grandes firmes ont subi de lourdes pertes car elles avaient sous-estimé les menaces qui pesaient sur leur réseau. Sony\textregistered, par exemple, a subit 170 millions de dollar\cite{PSN} de pertes suite au scandale du Playstation Network\texttrademark en 2011, au cours duquel des millions d'utilisateurs à travers le monde ont eu leurs informations personnelles et notamment bancaires volées par des individus malveillants.

 Les différents scandales politiques récents ont également aidé à sensibiliser le grand public à l'importance de la protection de la vie privée, le spectre de la NSA plane sur le monde entier et l'utilisateur moyen a perdu une partie de la confiance placée dans les télécommunications en général. Les postes d'expert en sécurité sont désormais légion. Ce secteur en pleine expansion regroupe une multitude de spécialités, et offre un panel variés d'activités à qui ose s'y intéresser.
 
La cryptographie, l'administration système, la sécurité organisationnelle, sont autant de spécialités pour lesquelles la demande est de plus en plus forte dans toutes les grandes entreprises.

\section{L'équipe}

Nous sommes sept étudiants de l'université de Rouen, faculté des sciences et techniques. Après une licence informatique générale, nous avons fait le choix de nous spécialiser dans la sécurité de systèmes informatiques.

 Durant notre première année de Master, il nous a été demandé de réaliser le projet que nous présentons ici.
 
 
\section{Le sujet}