\chapter{Mise en œuvre}
\section{La demande du client}


\section{La gestion de projet dans le projet}

\subsection{Communication au sein du projet}

Dans un projet de cette importance et menait à bien par autant de personnes, une bonne communication est essentielle.\\
Nous avons donc organisé des réunions de groupes toutes les 2 semaines ce qui permettais de maintenir une même vision du projet, la circulation de certaines informations, ainsi que la cohésion de l'équipe.\\
Le groupe étant divisé en équipe de nombreuses réunions d'équipe ont aussi eu lieue.\\
Des systèmes d'échanges d'informations ont aussi été mis en place comme le système de tickets (sur le git).\\

\subsection{Méthodologie employée}

Le développement de ce projet s'est fait selon les méthode Extrem programming et scrum pour la plus grande partie. Ces 2 méthodes ont aidé le groupe a travailler de façon cohérente les uns par rapport aux autres.\\

Les tests étaient réalisés en amont du développement (Test Driven Development), nous obtenions moins d'érreures avec cette méthode.\\

\subsection{Suivie du projet}
Des réunions régulières avec le client (toutes les 2 semaines) ont permis de faire évoluer le projet de façon conforme, c'est à dire en accord avec les besoins des clients. 
Le suivis était, pour la plus grande partie réalisé par les responsables de gestion client.\\

\subsection{Les documents de gestion de projet}

La délimitation des tâches a éffectuer étant formalisée par les différents documents de gestion de projet, nous a permis un découpage éfficace et donc d'avancer plus vite sur le développement.\\

Le diagramme de Gantt est d'une grande aide sur un projet comme le nôtre : on y voit clairement la situation de chacun par rapport au projet. L'avancement du projet y apparait aussi clairement.\\

Les documents ont donc étaient particulièrement utiles notamment l' ``analyse des risques'' dont le formalisme à rendu les sorties de ``crises'' plus simples.\\

\subsection{La gestion de projet, conclusion}

Tout au long du projet la gestion de projet a permis une meilleure coordination au sein de l'équipe, de limiter la perte de temps dû aux disgrétions.\\



\section{Le Module PAM}
%faire les sous sections (voir Damien)

\section{L'App Android}
%faire les sous sections
