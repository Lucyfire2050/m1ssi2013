\chapter{Conclusion}

\section{Enseignements tirés du projet}

\subsection{OTP}
\subsection{Outils cryptographiques}
\subsection{PAM} 
\subsection{Android}
\subsection{Gestion de projet}

%A rédiger




\section{Améliorations possibles}
\subsection{HOTP en question/réponse}
Utilisation d'un aléas à la place du compteur incrémental.

\subsection{Token sur Java Card}
Peut-être en M2...
%A rédiger



\section{Conclusion}

Cette première expérience de projet avec une équipe conséquente (7 personnes) nous a permis de découvrir et de comprendre l'utilité des méthodologies de gestion de projet.

En effet, la coordination des membres sur un projet de cet ampleur ne peut pas être laissée au hasard. Le rôle de chacun doit être défini et limité, et un point doit être fait régulièrement sur l'avancement de chaque partie du projet.

En cela, les méthodologies Agile prennent tout leur sens, et représentent un acquis important pour notre future vie professionnelle.
\begin{itemize}
\item Un projet dans l'ère du temps.
\item Un panel de connaissances variées.
\item Une première expérience de gestion de projet.
\end{itemize}
%A rédiger
