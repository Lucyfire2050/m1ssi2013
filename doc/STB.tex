\documentclass{"../res/univ-projet"}
\usepackage[latin1]{inputenc}
\usepackage[T1]{fontenc}
\usepackage[francais]{babel}

\begin{document}

\title{Sp�cification Technique des Besoins}
\author{}
\date{Novembre 2013}

%-------------------------------------------------------------------------------
\section*{Objet}
Etude et impl�mentations des syst�mes d'auth via mdp jetable :
\begin{itemize}
Besoin op.
\begin{itemize}
\item etat de l'art sur le syst existant
\item fournir un logiciel exploitable impl�mentant ce syst
\item objectif : mise en prod
\end{itemize}
Obj tech
\begin{itemize}	
\item mise en place �tat de l'art sur certain (voir tous les) syst
\item Impl�mentation
\end{itemize}
Contraintes et recom :
\begin{itemize}
\item Respecter les sp�cif de base
\item Syst secure (le prouver)
\item Choisir une impl�mentation adapt�e en fonction des besoin et de l'�tat de l'art �tablie
\end{itemize}
R�sultats attendus
\begin{itemize}
\item Etat de l'art le + exhaustif possible
\item Impl. sur C�P et/ou Android
\end{itemize}
\end{itemize}
%-------------------------------------------------------------------------------
\section*{Doc. appl et ref}
	- Les rfc 2289, 4226, 4256, 4793, 6238
	- Google auth

%-------------------------------------------------------------------------------
\section*{Terminologie et sigle}
	- 
%-------------------------------------------------------------------------------
\section*{Exigence fonct}
	Id	|	Intitul�			|	Acteur	|	Priorit�\\
	---------------------------------------------------------\\
	1	|	Etat de l'art		|	Equipe	|	Indispensable\\
	2	|	Association			|	Client	|	Indispensable\\
	3	|	G�n�ration			|	Client	|	Indispensable\\
	4	|	Authentification	|	Client	|	Indispensable\\
	5	|	Re-synchronisation	|	Equipe	|	Indispensable\\
	
(Pour les fiches prenez vos notes)
%-------------------------------------------------------------------------------
\section*{Exigences op�rationnelles}
\begin{itemize}
\item Respecter les RFC
\item ...
\end{itemize}

%-------------------------------------------------------------------------------
\section*{Exigences d'interface}

%-------------------------------------------------------------------------------
\section*{Exigences de qualit�}

%-------------------------------------------------------------------------------
\section*{Exigences de r�alisation}

%-------------------------------------------------------------------------------
\end{document}