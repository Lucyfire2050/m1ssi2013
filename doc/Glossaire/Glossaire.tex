\documentclass{"../../res/univ-projet"}
\usepackage[utf8]{inputenc}
\usepackage[T1]{fontenc}
\usepackage[francais]{babel}
\usepackage{colortbl}
\usepackage{algorithm}
\usepackage{algorithmic}


\logo{../../res/logo_univ.png}
\title{Terminologie et sigles}
\author{\bsc{Michotte} Maxime}
\projet{M1SSI}
\projdesc{Projet de génération d'OTP}
\filiere{M1SSI}
\version{1.1}
\relecteur{\bsc{Zigh} Benjamin}
\signataire{\bsc{Bardet} Magali}
\date{Décembre 2013}

\histentry{1.0}{15/12/2013}{Version relue et corrigée}
\histentry{0.1}{02/12/2013}{Premier jet}


\begin{document}
\maketitle
%-------------------------------------------------------------------------------
%-------------------------------------------------------------------------------    
\section{Glossaire}
\begin{description}
	\item[État de l'art]
  	Dresser un état de l'art dans un domaine consiste à rechercher toutes 
  	les informations existantes concernant ce domaine et à en faire une synthèse.
	\\
	\item[RFC]
  	Les RFC (Request For Comments) sont un ensemble de documents qui font 
  	référence auprès de la Communauté Internet et qui décrivent, 
  	spécifient, aident à l'implémentation, standardisent et débattent de 
  	la majorité des normes, standards, technologies et protocoles liés à 
  	Internet et aux réseaux en général.
	\\
	\item[Utilisateur]
  	L'utilisateur est une personne physique qui dans notre cas souhaite utiliser 
  	une authentification utilisant les OTP.
	\\
	\item[Mot de passe]
  	Le mot de passe est une méthode parmi d'autres pour effectuer une
  	authentification, c'est-à-dire vérifier qu'une personne correspond bien à 
  	l'identité déclarée. Il s'agit d'une preuve que l'on possède et que 
  	l'on transmet au service chargé d'autoriser l'accès. Le mot de passe doit être 
  	tenu secret pour éviter qu'un tiers non autorisé puisse accéder à 
  	la ressource ou au service.  
  	\\
	\item[OTP]
  	Un Mot de passe unique ou OTP (One-time password) est un mot de passe qui 
  	n'est valable que pour une session ou une transaction.
	\\
	\item[Token]
  	Un token désigne dans notre projet un élément logiciel ou matériel 
  	tiers servant à la génération d'un OTP.
  	\\
	\item[Serveur]
  	Un serveur est un dispositif informatique matériel ou logiciel qui offre 
  	des services, à différents clients. Pour notre cas celui-ci permettra 
  	l'association d'un token, la vérification de l'OTP généré par 
  	le token, et l'éventuelle re-synchronisation du token.
 	\\ 
	\item[Client]
  	Un client est le logiciel qui envoie des demandes à un serveur.Pour notre 
  	cas celui-ci permettra de communiquer au serveur les OTP générés 
  	par le token.
   	\\ 
    \item[Seed] Variable utilisée pour initialiser une séquence aléatoire.
    \\
    \item[Vérifieur] Un vérifieur est une entité qui pour un mécanisme 
    d'authentification de type défi/réponse propose un défi au prouveur et pour une authentification simple vérifie la validité du mot de passe fourni; dans notre
    cas il s'agit du serveur d'authentification.
    \\
    \item[Prouveur] Un prouveur est une entité qui pour un mécanisme 
    d'authentification de type défi/réponse assure au vérifieur qu'il est bien celui qu'il
    prétend être en répondant au défi via la détention d'un secret; dans notre cas,
    il s'agit du client.
\end{description}

%-------------------------------------------------------------------------------
\end{document}
