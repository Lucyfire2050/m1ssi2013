\documentclass{"../../res/univ-projet"}
\usepackage[utf8]{inputenc}
\usepackage{array}
\usepackage[T1]{fontenc}
\usepackage[francais]{babel}
\usepackage{textcomp}
\usepackage[toc,page]{appendix} 
\usepackage{listings}

\logo{../../res/logo_univ.png}
\title{Rapport de tests}
\author{\bsc{Ferry} Gaetan}
\projet{M1SSI}
\projdesc{Projet de génération d'OTP}
\filiere{M1SSI}
\version{1.0}
\signataire{\bsc{Bardet} Magali}
\date{Avril 2014}

\histentry{1.0}{29/11/2013}{Version initiale.}

\begin{document}
  \maketitle
  
	\section{Validation du temps de génération}
	L'exigence du client concernant le temps de génération des mots de passes
	jetables était que celui ci ne devait pas dépasser la seconde.
	La procédure de test numéro 1 décrite dans le cahier de recette devait
	vérifier cette contrainte.
	
	Ce test est exécute comme prévu sur un appareil Android aux performances
	minimales. On obtient, pour 100 mots de passes jetables, un temps de
	génération de 52 ms.
	
	L'exigence du client est donc respectée.
	\begin{center}	
		\begin{tabular}{| p{6cm} | p{6cm} | p{2cm} |}
			\hline
			\cellcolor{lightgray} Exigence & \cellcolor{lightgray} Résultat & 
			\cellcolor{lightgray} Validité \\ \hline
		
			Temps pour 100 mots de passes : 100 s & Temps pour 100 mots de passes : 52 ms & 
			\cellcolor{green} OK \\ \hline
		\end{tabular}
	\end{center}
	
  
	\section{Temps de vérification du module}  
    L'exigence du client concernant le temps de vérification des mots de passes par le
    module PAM était que celui ci ne devait pas dépasser 2 secondes. La procédure de test
    numéro 2 devait vérifier cette contrainte.
    
    Ce test est exécuté comme prévu sur une machine personnelle munie d'une architecture
    grand publique. On obtient un temps de vérification inférieur au 1/100 s.
    
    L'exigence du client est donc respectée.
	\begin{center}	
		\begin{tabular}{| p{6cm} | p{6cm} | p{2cm} |}
			\hline
			\cellcolor{lightgray} Exigence & \cellcolor{lightgray} Résultat & 
			\cellcolor{lightgray} Validité \\ \hline
		
			Temps de vérification : 2 s & Temps de vérification : 1/100 s & 
			\cellcolor{green} OK \\ \hline
		\end{tabular}
	\end{center}
    
	\section{Quantité de mémoire utilisée pour la génération}    
	L'exigence du client concernant la quantité de mémoire nécessaire à la génération du mot de passe
	était que celle ci ne devait pas dépasser 10 Ko. La procédure de test numéro 4 devait vérifier
	cette contrainte.
	
	Ce test est exécuté sur l'appareil Android de test. On obtient une consommation mémoire pour une 
	génération de mot de passe en dehors du contexte de l'application, dans lequel la présence des
	templates d'activity charge plus avant la mémoire, une consommation mémoire de 7 Ko.
	
	L'exigence du client est donc respectée.
	\begin{center}	
		\begin{tabular}{| p{6cm} | p{6cm} | p{2cm} |}
			\hline
			\cellcolor{lightgray} Exigence & \cellcolor{lightgray} Résultat & 
			\cellcolor{lightgray} Validité \\ \hline
		
			Mémoire utilisée : 10 Ko & Mémoire utilisée : 7Ko & \cellcolor{green} OK \\ \hline
		\end{tabular}
	\end{center}
	
	\section{Utilisation unique des mots de passe}
	Le système d'authentification par mot de passe jetable impose qu'un mot de passe ne peut pas être
	utilisé plus d'une fois. La procédure de test numéro 5 devait vérifier le respect de ce principe.
	
	Ce test est exécute sur la machine grand public via le module PAM mis en place sur un serveur SSH. On
	vérifie ainsi que la réutilisation des mots de passe, et donc les attaques par rejeu, n'est pas possible
	avec notre méthode d'authentification.
	
	Le principe du système est donc respecté.
	\begin{center}	
		\begin{tabular}{| p{6cm} | p{6cm} | p{2cm} |}
			\hline
			\cellcolor{lightgray} Exigence & \cellcolor{lightgray} Résultat & 
			\cellcolor{lightgray} Validité \\ \hline
		
			Pas de réutilisation possible& Pas de réutilisation possible & \cellcolor{green} OK \\ \hline
		\end{tabular}
	\end{center}
    
	\section{Resynchronisation des utilisateurs}
	Les spécifications des méthodes d'authentification par mot de passe jetable implantées indiquent qu'un
	utilisateur doit pouvoir utiliser un mot de passe qui n'est pas celui attendu si celui se situe dans un 
	intervalle limité de mots de passes passés (non utilisés) ou a venir. Ce système garantit le confort
	d'utilisation du système. La procédure numéro 6 devait vérifier le respect de ce principe.
	
	Ce test est exécute sur la machine grand public via le module PAM mis en place sur un serveur SSH. On 
	vérifie alors que la fonction de resynchronisation est bien fonctionnelle sur notre module de vérification.

	Cette spécification est donc respectée.
	\begin{center}	
		\begin{tabular}{| p{6cm} | p{6cm} | p{2cm} |}
			\hline
			\cellcolor{lightgray} Exigence & \cellcolor{lightgray} Résultat & 
			\cellcolor{lightgray} Validité \\ \hline
		
			Resynchronisation possible & Resynchronisation possible & \cellcolor{green} OK \\ \hline
		\end{tabular}
	\end{center}
	
	\section{Résistance aux attaques exhaustives}
	Afin d'assurer la sécurité du système d'authentification, il doit être improbable de s'authentifier sur le 
	service grâce à une attaque par force brute. La procédure numéro 7 devait vérifier cette résistance. Les
	deux méthodes d'OTP utilisées ne se	comportant pas de la même manière du point de vu de la force brute, le
	test à été exécute indépendamment sur chacune d'entre elles.
	
	Ce test est exécute sur la machine grand public via le module PAM mis en place sur un serveur SSH. Malgré 
	tous les efforts mis en œuvre pour s'authentifier par force brute n'ont pas aboutis.
	
	L'exigence du client est donc respectée.
	\begin{center}	
		\begin{tabular}{| p{4cm} | p{4cm} | p{4cm} | p{2cm} |}
			\hline
			\cellcolor{lightgray} Exigence & \cellcolor{lightgray} Résultat OTP & 
			\cellcolor{lightgray} Résultat OTP & \cellcolor{lightgray} Validité \\ \hline
		
			Attaque exhaustive improbable & Attaque exhaustive échouée & Attaque exhaustive échouée &
			\cellcolor{green} OK \\ \hline
		\end{tabular}
	\end{center}
    
	\section{Non prédictibilité des mots de passes jetables}
	Afin d'assurer la sécurité du système d'authentification, il faut qu'un attaquant potentiel ne puisse pas deviner
	l'OTP à fournir, même en ayant observer un nombre quelconque d'authentification précédente. La procédure numéro 8
	devait vérifier cette contrainte de sécurité.
	
	Ce test peut être exécuté sur n'importe quelle machine pouvant compiler le programme de génération définit dans le
	cahier de recette et installer et exécuter le programme dieharder. On obtient un taux de réussite aux tests 
	d'aléatoire de 71%.
	
	L'exigence de sécurité est donc vérifiée.
    
    \begin{center}	
		\begin{tabular}{| p{6cm} | p{6cm} | p{2cm} |}
			\hline
			\cellcolor{lightgray} Exigence & \cellcolor{lightgray} Résultat & 
			\cellcolor{lightgray} Validité \\ \hline
		
			Taux de réussite aux tests : 70\% & Taux de réussite aux tests : 71\% & \cellcolor{green} OK \\ \hline
		\end{tabular}
	\end{center}
	
	\section{Association des utilisateurs}
	Une des étapes essentielles au fonctionnement du système d'authentification par mot de passe jetable est
	l'association d'un utilisateur au service requérant une authentification. Cette fonctionnalité devait être validée
	grâce à la procédure numéro 9.
	
	Le test est réalisé sur la machine grand publique via le module PAM installé sur le service SSH. Il permet de 
	vérifier que la fonction d'association est fonctionnelle.
	
	L'exigence de fonctionnalité est vérifiée.
    
    \begin{center}	
		\begin{tabular}{| p{6cm} | p{6cm} | p{2cm} |}
			\hline
			\cellcolor{lightgray} Exigence & \cellcolor{lightgray} Résultat & 
			\cellcolor{lightgray} Validité \\ \hline
		
			Association possible & Association possible & \cellcolor{green} OK \\ \hline
		\end{tabular}
	\end{center}
	
\end{document}
