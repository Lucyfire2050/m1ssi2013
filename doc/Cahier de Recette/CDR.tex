\documentclass{"../../res/univ-projet"}
\usepackage[utf8]{inputenc}
\usepackage{array}
\usepackage[T1]{fontenc}
\usepackage[francais]{babel}
\usepackage{textcomp}
\usepackage[toc,page]{appendix} 
\usepackage{listings}

\logo{../../res/logo_univ.png}
\title{Cahier de recettes}
\author{Gaetan \bsc{Ferry}, Claire \bsc{Hardouin}}
\projet{M1SSI}
\projdesc{Projet de génération d'OTP}
\filiere{M1SSI}
\version{2.1}
\relecteur{Maxime \bsc{Michotte}}
\signataire{\bsc{Bardet} Magali}
\date{Avril 2014}

\histentry{2.1}{08/04/2014}{Explicitation des tests d'aléatoirité}
\histentry{2.0}{20/02/2014}{Version post état de l'art.}
\histentry{1.2}{16/01/2014}{Version pour la revue de lancement.}
\histentry{1.1}{15/12/2013}{Version relue et corrigée.}
\histentry{1.0}{29/11/2013}{Version initiale.}

\begin{document}
  \maketitle
  \section{Introduction}
  Fonctionnalités du logiciel :
  \begin{itemize}
    \item Token : 
    \begin{itemize}
      \item Générer des mots de passe jetables.
    \end{itemize}

    \item Module PAM :
    \begin{itemize}
      \item Se brancher sur n'importe quel serveur utilisant PAM.
      \item Authentifier des utilisateurs.
      \item Stocker et ajouter des paramètres utilisateurs.
      \item Gérer la synchronisation des deux parties.
    \end{itemize}
  \end{itemize}

  Les objets à tester sont le token et le vérifieur (ici le module PAM), et plus précisément leurs
  composantes logicielles. L'ensemble des tests sera réalisé dans un contexte \og{}en 
  production\fg{} ou en simulant celui-ci.
  
  Les tests qui seront à effectuer pour les composantes Android seront effectués sur un terminal
  mobile Samsung Galaxy ACE.
  
  \section{Documents applicables et de références.}
  \begin{tabular}{p{1,5cm}>{\raggedright\arraybackslash}p{13cm}}
    {[ANS10]} & {ANSSI. Référentiel général de sécurité. 
    \href{http://www.ssi.gouv.fr/fr/reglementation-ssi/referentiel-general-de-securite}
    {http://www.ssi.gouv.fr/fr/reglementation-ssi/referentiel-general-de-securite}, 2010.}
    \tabularnewline
    \\
    {[MvOV97]} & {Alfred J. Menezes, Paul C. van Oorschot, and Scott A. Vanstone. Handbook of applied
    cryptography. CRC Press Series on Discrete Mathematics and its Applications. CRC Press, Boca
    Raton, FL, 1997. With a foreword by Ronald L.Rivest.}
    \tabularnewline
    \\
    {[RFC98]} & {A One-Time Password System. \href{http://tools.ietf.org/html/rfc2289}
    {http://tools.ietf.org/html/rfc2289}, 1998.}
    \tabularnewline
    \\
    {[RFC05]} & {HOTP:An HMAC-Based One-Time Password Algorithm \href{http://tools.ietf.org/html/rfc4226}{http://tools.ietf.org/html/rfc4226}, 2005.}
    \tabularnewline
    \\
    {[RFC06]} & {Generic Message Exchange Authentication for the Securer Shell Protocol (SSH).\href{http://tools.ietf.org/html/rfc4256}{http://tools.ietf.org/html/rfc4256}, 2006.}
    \tabularnewline
    \\
    {[RFC07]} & {The EAP Protected One-Time Password Protocol (EAP-POTP). \href{http://tools.ietf.org/html/rfc4793}{http://tools.ietf.org/html/rfc4793}, 2007.}
    \tabularnewline
    \\
    {[RFC11]} & {TOTP: Time-Based One-Time Password Algorithm \href{http://tools.ietf.org/html/rfc6238}{http://tools.ietf.org/html/rfc6238}, 2011.}
    \tabularnewline
    \\
    {[goo]} & {Google Authenticator \href{https://code.google.com/p/google-authenticator/}{https://code.google.com/p/google-authenticator/}.}
    \tabularnewline
    \\
  \end{tabular}
  
  \section{Environnement de test}
  L'ensemble des tests sera réalisé sur le site de l'Université de Rouen, antenne du Madrillet.
  Pour les tests, nous utiliserons des machines munies d'architectures grand public. Pour assurer le maximum de compatibilité, des systèmes 32 et 64 bits seront utilisés 
  indépendamment. Les tests sur Android seront réalisés sur un terminal mobile muni d'un processeur mono-cœur ARM v6 cadencé à 832MHz, disposant de 278 Mo de mémoire et tournant sur un 
  système d'exploitation Android 2.3.6.
  
  Afin de tester le module PAM, nous le brancherons sur plusieurs applications compatibles. Nous 
utiliserons en pratique un gestionnaire de login (Slim, GDM, KDM ...) ainsi que le logiciel openSSH.
Ces applications seront installées à la fois sur des machines personnelles et sur un serveur dont les 
spécification techniques seront détaillées ultérieurement.
  
  Dans l'ensemble, les tests seront effectués sur des configurations modestes afin d'assurer la performance sur la plupart des machines actuelles et futures.
  
  Les données utilisées pour les tests seront une succession de valeurs de mots de passe jetables successivement correctes, erronées et incohérentes.
  
  \section{Responsabilités}
  La création des procédures de test est laissée aux soins des responsables qualité de l'équipe. Ces procédures seront exécutées conformément au fonctionnement d'eXtreme Programming. A savoir : les tests sont exécutés tout au long du développement par les équipes. Chaque équipe travaillant au développement d'un composant est chargée de tester celui-ci continuellement et de publier les résultats des tests.
  
  Afin d'assurer l'appropriation collective du code source, toute l'équipe peut signaler une anomalie révélée au cours du développement : soit elle est résolue par le même acteur, soit l'acteur transmet les résultats 
des tests à une autre équipe afin que celle-ci travaille à la réparation de l'anomalie.
  
  À tout moment une équipe peut reporter des anomalies dans une procédure de test. Dans ce cas, les responsables techniques travailleront à améliorer la procédure incriminée et diffuseront les mises à jour à l'ensemble de l'équipe.
  
  \section{Stratégie de test}
  Les procédures de tests seront effectués avant le début de la réalisation de chaque composante (token, client, serveur). Toute composante testée qui serait non conforme ferait l'objet d'un 
  rapport d'erreur avant la reprise de son développement.
  
  Les tests seront toujours tous appliqués tout au long du développement des composants afin de suivre la politique de développement guidés par les tests (Test Driven Development).
    
  Les tests pourront généralement être exécutés indépendamment sur le token et le vérifieur. Les dépendances entre tests seront précisées dans la suite de ce document.
  
  Les tests de la mémoire utilisée seront réalisés afin d'assurer que la mémoire ne soit pas saturée.
  
  Les tests de temps assureront une réactivité minimum de l'application.
  
  Les attaques permettront de mettre en évidences d'éventuelles failles de sécurité et ainsi de les corriger.
  
  Les tests finaux viseront à relever les éventuelles mises en défaut du système par utilisation brutale. C'est à dire à tester la possibilité de mise à mal du système par une utilisation inappropriée ou volontairement abusive.
  
  \section{Gestion des anomalies}
  Lorsque des anomalies sont détectées lors des tests, voici la procédure qui devrait être suivie :
  \begin{itemize}
   \item Création d'une note / mémo précisant l'anomalie rencontrée ainsi que les informations concernant l'état du système lors de la rencontre de l'anomalie. Un 
   identifiant unique sera donné au mémo de test.
   \item Ajout d'une entrée au journal de test précisant la date du test, la référence du mémo, la référence de l'objet testé ainsi que la référence de l'exigence de 
   qualité contrariée.
   \item Diffusion de la note à l'équipe de développement.
   \item Une personne est désignée pour corriger l'anomalie.
   \item La personne vérifie la portée de l'anomalie, puis la corrige.
   \item Une deuxième personne vérifie que l'anomalie est bien corrigée.
   \item Une fois le feu vert donné, établir une contre-note portant le même identifiant et précisant les raisons supposées 
   de l'anomalie ainsi que les contremesures mises en place.
  \end{itemize}
  
  Dans la pratique, la gestions des anomalies sera faite à l'aide de l'outil \emph{issue tracker} de
  la plate-forme d'hébergement Github. Cet outil permet la création de files de discussion associées à un bug
  de l'application. Il laisse aussi la possibilité aux collaborateurs du projet de prendre en charge 
  une \emph{issue} particulière. L'ensemble des \emph{issues} ouvertes est conservé par Github ainsi
  que les files de discussion associés permettant de parcourir l'historique des bugs du projet.

  \section{Procédures de test}  
  \begin{center}
    %---------------------------------------Test N° 1------------------------------------------------------------------------------
    \begin{tabular}{|c|p{5cm}|p{5cm}|p{1.5cm}|p{1.5cm}|}
      \hline
      \multicolumn{3}{|l|}{Objet testé : Token} & \multicolumn{2}{c|}{Version : 1.1}\\ \hline
      \multicolumn{5}{|l|}{Objectif de test : Valider le temps de génération d'un OTP}\\ \hline
      \multicolumn{3}{|l|}{Procédure n° 1} & \multicolumn{2}{p{3cm}|}{Pré-requis : aucun}\\ \hline
      \multicolumn{1}{|c|}{N°} & \multicolumn{1}{c|}{Actions} & \multicolumn{1}{c|}{Résultats attendus} & 
      \multicolumn{1}{c|}{Exigence} & \multicolumn{1}{c|}{OK/KO}\\ \hline
      1 & Demander un OTP au token & Un OTP est retourné & EP\_01 & OK\\
      2 & Mesurer le temps de génération & Temps < 1s & EQ\_01 & OK\\
      3 & Répéter 100 fois 1) et 2) & Temps toujours < 1s pour chaque itération & EQ\_01 & OK\\ \hline
    \end{tabular}
    \vfill
    %---------------------------------------Test N° 2------------------------------------------------------------------------------
    
    \begin{tabular}{|c|p{5cm}|p{5cm}|p{1.5cm}|p{1.5cm}|}
      \hline
      \multicolumn{3}{|l|}{Objet testé : Module PAM} & \multicolumn{2}{c|}{Version : 2.1}\\ \hline
      \multicolumn{5}{|l|}{Objectif de test : Mesurer le temps de vérification du module}\\ \hline
      \multicolumn{3}{|l|}{Procédure n° 2} & \multicolumn{2}{p{3cm}|}{Pré-requis : EP\_01}\\ \hline
      \multicolumn{1}{|c|}{N°} & \multicolumn{1}{c|}{Actions} & \multicolumn{1}{c|}{Résultats attendus} & 
      \multicolumn{1}{c|}{Exigence} & \multicolumn{1}{c|}{OK/KO}\\ \hline
      1 & Ecrire un script utilisant la fonction de vérification du mot de passe & - & - & - \\ 
      2 & Appeler le script sur un otp valide et passer l'exécutable dans la commande time &
            Temps < 2s & EQ\_02 & OK \\
      3 & Réitérer l'opération avec un otp invalide & Temps < 2s & EQ\_02 & OK \\
      \hline
    \end{tabular}
    \vfill
    %---------------------------------------Test N° 4------------------------------------------------------------------------------
    
    \begin{tabular}{|c|p{5cm}|p{5cm}|p{1.5cm}|p{1.5cm}|}
      \hline
      \multicolumn{3}{|l|}{Objet testé : Token} & \multicolumn{2}{c|}{Version : 1.1}\\ \hline
      \multicolumn{5}{|l|}{Objectif de test : Mesurer la quantité de mémoire requise.}\\ \hline
      \multicolumn{3}{|l|}{Procédure n° 4} & \multicolumn{2}{p{3cm}|}{Pré-requis : aucun}\\ \hline
      \multicolumn{1}{|c|}{N°} & \multicolumn{1}{c|}{Actions} & \multicolumn{1}{c|}{Résultats attendus} & 
      \multicolumn{1}{c|}{Exigence} & \multicolumn{1}{c|}{OK/KO}\\ \hline
      1 & Lancer un debugger & - & - & - \\
      2 & Charger un programme de test utilisant le générateur d'OTP dans le debugger & - & - & -\\
      3 & Mesurer la charge mémoire actuelle & $m_0$ = charge mémoire & - & -\\
      \hline
      \multicolumn{5}{|l|}{Répéter 4 et 5 jusqu'à la fin de l'exécution du programme.}\\
      \hline
      4 & Mesurer la charge mémoire actuelle & m - $m_0$ > 10Ko & EQ\_05 & KO\\
      5 & Avancer d'une instruction dans le debuggeur & - & - & -\\ \hline
      6 & Fin de l'exécution du programme & charge mémoire toujours < 10Ko & EQ\_05 & OK \\ \hline
    \end{tabular}
    \vfill
    
    %---------------------------------------Test N° 5------------------------------------------------------------------------------
    \begin{tabular}{|c|p{5cm}|p{5cm}|p{1.5cm}|p{1.5cm}|}
      \hline
      \multicolumn{3}{|l|}{Objet testé : Module PAM} & \multicolumn{2}{c|}{Version : 2.1}\\ \hline
      \multicolumn{5}{|l|}{Objectif de test : Mot de passe utilisable une seule fois}\\ \hline
      \multicolumn{3}{|l|}{Procédure n° 5} & \multicolumn{2}{p{3cm}|}{Pré-requis : EP\_01}\\ \hline
      \multicolumn{1}{|c|}{N°} & \multicolumn{1}{c|}{Actions} & \multicolumn{1}{c|}{Résultats attendus} & 
      \multicolumn{1}{c|}{Exigence} & \multicolumn{1}{c|}{OK/KO}\\ \hline
      1 & Générer un OTP $K$& Un OTP est généré & EP\_01 & OK \\
      2 & Demander une authentification serveur avec l'OTP $K$ & authentifié & EQ\_04 & OK\\
      3 & Demander une authentification serveur avec l'OTP $K$ & échec de l'authentification & EP\_02; EP\_05 & OK\\ \hline
    \end{tabular}
    \vskip 2.5cm
    %---------------------------------------Test N° 6------------------------------------------------------------------------------
    \begin{tabular}{|c|p{5cm}|p{5cm}|p{1.5cm}|p{1.5cm}|}
      \hline
      \multicolumn{3}{|l|}{Objet testé : Module PAM} & \multicolumn{2}{c|}{Version : 1.1}\\ \hline
      \multicolumn{5}{|l|}{Objectif de test : Processus de resynchronisation}\\ \hline
      \multicolumn{3}{|l|}{Procédure n° 6} & \multicolumn{2}{p{3cm}|}{Pré-requis : EF\_04 }\\ \hline
      \multicolumn{1}{|c|}{N°} & \multicolumn{1}{c|}{Actions} & \multicolumn{1}{c|}{Résultats attendus} & 
      \multicolumn{1}{c|}{Exigence} & \multicolumn{1}{c|}{OK/KO}\\ \hline
      1 & Générer $N$ OTP, avec $1 < N \leq{} s$, s la taille de la fenêtre de resynchronisation& $N$ OTP générés & EP\_01 & OK \\
      2 & Tenter une connexion avec le $N$\up{ième} OTP (Dernière OTP valable avant de sortir de la fenêtre) & Connexion acceptée. & EF\_05 & OK \\ \hline      
    \end{tabular} 
    \vskip 2.2cm 
     %---------------------------------------Test N° 7------------------------------------------------------------------------------
    \begin{tabular}{|c|p{5cm}|p{5cm}|p{1.5cm}|p{1.5cm}|}
      \hline
      \multicolumn{3}{|l|}{Objet testé : Module PAM} & \multicolumn{2}{c|}{Version : 1.1}\\ \hline
      \multicolumn{5}{|l|}{Objectif de test : Résistance aux attaques exhaustives}\\ \hline
      \multicolumn{3}{|l|}{Procédure n° 7} & \multicolumn{2}{p{3cm}|}{Pré-requis : EQ\_04 }\\ \hline
      \multicolumn{1}{|c|}{N°} & \multicolumn{1}{c|}{Actions} & \multicolumn{1}{c|}{Résultats attendus} & 
      \multicolumn{1}{c|}{Exigence} & \multicolumn{1}{c|}{OK/KO}\\ \hline
      0 & $cpt \leftarrow 0$ & - & - & -\\
      1 & Générer un OTP aléatoire $T$ & 1 OTP généré & EP\_01 & OK\\
      2 & Tenter une connexion avec $T$ & Authentifié & EQ\_04 & OK \\ \hline      
      \multicolumn{5}{|l|}{Répéter 1 et 2 jusqu'à la connexion ou le rejet}\\
      \hline
      3 & si accepté $cpt \leftarrow cpt + 1$ & & & \\
      \hline
      \multicolumn{5}{|l|}{Répéter 1 à 3 : $N$ fois.}\\
      \hline
      4 & Calculer $P = \frac{cpt}{N}$ & $P < \frac{1}{100000}$ & EP\_04 & OK \\ \hline 
    \end{tabular}
    \vskip 2.2cm 
	%---------------------------------------Test N° 8------------------------------------------------------------------------------
	\begin{tabular}{|c|p{5cm}|p{5cm}|p{1.5cm}|p{1.5cm}|}
      \hline
      \multicolumn{3}{|l|}{Objet testé : Générateur d'OTP} & \multicolumn{2}{c|}{Version : 2.1}\\ \hline
      \multicolumn{5}{|l|}{Objectif de test : Non prédictibilité des OTPs}\\ \hline
      \multicolumn{3}{|l|}{Procédure n° 8} & \multicolumn{2}{p{3cm}|}{Pré-requis : EP\_01 }\\ \hline
      \multicolumn{1}{|c|}{N°} & \multicolumn{1}{c|}{Actions} & \multicolumn{1}{c|}{Résultats attendus} & 
      \multicolumn{1}{c|}{Exigence} & \multicolumn{1}{c|}{OK/KO}\\ \hline
      1 & Générer un fichier aléatoire generated\_bin en utilisant l'utilitaire développé *& Fichier généré & - & - \\
      2 & Appeler dieharder -a -g 201 -f generated\_bin -D default -D histogram -D description pour générer un rapport
      	& Rapport généré & - & - \\
      3 & Analyser le rapport & Au moins 70\% des tests passent & EP\_03 & OK \\
      \hline
    \end{tabular} 
    
	* Le code de l'utilitaire est donné en annexe.
	
	
	\vskip 2.2cm
	\begin{tabular}{|c|p{5cm}|p{5cm}|p{1.5cm}|p{1.5cm}|}
      \hline
      \multicolumn{3}{|l|}{Objet testé : Module PAM} & \multicolumn{2}{c|}{Version : 2.1}\\ \hline
      \multicolumn{5}{|l|}{Objectif de test : Asociation des utilisateurs}\\ \hline
      \multicolumn{3}{|l|}{Procédure n° 9} & \multicolumn{2}{p{3cm}|}{Pré-requis : EP\_01, EQ\_04 }\\ \hline
      \multicolumn{1}{|c|}{N°} & \multicolumn{1}{c|}{Actions} & \multicolumn{1}{c|}{Résultats attendus} & 
      \multicolumn{1}{c|}{Exigence} & \multicolumn{1}{c|}{OK/KO}\\ \hline
      1 & Utiliser l'utilitaire d'association pour le login "login" & Clef privée K générée & - & - \\
      2 & Générer un OTP pour la clef privée K & Un OTP est retourné & - & - \\
      3 & Tenter une authentification avec le login "login" et l'OTP généré & Authentification réussie & EF\_02 & OK \\      
      \hline
    \end{tabular} 
	 
  \end{center}
  \section{Couverture de tests}
  \begin{center}
    \begin{tabular}{|p{2.8cm}|p{4.2cm}|p{3cm}|p{5cm}|}
      \hline
      Id Exigence STB & Méthode de vérification & Procédure utilisée & Commentaire\\ \hline
      EF\_02 & Associer un token au serveur. & Procédure N° 9 & \\ \hline
      EF\_03 & Demander un OTP au token et vérifier la validité de cet OTP & Procédure N° 1, 5 & \\ \hline
      EF\_04 & Vérifier qu'il y a connexion avec un OTP valide et qu'on est rejeté avec un OTP invalide. & Procédure N° 2 & \\ \hline
      EF\_05 & Vérifier que les OTP de la fenêtre de resynchronistation sont acceptés. & Procédure N° 6 & \\ \hline
      EP\_01 & Demander un OTP au token et vérifier la validité de cet OTP & Procédure N° 1, 5 & \\ \hline
      EP\_02 & Tester deux fois le même OTP & Procédure N° 5 & \\ \hline
      EP\_03 & Utilisation d'un logiciel \og{}die hard\fg{} pour vérifier l'aléatoirité des mots de passe. & Procédure 	N° 8 & Pour exécuter cette procédure, il est nécessaire de compiler l'utilitaire de génération de fichier aléatoire\\ \hline
      EP\_04 & Tester des OTP jusqu'à réussite de l'attaque ou rejet par l'application. & Procédure N° 7 & \\ \hline
      EP\_05 & Rejouer 2 fois le même OTP & Procédure N° 5 & \\ \hline
      EQ\_01 & Mesurer le temps de génération d'un OTP. & Procédure N° 1 & \\ \hline
      EQ\_02 & Mesurer le temps d'exécution du serveur. & Procédure N° 2 & \\ \hline
      EQ\_04 & Vérifier qu'il y a connexion avec un OTP valide et qu'on est rejeté avec un OTP invalide. & Procédure N° 2, 5 & \\ \hline
      EQ\_05 & Vérifier qu'il n'y a pas plus de 10Ko de mémoire utilisés en plus après la génération. & Procédure N° 4 & \\ \hline
    \end{tabular}  
  \end{center}
	
	\newpage
	\begin{appendices}
	\chapter{Génération du fichier aléatoire}
		\lstset{language=C}
		\begin{lstlisting}
#define _GNU_SOURCE

#include <stdio.h>
#include <stdlib.h>
#include <unistd.h>
#include <string.h>
#include <sys/types.h>
#include <fcntl.h>
#include <sys/stat.h>

#include "../utils/otp.h"
#include "../utils/secret.h"

int main(int argc, char *argv[]) {
    /*
     * Parameters validation.
     */
    if (argc != 3) {
        printf(
            "Usage : %s seed nb len\n",
            argv[0]
            );
        printf("\tseed : The hexadecimal seed for generation\n");
        printf("\tnb : The number of values to generate.\n");
        exit(EXIT_SUCCESS);
    }
    long nb = strtol(argv[2], NULL, 10);
    if (nb <= 0) {
        printf("Invalid argument %s. nb must be a number.\n", argv[2]);
        exit(EXIT_SUCCESS);
    }
    long len = 8;
    /*
     * Seed validation
     */
    char *arg = argv[1];
    for (int i = 0; i < strlen(arg); i++) {
        char c[2];
        snprintf(c, 2, "%c", arg[i]);
        if (strcasestr("abcdef0123456789", c) == NULL) {
            printf("Seed must be an hexadecimal string.\n");
            exit(EXIT_SUCCESS);
        }
    }
    /*
     * Generation loop.
     */
    secret s = hexToSecret(argv[1]);
    int fd = open(
        "generated_bin",
        O_CREAT | O_TRUNC | O_WRONLY,
        S_IRUSR | S_IWUSR);
    if (fd == -1) {
        perror("Opening generated.bin");
        exit(EXIT_FAILURE);
    }
    for (int i = 0; i < nb; i++) {
        int value = generate_otp(s, i, len);
    /*
     * Write exclude extra 0 at the begining comming from the sign bit and
     * from the fact that otp < 10^8 whereas an int is 2^32.
     */
        if (write(fd, ((char *) &value), sizeof(int) - 1) == -1) {
            perror("writing");
            close(fd);
            exit(EXIT_FAILURE);
        }
    }
    close(fd);
    exit(EXIT_SUCCESS);
}
		\end{lstlisting}
	\end{appendices}  
\end{document}
