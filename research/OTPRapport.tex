\documentclass{../res/univ-projet}

\title{\'Etude du syst\`eme d'OTP g\'en\'eral}
\author{Adrien \bsc{Smondack}}

\projet{One Time Project}
\projdesc{\'Etude des syst\`emes de mots de passe jetable}
\filiere{M1SSI}
\logo{../res/logo_univ.png}

\usepackage[T1]{fontenc}
\usepackage[utf8]{inputenc}

\usepackage{algorithmic}
\usepackage{algorithm}

\begin{document}
\maketitle

\begin{abstract}
  Ce document présente une analyse du système d'OTP général. Toutes les 
informations présentées et analysées proviennent de la RFC 2289 qui est une mise à jour
de la RFC 1938. Le but de cet 
article est de déterminer dans quelles condition le système OTP est utilisable, 
sous quelles conditions et avec quelles garanties de sécurité.
  
  Ce documents sera mis en relation avec plusieurs autres afin de réaliser un 
comparatif entre les systèmes d'OTP majeurs.
\end{abstract}
\newpage
\tableofcontents
\newpage

%------------------------------------------------------------------------------
\section{Prérequis}
  Le système OTP repose principalement sur l'utilisation de fonctions de 
  hachage, les algorithmes les plus courants seront présentés dans ce 
  document.

  OTP est une évolution de S/KEY (système publié par Bellcore), qui est 
  lui-même un système OTP dont la description complète est disponible
  dans la RFC 1760 (datant de 1995).

%------------------------------------------------------------------------------
\section{Généralités}
  Le système OTP (One-Time Password) est un standard décrivant des protocoles 
destinés à la mise en service d'un système d'authentification par mot de passe 
jetable. Ce standard date de 1998.

  Ce système propose une solution de prévention des attaques dites \emph{par 
rejeu} (c'est à dire récupérer les login et mot de passe d'une personne pour un 
usage ultérieur). Tous au long de ce document, nous considérerons plusieurs 
entités :

  \begin{description}
    \item[L'utilisateur :] le sujet de l'authentification en vue d'une 
    connection au serveur;
    \item[Le client :] un logiciel doté d'une interface utilisateur ou un site 
    web permettant une authentification auprès d'un serveur adapté;
    \item[Le serveur :] l'entité qui permet d'authentifier un utilisateur,
    contient une base de données contenant des informations sur l'utilisateur 
    (login, dernier mot de 
    passe utilisé, ...) et dispose d'une routine permettant de vérifier 
    l'authenticité de celui-ci avec le mot de passe jetable qu'il lui a fournit;
    \item[Le token :] élément physique ou logiciel permettant de générer un mot 
    de passe jetable pour un utilisateur;
    \item[Un attaquant :] hacker, malware, etc... Une entité cherchant à 
    usurper 
    l'identité de l'utilisateur vis à vis du serveur en récupérant ses informations 
    de connection. Ou son accès au service sans ces informations\\
  \end{description}

  Le principe est simple :

  \begin{itemize}
    \item Un utilisateur souhaitant se connecter sur un service via un client
    employant le système OTP dispose d'un token lui permettant de générer un 
    mot de passe jetable 
    (qu'on appellera par la suite otp) que l'interface lui demande;
    \item Le token exécute une série de calculs sur une clé secrète et en tire 
    un otp qu'il communique à l'utilisateur;
    \item l'utilisateur renseigne son login et l'otp au client;
    \item Le client envoie une requ\^ete au serveur et lui fournit l'otp; 
    \item Le serveur vérifie le mot de passe en appliquant la fonction de 
    hachage sur celui-ci. Si le résultat correspond au dernier otp utilisé par cet 
    utilisateur, celui-ci est authentifié. Et conserve cet OTP comme dernier
    utilisé\\
  \end{itemize}

  Le mot de passe étant à usage unique, il ne pourra pas servir à un attaquant 
l'ayant récupéré en écoutant les communications entre le client et le serveur.

  Les principaux avantages sont les suivants :

  \begin{itemize}
    \item Le secret secret n'est jamais transmis entre le client et le serveur.
    \item La sécurité des mots de passe générés repose sur la non-réversibilité 
    de la fonction de hachage utilisée. Ainsi, il est \^extrèmement difficile (voir 
    impossible) de prédire les prochains mot de passe générés.\\
  \end{itemize}

  Le système OTP fournit ainsi un protection affirmée contre tout type 
  d'attaque contre le vol d'identifiants.
  En revanche elle ne protège pas le client des attaques telles 
  que le vol de session\footnote{Vol de cookies.} et l'espionnage.

%------------------------------------------------------------------------------
\section{Approfondissement}
  Nous étudirons ici plus en détails  les propriétés du système OTP.

  \subsection{Generation et partage d'un secret}
    Le secret ici est la clé utilisée par le token pour calculer un otp, il 
s'agit d'une phrase de passe généralement fournit en clair par l'utilisateur 
sous forme de données textuelles. 

    À aucun moment cette clé ne doit traverser le réseau, que ce soit lors 
d'une authentification ou d'un changement de clé.

    Ce qui résout la question du partage de la clé. Seul le token connait la 
clé permettant de générer les mots de passe, il est donc impossible à un attaquant 
de la récupérer pour générer lui même les mot de passe (à moins de récupérer 
le token lui-même).

  \subsection{Génération d'un mot de passe jetable}

    La génération d'un otp se déroule en trois étapes :
    \begin{description}
      \item [phase initiale :] combinaison des entrée;
      \item [phase de calcul :] application de la fonction de hachage un 
      certain nombre de fois;
      \item [phase de conversion :] une fonction renvoyant les 64 bits de l'otp 
      sous forme lisible.\\
    \end{description}

    Voyons ces trois étapes en détails.

    \subsubsection{Phase initiale}
      Chaque otp est calculé à partir du \emph{challenge}.

      Le serveur fournit en clair une graine que l'on concatène à la clé. Un 
utilisateur peut donc utiliser la même clé depuis plusieurs machines, le serveur 
enverra des graines différentes. La graine ne doit contenir que des caractères 
alphanumériques et la longueur doit être inférieure à 16. C'est une chaîne de 
caratères sans espace et contenant de préférence des caractères issus du code 
ISO-646. Elle n'est pas sensible à la casse, et on se conformera à la convertir 
en minuscule avant tout calcul.

      La clé et la graine forment ensemble le challenge. La syntaxe du challenge
doit être standard de façon à ce que les générateurs automatisés puissent les 
reconnaitre et en extraire des informations. La syntaxe standard prend la forme 
suivante:
      \begin{center}
          \emph{otp-<algorithmID> <séquence d'entier> <graine>}\\
      \end{center}

      Les trois segments sont séparés par une combinaison de caractères 
d'espacement (espace et/ou tabulation) et doit se terminer de même ou par un 
retour chariot. 

      La chaîne "otp" au début doit être en minuscule.

      L'ID de l'algorithme est sensible à la casse, ceux pré-existant sont en 
      minuscule. Voici ceux utilisés le plus couramment :
      \begin{description}
          \item [MD4] id : md4
          \item [MD5] id : md5
          \item [SHA1] id : sha1 
      \end{description}
      Tout algorithme supplémentaire peut être reconnu en définissant un 
identifiant court.

      La graine suit les mêmes contraintes qu'auparavant : insensible à la 
    casse, convertit en minuscule avant usage.

      Voici un exemple de challenge :
      \begin{center}
          \emph{otp-md5 487 dog2 }\\ 
      \end{center}

      Le challenge est passé à la fonction de hachage puis dans une fonction
      de réduction pour arriver à une taille de 64 bits. Par la suite,
      on notera ce résultat \emph{S}.

    \subsubsection{Phase de calcul}
      Une série d'otp est produite par application successive de la fonction de 
hachage (un nombre \emph{N} de fois spécifié par l'utilisateur) sur S. Le 
résultat sera le 1\textsuperscript{ier} otp communiqué à l'utilisateur. Le 
suivant sera le résultat de \emph{N-1} occurrences de la fonction de hachage 
sur 
S. 

      Un espion interceptant la transmission d'un otp ne pourra pas déterminer 
le prochain (l'occurence \emph{N-2}) car cela impliquerait qu'il ait inversé la 
fonction de hachage. Or c'est sur la non-réversibilité de la fonction de 
hachage 
choisie que repose la sécurité des otp générés.\\



    \subsubsection{Phase de conversion}
        L'otp généré, fait 64 bit de long. Tout serveur doit être en mesure 
d'accepter un otp sous la forme de 6 mots de 1 à 4 lettres appartenant à la 
norme ISO-646, chaque mots étant choisit dans un dictionnaire de 2048 mots. On compte 
11 
bits par mots, pour un total de 66 bits (\emph{à quoi servent les 2 bits 
supplémentaires?} voir page 5).

        La présentation de l'otp sous la forme de 6 mots doit se faire comme 
suit : en majuscules, et séparés avec un espace. Toutefois, tout serveur 
supportant ce format doit accepter sans vérifier la casse ni les séparateurs. 
Le 
premier exemple ci-dessous illustre un sortie de générateur conforme et 
acceptable en entrée. Le second est uniquement acceptable en entrée :
        \begin{center}
            OUST COAT FOAL MUG BEAK TOTE\\
            oust coat foal mug beak tote
        \end{center}

        L'interopérabilité implique que les serveurs et le générateurs 
emploient 
tous le même dictionnaire. On utilise couramment celui fournit dans la RFC 1760 
(fournit en index D).

        Il est possible de fournir une sortie de généateur en hexadecimal. Le 
serveur doit accepter l'hexa-décimal en entrée (sans tenir compte de la casse). 
Les chiffres/lettres peuvent être séparés par des espaces qui seront ignorés 
par 
le serveur. Voici quelque exemples:
        \begin{itemize}
            \item 3503785b369cda8b              0x3503785b369cda8b
            \item e5cc a1b8 7c13 096b           0xe5cca1b87c13096b
            \item C7 48 90 F4 27 7B A1 CF       0xc74890f4277ba1cf
            \item 47 9 A68 28 4C 9D 0 1BC       0x479a68284c9d01bc
        \end{itemize}

        En plus du format 6 mots et hexadécimal, le serveur devrait accepter 
l'encodage décrit dans l'index B. Le répertoire de mot du dictionnaire standard ne doit 
cependant pas \^etre altéré ou oubli\'e. Afin d'\'eviter toute ambiguïté avec 
l'hexadécimal, on \'evitera les lettre de A à F. Le mots du nouveau dictionnaire sont 
sensibles à la casse et la casse doit être conservé.\\

        Au final, le serveur doit accepter :
        \begin{itemize}
            \item le format 6 mots basé dur les dictionnaire de la RFC1760 et de 
            l'annexe D;
            \item l'hexadécimal;
            \item éventuellement le format 6 mots basé sur le dictionnaire alternatif 
            (annexe B).
        \end{itemize}
  
  \subsection{Soumission et protocole de vérification}
        Voici ce qui est dit dans la RFC 2289:
    une fois l'OTP transmis au serveur il décodé en une clé de 64 bits puis haché.
    Si ce haché correspond à l'OTP précédent il est donc valide.
    L'utilisateur est authentifié et l'OTP prends la place du précédent.
    
        Seulement un OTP est constitué de 64 bits. selon les fonction un haché
    d'OTP fait entre 128 et 160 bits. Donc un haché ne pourra jamais correspondre
    à un OTP. La RFC 2289 n'est pas précise sur ce point.
  
  \subsection{Synchronisation}
    Ce système fonctionnant sur un modèle défi réponse il est impossible de perdre
    la synchronisation. À chaque fois le serveur demande un OTP bien précis dans la séquence.
  
  \subsection{Reinitialisation}
    Le système étant limité en nombre d'utilisation (génération du $n^{i\grave{e}me}$ 
    hachés lors de l'initialisation) il est nécessaire de réinitialiser le système
    une fois cette limite atteinte sous peine de ne plus pouvoir s'authentifier.
    
    Lors de la ré-initialisation le paramètre seed est renouvelé ce qui permet à
    l'utilisateur de garder la même clé secrète. Soit l'utilisateur spécifie la valeur
    pour seed et le nombre d'OTP à générer soit il utilise la valeur aléatoire suggérée.
    
    La RFC 2289 préconise de permetrre à un utilisateur de réinitialiser son système
    à distance sans révéler sa phrase de passe secrète. Ceci n'intervient pas dans
    une implémentation où seul le haché précédent est stocké.

%------------------------------------------------------------------------------
\section{Analyse générale et sécurité}
    \subsection{Avantages et intér\^ets}
    Les avantages fournits par le système OTP sont ceux fournits par tous les
    systèmes d'authentification à mot de passe jetable. Ce mode de connexion 
    offre donc une résistance aux attaques par rejeu.
    
    Il permet ainsi de s'assurer que la réutilisation de mot de passe par un
    attaquant qui les aurait récupérés lors d'une authentification n'est pas 
    possible.
    
    Ce système peut, de plus, être utilisé conjointement avec un système 
    classique d'authentification par login / mot de passe. On met ainsi en 
    place un système d'authentification dit forte assurant un niveau de 
    sécurité plus important bien que restant moins élevé que les systèmes
    d'authentification cryptographique.
  
  \subsection{Inconvenients et limites}
  Le protocole d'OTP étudié ici est le premier né des systèmes 
  d'authentification à mot de passe jetable. Ainsi, bien que posant les bases
  de la méthode, il souffre de certains inconvénients d'utilisation. Ces 
  dèrniers n'engendre pas de problèmes de sécurité mais empêche toute 
  utilisation confortable du système.
  
  Parmis ces soucis de confort, le plus remarquable est sans doute le fait
  qu'il n'y a qu'un nombre fixé de connections possibles pour une 
  initialisation donnée. En effet, si lors de l'association du client avec 
  le serveur le nombre de hachés successif calculés est fixé à 100, seuls 
  99 connections seront possibles successivement. Une fois cette limite 
  atteinte il sera nécessaire de changer la graine de génération et donc
  de réinitialiser le protocole.
  
  Un autre point important est relatif aux performances de la méthode. En effet
  lorsque le token calcul l'OTP, il doit hacher successivement plusieurs 
  valeurs jusqu'a atteindre le numéro de mot de passe demandé. Or le coup en 
  calcule peut être important pour les premiers OTP demandés.
  
  \subsection{Considérations de sécurité}
    \subsubsection{Attaques exhaustives}
    Les attaques par recherche exhaustives sur le système OTP peuvent être 
    éxecutées dans deux buts différents. L'attaquant peut soit rechercher à
    trouver la clef secrète de l'utilisateur soit à trouver la valeur de l'OTP
    actuellement demandé.
    
    Dans le premier cas, l'attaquant demande une authentification et récupère 
    le challenge de connexion. Il devra alors calculer la valeur du $N^{ième}$ 
    haché pour toute les valeurs de clef secrète possible. La résistance aux
    attaques sur ce point dépend donc de la clef de l'utilisateur. Cependant, 
    la longueur de cette clef est limitée inférieurement à 10 caractères. Si
    l'on considère que la clef est codée en ascii 8bits, on obtient donc une 
    taille minimum de clef de $2^{80}$ bits. Au vues du coup nécessaire au 
    calcul d'un OTP, cette valeur semble raisonnable. De plus, la longueur de 
    la clef peut êotre étendue jusqu'a plus de 60 caractères augmentant encore
    la résistance.
    
    Dans le second cas, l'attaquant demande un authentification et tente 
    ensuite de trouver la valeur à entrer pour s'authentifier. La méthode est
    simple puisqu'il suffit d'énumérer les valeurs possibles pour un OTP. Dans
    notre cas, l'OTP étant codé sur 64 bits l'attaque s'éxecutera en $2^{64}$ 
    opérations. Si l'on considère qu'il est possible d'effectuer $10^9$ 
    tentatives par seconde, alors cette attaque réussit en 58 ans. Le mot de 
    passe étant jetable, ce temps est plus qu'acceptable.
    
    \subsubsection{Attaques par collisions}
        Les collisions recherchées dans cette section sont des collisions dîtes fortes.
    puisqu'il faut trouver un OTP aillant une empreinte bien précise.
    
        Il y a deux façon de voir l'attaque, soit on cherche une collision direct sur
    le prochain OTP au quel cas on applique une fois la fonction de hachage sur $n$ clefs.
    Comme on considère des hachés de 128 à 160 bits cela fait des attaques entre $O(2^128)$
    et $O(2^160)$ ce qui n'est pas envisageable. en cas de réussite cette attaque permet 
    d'obtenir un seul identifiant et donc une seule session.
    
        Soit on génère une suite d'OTP a partir d'une clé choisie jusqu'a trouver
    une collision avec le prochain OTP. les complexité restent les même mais ce type
    d'attaque permettrait d'obtenir plusieurs identifiant et donc plusieurs sessions.
    
    \subsubsection{Failles connues}
      Tout est dans le titre.
    
    \subsubsection{Précautions et préconisation}
    La clé doit contenir au moins 10 caractères pour réduire les risques de la 
retrouver par la méthode de recherche exhaustive ou par attaque du dictionnaire. 
Cependant toute implémentation du système OTP doit supporter des clés d'au 
moins 10 à 63 caractères.

    Afin d'assurer l'interchangeabilité des générateurs d'otp, chacun doit 
supporter des clés de 10 à 63 caractères, des clés plus longues sont 
envisageables (mais attention à la compatibilité avec des générateurs plus 
restreints).
  
%------------------------------------------------------------------------------
\section{Utilisations}
  Le protocole ne trouve plus d'utilisation actuellement du fait qu'il existe des protocoles remplissant
  les mêmes fonctions tout en étant plus simple et plus performant on peut citer par exemple HOTP ou TOTP, cependant
  cette méthode pourrait quand même être utilisée en remplacement.
  
%------------------------------------------------------------------------------
\section{Conclusion}
  La conclusion
  \subsection{Utilisation dans le cadre du projet}
    Reprendre les éléments de l'analyse telles que le performances du système 
le 
niveau de sécurité et autre afin de définir si, et dans quels contextes, le 
système devrait\^etre utlisé dans le cadre du projet.
  
  \subsection{Perenité du système}
    A voir, pas forcément utile. Se référer aux résultats de la réunion du 19 
novembre.
    
\end{document}
