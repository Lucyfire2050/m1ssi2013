\documentclass{../res/univ-projet}

\title{\'Etude du syst\`eme d'OTP g\'en\'eral}
\author{Adrien \bsc{Smondack}}

\projet{One Time Project}
\projdesc{\'Etude des syst\`emes de mots de passe jetable}
\filiere{M1SSI}
\logo{../res/logo_univ.png}

\usepackage[T1]{fontenc}
\usepackage[utf8]{inputenc}

\usepackage{algorithmic}
\usepackage{algorithm}

\begin{document}
\maketitle

\begin{abstract}
  Ce document présente une analyse du système d'OTP général. Toutes les 
informations présentées et analysées proviennent de la RFC 2289. Le but de cet 
article est de déterminer dans quelles condition le système OTP est utilisable, 
sous quelles conditions et avec quelles garanties de sécurité.
  
  Ce documents sera mis en relation avec plusieurs autres afin de réaliser un 
comparatif entre les systèmes d'OTP majeurs.
\end{abstract}
\newpage
\tableofcontents
\newpage

%------------------------------------------------------------------------------
\section{Prérequis}
  Le système OTP repose principalement sur l'utilisation de fonctions de 
hachage, les algorithmes les plus courants seront présentés dans ce document.

  OPT est une évolution de S/KEY (système publié par Bellcore), qui est lui-même 
un système OTP dont la description complète est disponible dans la RFC 1760 
(datant de 1995).

%------------------------------------------------------------------------------
\section{Généralités}
  Le système OTP (One-Time Password) est un standard décrivant des protocoles 
destinés à la mise en service d'un système d'authentification par mot de passe 
jetable. Ce standard date de 1998.

  Ce système propose une solution de prévention des attaques dites \emph{par 
rejeu} (c'est à dire récupérer les login et mot de passe d'une personne pour un 
usage ultérieur). Tous au long de ce document, nous considérerons plusieurs 
entités :

  \begin{description}
    \item[L'utilisateur :] le sujet de l'authentification en vue d'une 
connection au serveur;
    \item[Le client :] un logiciel doté d'une interface utilisateur ou un site 
web requierant une authentification;
    \item[Le serveur :] l'entité à laquelle on veut accéder, contient une base 
de données contenant des informations sur l'utilisateur (login, dernier mot de 
passe utilisé, ...) et dispose d'une routine permettant de vérifier 
l'authenticité de celui-ci avec le mot de passe jetable qu'il lui a fournit;
    \item[Le token :] élément physique ou logiciel permettant de générer un mot 
de passe jetable;
    \item[Un attaquant :] hacker, malware, etc... Une entité cherchant à usurper 
l'identité de l'utilisateur vis à vis du serveur en récupérant ses informations 
de connection.\\
  \end{description}

  Le principe est simple :

  \begin{itemize}
    \item Un utilisateur souhaitant se connecter sur un client employant le 
système OTP dispose d'un token lui permettant de générer un mot de passe jetable 
(qu'on appellera par la suite otp) que l'interface lui demande;
    \item Le token éxecute une série de calculs sur une clé secrète et en tire 
un otp qu'il communique à l'utilisateur;
    \item l'utilisateur renseigne son login et l'otp au client;
    \item Le client envoie une requ\^ete au serveur et lui fournit l'otp; 
    \item Le serveur vérifie le mot de passe en appliquant la fonction de 
hachage sur celui-ci. Si le résultat correspond au dernier otp utilisé par cet 
utilisateur, celui-ci est authentifié.\\
  \end{itemize}

  Le mot de passe étant à usage unique, il ne pourra pas servir à un attaquant 
l'ayant récupéré en écoutant les communications entre le client et le serveur.

  Les principaux avantages sont les suivants :

  \begin{itemize}
    \item Tout d'abord, aucun secret n'a besoin d'être partagé entre le client 
et le serveur;
    \item La sécurité des mots de passe générés repose sur la non-réversibilité 
de la fonction de hachage utilisée. Ainsi, il est \^extrèmement difficile (voir 
impossible) de prédire les prochains mot de passe générés.\\
  \end{itemize}

  Le système OTP fournit ainsi un protection affirmée contre tout type d'attaque 
passive. En revanche elle ne protège pas le client des attaques actives telles 
que l'usurpation d'identité et l'espionnage.

%------------------------------------------------------------------------------
\section{Approfondissement}
  Nous étudirons ici plus en détails  les propriétés du système OTP.

  \subsection{Generation et partage d'un secret}
    Le secret ici est la clé utilisée par le token pour calculer un otp, il 
s'agit d'une phrase de passe généralement fournit en clair par l'utilisateur 
sous forme de donnes textuelles. 

    À aucun moment cette clé ne doit traverser le réseau, que ce soit lors d'une 
authentification ou d'un changement de clé.

    Ce qui résout la question du partage de la clé. Seul le token connait la clé 
permettant de générer les mots de passe, il est donc impossible à un attaquant 
de la récupérer pour générer elle même les mot de passe (à moins de récupérer le 
token lui-même).
  
  \subsection{Génération d'un mot de passe jetable}

    La génération d'un otp se déroule en trois étapes :
    \begin{description}
      \item [phase initiale :] combination des entrée;
      \item [phase de calcul :] application de la fonction de hachage un certain 
nombre de fois;
      \item [phase de conversion :] une fonction renvoyant les 64 bits de l'otp 
sous forme lisible.\\
    \end{description}

    Voyons ces trois étapes en détails.

    \subsubsection{Phase initiale}
      Chaque otp est calculé à partir du \emph{challenge}.

      Le serveur fournit en clair une graine que l'on concatène à la clé. Un 
utilisateur peut donc utiliser la même clé depuis plusieurs machines, le serveur 
enverra des graines différentes. La graine ne doit contenir que des caractères 
alphanumériques et la longueur doit être inférieure à 16. C'est une chaîne de 
caratères sans espace et contenant de préférence des caractères issus du code 
ISO-646. Elle n'est pas sensible à la casse, et on se conformera à la convertir 
en minuscule avant tout calcul.

      La clé et la graine forment ensemble le challenge. La syntaxe du challenge 
doit être standard de façon à ce que les générateurs automatisés puissent les 
reconnaitre et en extraire des informations. La syntaxe standard prend la forme 
suivante:
      \begin{center}
          \emph{otp-<algorithmID> <séquence d'entier> <graine>}\\
      \end{center}
      
      Les trois segments sont séparés par une combinaison de caractères 
d'espacement (espace et/ou tabulation) et doit se terminer de même ou par un 
retour chariot. 

      La chaîne "otp" au début doit être en minuscule.

      L'ID de l'algorithme est sensible à la casse, ceux pré-existant sont en 
minuscule. Voici ceux utilisés le plus couramment :
      \begin{description}
          \item [MD4] id : md4
          \item [MD5] id : md5
          \item [SHA1] id : sha1 
      \end{description}
      Tout algorithme supplémentaire peut être reconnu en définissant un 
identifiant court.

      La graine suis les mêmes contraintes qu'auparavant : insensible à la 
casse, convertit en minuscule avant usage.

      Voici un exemple de challenge :
      \begin{center}
          \emph{otp-md5 487 dog2 }\\    ATTENTION GROS DOUTE : LE CHALLENGE 
C'EST TT CA OU JUSTE LA SEQ D'ENTIERS + LA GRAINE ? 
<-----------------------------------
      \end{center}

      Le challenge est passé à la fonction de hachage afin de le réduire à une 
taille de 64 bits. Par la suite, on notera ce résultat \emph{S}.

    \subsubsection{Phase de calcul}
      Une série d'otp est produite par application successive de la fonction de 
hachage (un nombre \emph{N} de fois spécifié par l'utilisateur) sur S. Le 
résultat sera le 1\textsuperscript{ier} otp communiqué à l'utilisateur. Le 
suivant sera le résultat de \emph{N-1} occurrences de la fonction de hachage sur 
S. 

      Un espion interceptant la transmission d'un otp ne pourra pas déterminer 
le prochain (l'occurence \emph{N-2}) car cela impliquerait qu'il ait inversé la 
fonction de hachage. Or c'est sur la non-réversibilité de la fonction de hachage 
choisie que repose la sécurité des otp générés.\\



    \subsubsection{Phase de conversion}
        L'otp généré, fait 64 bit de long. Tout serveur doit être en mesure 
d'accepter un otp sous la forme de 6 mots de 1 à 4 lettres appartenant à la 
norme ISO-646, chaque mots étant choisit dans un dico de 2048 mots. On compte 11 
bits par mots, pour un total de 66 bits (\emph{à quoi servent les 2 bits 
supplémentaires?} voir page 5).

        La présentation de l'otp sous la forme de 6 mots doit se faire comme 
suit : en majuscules, et séparés avec un espace. Toutefois, tout serveur 
supportant ce format doit accepter sans vérifier la casse ni les séparateurs. Le 
premier exemple ci-dessous illustre un sortie de générateur conforme et 
acceptable en entrée. Le second est uniquement acceptable en entrée :
        \begin{center}
            OUST COAT FOAL MUG BEAK TOTE\\
            oust coat foal mug beak tote
        \end{center}

        L'interopérabilité implique que les serveurs et le générateurs emploient 
tous le même dictionnaire. On utilise couramment celui fournit dans la RFC 1760 
(fournit en index D).

        Il est possible de fournir un sortie de généateur en hexadecimal. Le 
serveur doit accepter l'hexa-décimal en entrée (sans tenir compte de la casse). 
Les chiffres/lettres peuvent être séparés par des espaces qui seront ignorés par 
le serveur. Voici quelque exemples:
        \begin{itemize}
            \item 3503785b369cda8b              0x3503785b369cda8b
            \item e5cc a1b8 7c13 096b           0xe5cca1b87c13096b
            \item C7 48 90 F4 27 7B A1 CF       0xc74890f4277ba1cf
            \item 47 9 A68 28 4C 9D 0 1BC       0x479a68284c9d01bc
        \end{itemize}

        En plus du format 6 mots et hexadécimal, le serveur devrait accepter 
l'encodage décrit dans l'index B. Le répertoire de mot du dico standard ne doit 
cependant pas \^etre altéré ou oubli\'e. Afin d'\'eviter toute ambiguïté avec 
l'hexadécimal, on \'evitera les lettre de A à F. Le mots du nouveau dico sont 
sensibles à la casse et la casse doit être conservé.\\

        Au final, le serveur doit accepter :
        \begin{itemize}
            \item le format 6 mots basé dur les dico de la RFC1760 et de l'index 
D;
            \item l'hexadécimal;
            \item éventuellement le format 6 mots basé sur le dico alternatif 
(index B).
        \end{itemize}
  
  \subsection{Soumission et protocole de vérification}
    Tout est dans le titre. Comment le mot de passe est-il soumis et quelles 
sont les étapes suivies par le serveur pour déterminer la validité de celui-ci.
  
  \subsection{Synchronisation}
    Token et verificateurs peuvent ils perdre leur synchronisation. Comment ce 
problème est-il évité? Comment la synchronisation est elle rétablie?
  
  \subsection{Reinitialisation}
    Ne concerne a priori pas tous les systèmes. Expliquer les procédures a 
utiliser pour reinitialiser le secret et autres.

%------------------------------------------------------------------------------
\section{Analyse générale et sécurité}
  Partie cruciale du rapport. Merci de me la faire au petits oignons.

  \subsection{Avantages et intér\^ets}
    Tout est dans le titre. Merci d'\^etre exhaustif.
  
  \subsection{Inconvenients et limites}
    Tous est dans le titre. Merci d`\^etre exhaustif.
  
  \subsection{Considérations de sécurité}
    Trier tt ça :

    La clé doit contenir au moins 10 caractères pour réduire les risques de la 
retrouver par la méthode de recherche exhaustive ou par attaque du dictionnaire. 
Cependant toute implémentation du système OTP doit supporter des clés d'au moins 
10 à 63 caractères.

    Afin d'assurer l'interchangeabilité des générateurs d'otp, chacun doit 
supporter des clés de 10 à 63 caractères, des clés plus longues sont 
envisageables (mais attention à la compatibilité avec des générateurs plus 
restreints).

    \subsubsection{Attaques exhaustives}
      Analyse de la résistance aux attaques exhaustives. Un calcul et / ou 
démonstration seront certainement bien vus par votre chargé de mission 
      \og{}Coordination Recherche\fg{} ainsi que par les clients. N'ayez pas 
peur tous va bien se passer.
    
    \subsubsection{Attaques par collisions}
      Cette partie n'est pas forcement nécessaire. Se référer au résultats de la 
réunion du 19 Novembre.
    
    \subsubsection{Failles connues}
      Tout est dans le titre.
    
    \subsubsection{Précautions et préconisation}
      Tout est dans le titre.
  
%------------------------------------------------------------------------------
\section{Utilisations}
  On présente ici les possibles utilisation du système.
  \subsection{Cas concrets d'utilisation}
    Tout est dans le titre.
  
  \subsection{Cas d'utilisation envisagés}
    Cas possibles d'utilisation non encore exploité (On peut toujours r\^ever)

%------------------------------------------------------------------------------
\section{Conclusion}
  La conclusion
  \subsection{Utilisation dans le cadre du projet}
    Reprendre les éléments de l'analyse telles que le performances du système le 
niveau de sécurité et autre afin de définir si, et dans quels contextes, le 
système devrait\^etre utlisé dans le cadre du projet.
  
  \subsection{Perenité du système}
    A voir, pas forcément utile. Se référer aux résultats de la réunion du 19 
novembre.
    
\end{document}
