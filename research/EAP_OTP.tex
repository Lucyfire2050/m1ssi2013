\documentclass{article}

\usepackage[utf8]{inputenc}
\usepackage[T1]{fontenc}
\usepackage[francais]{babel}
\usepackage{lmodern}
\usepackage{amsmath}
\usepackage{amssymb}
\usepackage{mathrsfs}
\usepackage{graphics}
\usepackage{multicol}
\usepackage{algorithmic}
\usepackage{algorithm}

\title{EAP-POTP - The EAP Protected One-Time Password Protocol}
\author{Claire et Yves}

\begin{document}
\maketitle

\newpage
\tableofcontents
\newpage

\section{Introduction}
La méthode EAP-POTP, d'après la rfc 4793, décrit la connexion entre un Client ("peer") et un serveur grâce à un OTP.
Cette méthode est indépendante de l'algorithme de génération d'un OTP.

\section{Prérequis}
On a besoin d'EAP et d'un OTP (généré par exemple par un périphérique relié à la machine).
L'algorithme HMAC et SHA256 (par défault, d'autres version de SHA sont prise en charges) et de la fonction PBKDF2.

\section{Généralités}
Fournit une authentification unilatérale ou mutuelle.
Protocol de la couche liaison de données comme PPP.

\section{Approfondissement}

\subsection{Format d'un paquet EAP-POTP}

\begin{tabular}{  l  l  }
	bit 1 &: Code (1 si demande, 2 si réponse)\\
	bit 2 &: Identifier (Permet aux réponses de correspondre aux demandes)\\
	bit 3/4 &: Length (Indique la longueur du paquet entier)\\
	bit 5 &: Type (Spécifie l'algorithme OTP utilisé)\\
	bit 6 &: Reserved (Emplacement réservé pour une future utilisation)\\
	bit 7... &: TLV-based EAP-POTP message \\
\end{tabular}

\subsection{Déroullement type d'une authentification}

\begin{description}
   \item \textbf{Type de TLV :}
\item   1 - Version TLV : Contient la version de la méthode EAP-POTP supporté.
\item   2 - Server-Info TLV : Fournit les informations de sessions et des informations sur le serveur.
\item   3 - OTP TLV : Contient des informations sur la procédure d'authentification.
\item   4 - NAK TLV : Indique que le client ne prend pas en charge le paquet reçut.
\item   5 - New PIN TLV : Permet de demander un nouveau code PIN à l'utilisateur.
\item   6 - Confirm TLV : Confirme l'authentification du "peer", le serveuréssaie de s'authentifié auprès du "peer".
\item   7 - Vendor-Specific TLV : 
\item   8 - Resume TLV : envoyé par le client pour tenter la reprise de session.
\item   9 - User Identifier TLV : Donne un identifiant au détenteur du token.
\item 10 - Token Key Identifier TLV : Fournit l'identifiant au token pour qu'il puisse générer un OTP.
\item 11 - Time Stamp TLV : Envoyé par le client pour simplifier l'authentification, fournit le temps du token.
\item 12 - Counter TLV : Envoyé par le client pour simplifier l'authentification, fournit le compteur du token.
\item 13 - Keep-Alive TLV : Envoyé par le "peer" pour éviter les délais d'attente lors de l'attente d'un action utilisateur.
\item 14 - Protected TLV : Chiffre les TLV.
\item 15 - Crypto Algorithm TLV : Permet la négociation de l'algorithme de chiffrement.
\item 16 - Challenge TLV : Fournit le challenge tel que donné par le token.
\end{description}


%\begin{tabular}{|  l | p{8cm} | }
% \hline
%	Client (Peer) & Serveur EAP \\
% \hline
%	 & $\Leftarrow$ EAP-demande \\
%	 & Type=Identity \\
%	EAP-réponse $\Rightarrow$ &  \\
%	Type=Identity &  \\
%	 & $\Leftarrow$ EAP-demande \\
%	 & Type=OTP-X \\
%	 & Version TLV : Highest, Lowest \\
%	 & OTP TLV : envoie un chalenge* \\
%	 & \vspace*{7mm} informations sur la politique du serveur* \\
%	 & \vspace*{7mm} ...* \\
%	 & Si (reprise de session) alors \\
%	 & Server-Info TLV : renvoie l'identifiant serveur et l'identifiant session \\
%	 & fsi \\
%	%Si (POTP-X ou la version d'EAP-POTP n'est pas supporté(e)) alors &  \\
%	EAP-réponse Nak $\Rightarrow$ &  \\
%	Sinon &  \\
%	EAP-réponse $\Rightarrow$ &  \\
%	Version TLV : Plus haute version d'EAP-POTP possible & \\
%	fsi &  \\
%	%Si (session déjà existante && reprise de session possible) alors &  \\
%	Resume TLV &  \\
%	Sinon &  \\
%	un OTP est calculé en fonction de la requête EAP &  \\
%	fsi &  \\
%\hline
%\end{tabular}

Le serveur envoi une EAP-demande type=Identity
Le client envoi une EAP-réponse type=Identity

\begin{flushleft}
\begin{algorithm}
	\caption{Serveur}
	\label{TOTP:verif}
\begin{algorithmic}
\STATE envoi une EAP-demande avec le type de l'algorithme OTP, la plus haute et la plus basse version de la méthode EAP possible
\STATE cette EAP-demande peut potentiellement contenir un chalenge, des information sur la politique du serveur...
\IF{Cas d'une reprise de session}
	\STATE Server-Info TLV : renvoie l'identifiant serveur et l'identifiant session
\ENDIF

\end{algorithmic}
\end{algorithm}
\end{flushleft}

%\begin{minipage}[c]{0.45\linewidth}
\begin{flushright}
\begin{algorithm}
	\caption{Client}
	\label{TOTP:verif}
\begin{algorithmic}
\IF{POTP-X ou la version d'EAP-POTP n'est pas supporté(e)}
	\STATE envoi une EAP-réponse Nak
\ELSE
	\STATE envoi une EAP-réponse avec Version TLV : plus haute version d'EAP-POTP possible
\ENDIF

\end{algorithmic}
\end{algorithm}
\end{flushright}

%\end{minipage} \hfill
%\begin{minipage}[c]{0.45\linewidth}
\begin{flushleft}
\begin{algorithm}
	\caption{Serveur}
	\label{TOTP:verif}
\begin{algorithmic}
\IF{$EAP-réponse = Nak TLV$}
	\STATE la négociation a échoué
	\STATE le serveur peut tenter une autre méthode EAP
\ENDIF
\IF{c'est une reprise de session et le numéro de session est mauvais}
	\STATE le serveur indique au client que la reprise de session et impossible
	\STATE le serveur demande un nouvel OTP
\ENDIF
\IF{EAP-réponse type POTP-X est vide}
	\STATE envoie EAP-échec
\ENDIF
\IF{Authentification possible}
	\STATE accepte authentification
	\STATE envoie EAP-demande avec un Confirm TLV
\ELSE
	\STATE échec de l'authentification
	\STATE envoi EAP-failure
\ENDIF


\end{algorithmic}
\end{algorithm}
\end{flushleft}
%\end{minipage}


\begin{flushright}
\begin{algorithm}
	\caption{Client}
	\label{TOTP:verif}
\begin{algorithmic}
\IF{EAP-demande de notification}
	\STATE le client tente d'authentifier le serveur
	\IF{le serveur a bien été authentifié}
		\STATE envoi d'une EAP-réponse de type POTP-X avec un Confirm TLV
	\ELSE
		\STATE envoi une EAP-réponse de type POTP-X vide
	\ENDIF
\ENDIF

\end{algorithmic}
\end{algorithm}
\end{flushright}


\begin{flushleft}
\begin{algorithm}
	\caption{Serveur}
	\label{TOTP:verif}
\begin{algorithmic}
\IF{EAP-réponse type POTP-X est vide}
	\STATE envoie EAP-échec
\ENDIF

\end{algorithmic}
\end{algorithm}
\end{flushleft}

%\setlength\columnseprule{0.5pt} %pour un filet entre les colonnes

%\begin{multicols}{2}{
%Vue statique 
%\begin{itemize} \item{a} \item{b} \end{itemize} 

% \begin{itemize} \item{c} \item{d} \end{itemize}
%}
%\end{multicols}



* : non obligatoire
 Highest : donne la plus haute des versions d'EAP-POTP supportée.
 Lowest :  donne la plus basse des versions d'EAP-POTP supportée.

\subsection{Fonctions}
Pour la technologie HMAC on se réfère aux précédentes présentations.
PBKDF2 = password-based key derivation functions
SHA-256 = algorithme de hashage
$PBKDF2(otp,salt,pepper,authData,iteraction_count,Key Length) = K_MAC.K_ENC.MSK.EMSK$


$K_MAC$ : clé MAC utilisée pour l'authentification et l'intégrité mutuelle
$K_ENC$ : clé de cryptage utilisée pour protéger certaines données au cours de l'authentification
SRK : clé de reprise de session utilisée uniquement pour la reprise de session
MSK : clé de session maître
EMSK : clé de session maître étendue
pepper : utilisation précisée par le système. Génération aléatoire d'une séquence qui ne sera pas envoyé au serveur mais dont la taille sera envoyée au serveur. ceci ralentira le hacher, encore qu'on a une validation d'otp valide dans le temps.

Non envoyé en clair : OTP et pepper

\section{Analyse générale et sécurité}

\subsection{Avantages et intér\^ets}
  Inclut un grand nombre d'algorithmes OTP.
  Permet aussi bien une authentification mutuelle que unilatérale.
  Le mode protégé protège contre les attaques par rétrogradation de versions (version downgrade attacks).
  L'utilisation du poivre permet une meilleure sécurité pour un coût initial plus élevé mais égal sur le long terme.
  
  \subsection{Inconvenients et limites}
  L'utilisation d'un tunnel sécurisé est nécéssaire pour offrir une meilleure sécurité au système.
  Beaucoup d'attaques reste possible même si on connait les moyens pour les contrer.

\subsection{Considérations de sécurité}

\subsubsection{Attaques exhaustives (Brute-Force)}
    La session doit durer moins longtemps que le temps nécéssaire à une attaque exhaustive, afin que celle-ci n'abouttisse pas.

\subsubsection{Attaque de course (Race Attack)}
Il faut empécher la création simultanée de plusieurs sessions.

\subsubsection{Denial-of-service}
 Les attaques par déni de services sont possible, il faut donc ne pas tenir compte des OTP TLV avec un trop grand nombre d'itérations.

\subsection{Failles connues}
    Un ancien OTP permet de trouver certaines informations sur le PIN de l'utilisateur et représente donc toujours un risque (il faut donc s'assurer que le PIN de l'utilisateur change régulièrement).

   \textbf{ Pour la version avancée:}
    Protection par la variante de la méthode EAP : temps limité d'une session afin d'éviter que l'attaquant (man-in-the-middle) ai le temps de trouvé l'OTP valide.
    
    \textbf{Pour la version de base:}
    Une attaque par man-in-the-middle est possible (il faut être en authentification mutuelle pour s'en protéger).
    Détournement de session et attaques par rejeu (pour s'en protéger il faut être connécté de façon sécurisé à un serveur authentifié, en utilisant PEAPv2 par exemple).
    Lors de la reprise de sessions (en canal sécurisé) il reste possible de trouver la combinaison identifiant de session/OTP (pour s'en protéger il faut garder une trace du nombre de tentatives infructueuses de reprises).

\subsection{Précautions et préconisations}
    Dans le cas d'une reprise de session, celle-ci doit durer moins longtemps que le temps nécéssaire pour trouver l'OPT utiliser.\\
    La version de base comporte de nombreuses failles, il vaut mieux utiliser la variante protégée (protected variante).
    Le poivre doit être stocké de façon sur.

\section{Conclusion}

  \subsection{Utilisation dans le cadre du projet}
  La réussite (unilatérale ou mutuelle) de l'authentification grâce aux mots de passes jetables de deux systèmes via le réseau.
  
  \subsection{Perenité du système}
  La perennité est au plus égale à celle de la fonction de hachage utilisée.

\section{Définitions}
OTP-X : Désigne une méthode OTP quelleconque.
Poivre ou pepper : Valeur inconnue des tierces personnes (secret partagé) qu'on inclut dans la fonction de hachage.
PBKDF2 : password-based key derivation functions
SHA-256 : algorithme de hashage

\end{document}
