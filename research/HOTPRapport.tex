\documentclass{../res/univ-projet}

\title{\'Etude du syst\`eme d'OTP \og{}HOTP\fg{}}
\author{Les chats sauvages}

\projet{One Time Project}
\projdesc{\`Etude des syst\`emes de mots de passe jetable}
\filiere{M1SSI}
\logo{../res/logo_univ.png}

\usepackage[T1]{fontenc}
\usepackage[utf8]{inputenc}

\usepackage{algorithmic}
\usepackage{algorithm}

\begin{document}
\maketitle

\begin{abstract}
Ce document présente une analyse du système d'OTP \og{}HOTP\fg{}. Toutes les informations présentées et analysées proviennent des RFCs 4226 et 2104, truc et truc ainsi que 
de leurs correctifs. Le but de cet article est de déterminer dans quelles condition le système blabla est utilisable, sous quelles conditions et avec quelles garanties de 
sécurité.Ce documents sera mis en relation avec plusieurs autres afin de réaliser un comparatif entre les systèmes d'OTP majeurs.
\end{abstract}
\newpage
\tableofcontents
\newpage

\section{Prérequis}
Le système d'OTP \og{}HOTP\fg{} est un système d'authentification par mot de passe jetable qui étend le système le système OTP définit dans la RFC 2289. Le système OTP 
est étudié plus haut.
Le système HOTP repose principalement sur l'utilisation du mécanisme d'authentification de message HMAC. Ce mécanisme destiné à vérifier si un message n'a pas été altéré
et à authentifier l'expéditeur, repose lui m\^eme sur l'utilisation d'une fonction de hachage et d'un secret partagé. La ou un mécanisme classique de vérification 
d'intégrité de message se contente de hacher ce dernier pour obtenir une empreinte, le mécanisme HMAC utilise une clef secrète pour authentifier l'expéditeur. Cette clef 
est prise en compte de la manière suivante : 
\begin{center}
$HMAC_K(M) = h(K \oplus opad || h(K \oplus ipad || M))$ 
\end{center}
o\`u :\newline
$K$ est la clef secrète \newline
$opad$ = 0x36 répété un certain nombre de fois \newline
$ipad$ = 0x5C répété le m\^eme nombre de fois \newline
$\oplus$ est l'opération XOR \newline
$M$ est le message \newline


Dans le cadre de l'authentification de messages, le code HMAC du texte sera envoyé au destinataire sur un canal sécurisé différent de celui utilisé pour la transmission
du message. Le destinataire, une fois reçu et le message et le code HMAC pourra vérifier l'authenticité du message en calculant le code HMAC de ce dernier et en le 
comparannt avec celui envoyé.

Le mécanisme HMAC sera ici utilisé pour la génération de mots de passes jetables.

\section{Généralités}
Lors de l'utilisation d'un système de mot de passe jetable le client (ou plus généralement l'entité souhaitant s'authentifier sur le service) fournit au service un 
classique login et un mot de passe. Contrairement aux méthode classique d'authentification par login et mot de passe, ici le mot de passe n'est utilisable qu'une seule et 
unique fois et sera donc différent à chaque connexion.
Il existe deux grandes catégories de système de mots de passe jetable : les systèmes dits \og{}synchrones\fg{} et les systèmes dits \og{}asynchrones\fg{}. Les premiers 
reposent sur le partage d'un secret entre le client et le serveur ainsi sur un élément de synchronisation (i.e un compteur partagé, l'heure actuelle...). Les seconds 
au contraire ne reposent que sur le partage du secret et ne demande aucune synchronisations entre client et serveur.
Le système \og{}HOTP\fg{} étudié ici est synchrone. Sont principe de fonctionnement est le suivant :
\begin{itemize}
 \item L'utilisateur demande à son token\footnote{Le token est un élément matériel ou logiciel dont la seule fonction est de générer des mots de passe jetables.} de 
 générer un mot de passe jetable.
 \item Le token génère le mot de passe en calculant $HMAC_K(C)$, avec K le secret partagé et C le compteur de synchronisation, et incrémente le compteur.
 \item L'utilisateur récupère le mot de passe jetable fournit par le token et le passe au serveur d'authentification.
 \item Le serveur calcul la valeur de $HMAC_K(C)$, avec K le secret partagé et le compteur de synchronisation, et incrémente le compteur.
 \item Si la valeur calculée par le serveur coincide avec la valeur envoyée par l'utilisateur, ce dernier est authentifier. Dans le cas contraire, il est rejeté
\end{itemize}

Le principale intér\^et des méthodes d'authentification par mot de passe jetable, et donc de la méthode HOTP, est d'éviter les attaques dites \og{}par rejeu\fg{} durant 
lesquelles un attaquant récupère le mot de passe durant une communication client-serveur et l'utilise ensuite pour usurper l'identité du client.

\section{Approfondissement}
  Étudions maintenant le fonctionnement précis du système de mot de passe jetable \og{}HOTP\fg{}.
  \subsection{Génération et partage d'un secret}
    Le secret partagé entre le client et le serveur est le point le plus sensible du système d'authentification par mot de passe jetable. En effet, si un attaquant 
    parvient à s'emparer du secret d'un utilisateur, il sera capable de s'authentifier à sa place mettant en péril la sécurité du compte utilisateur. La méthode 
    \og{}HOTP\fg{} propose pour la génération des secrets deux méthodes : une méthode déterministe et une méthode aléatoire. Détaillons ces deux méthodes.
    
    \subsubsection{Méthode déterministe}
    Pour la méthode déterministe, une clef maitresse, notée $K_M$ doit \^etre enregistrée sur le serveur. Puisque les clefs secr\^etes de tous les clients seront dérivées 
    de $K_M$, cette clef devra \^etre stockée dans une zone très sécurisée voir inviolable. Lorsque qu'un nouveau client sera ajouté au service de connection, celui ci
    se verra muni d'une clef secrète, $K_i$, personnelle qui sera partagée avec le serveur. Cette clef sera calculée à partir de la clef $K_M$ et d'une information, $i$, 
    identifiant le client de la manière suivante : \newline
    \begin{center}
     $K_i = h(K_M || i)$ 
    \end{center}
    \hfill{},avec h une fonction de hachage.\newline
    Cette clef constitue un secr\^et unique et robuste si l'on respecte les conditions suivante :
    \begin{itemize}
     \item La fonction $h$ est une fonction de hachage non inversible (md5, SHA\_1...).
     \item Le secret $K_M$ est inviolable.
     \item Le paramètre $i$ est un identifiant unique à chaque utilisateur (ID matériel, numéro de série...)
    \end{itemize}
    
    \subsubsection{Méthode aléatoire}
    Pour la seconde méthode, une clef aléatoire est générée pour chaque nouveau client demandant l'association.
    Afin d'assurer la robustesse du secret, le générateur aléatoire doit \^etre choisi le moins prédictible possible. Il est en général préférable de choisir un génrateur 
    matériel, reposant sur l'aléatoire d'un phénomène physique (décomposition de noyaux atomiques, bruit thermique...). A défaut, un très bon générateur pseudo-aléatoire 
    reposant sur l'execution d'un algorithme (\emph{Yarrow}, \emph{Fortuna}, \emph{Blum Blum Shub}, \emph{ISAAC}...) peut \^etre utilisé. Quel que soit le générateur 
    utilisé, celui ci doit fournir une sortie imprédictible et une distribution de probabilité entre les clef aussi uniforme que possible.
  
  Quelle que soit la méthode de génération utilisée, la clef produite doit \^etre d'une longueure correspondant au standards de sécurité actuels. A l'heure ou ce rapport 
  est écrit, une longueure de clef de 128 bits parait \^etre un minimum. Choisir une longueure supérieur ou égal à 160 bits semble plus raisonnable et une clef de 256 
  bits ou plus fournit un très bon niveau de sécurité.
    
  \subsection{Génération d'un mot de passe jetable}
  La génération d'un OTP avec la méthode HOTP repose sur l'utilisation d'un compteur incrémental et de la clef secrète tous deux partagés entre le token et le service de 
  validation. Une fonction HMAC définie pour une fonction de hachage donnée est aussi untilisée pour produire la base de l'OTP.
  Au global, la génération de l'OTP se décompose en trois étapes principales selon l'algorithme \ref{HOTP:gene}.
  \begin{algorithm}
  \caption{Génération d'un OTP par HOTP}
    \label{HOTP:gene}
   
    \begin{algorithmic}
    \STATE $HVal \leftarrow HMAC(K, C)$
    \STATE $HVal \leftarrow tronc(HVal)$
    \STATE $HVal \leftarrow binToDec(HVal)$
    \STATE $C \leftarrow C + 1$
    \newline
    \RETURN $HVal$
    \end{algorithmic}
  \end{algorithm}
  
  Dans ce schéma, la première étape consiste à calculer la valeur de HMAC pour la clé secr\^ete partagée et en prenant pour message le compteur de synchronisation. La 
  longueur de la valeur obtenue dépend de la fonction de hachage utilisée. On aura par exemple une chaine de 160 bits en utilisant la fonction SHA\_1.
  On pourrait utiliser la sortie de la fonction HMAC comme OTP cependant, écrit en décimal cette valeur se représenterai sur 48 chiffres et serait peut pratique à entrer 
  pour un utilisateur. C'est pourquoi les étapes 2 et 3 vont permettre de réduire la longueur de l'OTP à une valeur sur 10 chiffres via les fonctions $tronc$ et 
  $binToDec$. La fonction $tronc$ est définie par l'algorithme \ref{HOTP:trunc}.
  
  \begin{algorithm}
    \caption{Réduction de la sortie de $HMAC$}
    \label{HOTP:trunc}
    
    \begin{algorithmic}
      \REQUIRE S une chaine de bit
      \ENSURE S est de longueur 31 bits
    \end{algorithmic}
  \end{algorithm}

  
  \subsection{Soumission et protocole de vérification}
  Tout est dans le titre. Comment le mot de passe est-il soumis et quelles sont les étapes suivient par le serveur pour déterminer la validité de celui-ci.
  
  \subsection{Synchronisation}
  Token et verificateurs peuvent ils perdre leur synchronisation. Comment ce problème est-il évité? Comment la synchronisation est elle rétablie?
  
  \subsection{Reinitialisation}
  Ne concerne a priori pas tous les systèmes. Expliquer les procédures a utiliser pour reinitialiser le secret et autres.
  
\section{Analyse générale et sécurité}
Partie cruciale du rapport. Merci de me la faire au petits oignons.

  \subsection{Avantages et intér\^ets}
  Tout est dans le titre. Merci d'\^etre exhaustif.
  
  \subsection{Inconvenients et limites}
  Tous est dans le titre. Merci d`\^etre exhaustif.
  
  \subsection{Considérations de sécurité}
  Le clou du spectacle.
    \subsubsection{Attaques exhaustives}
    Analyse de la résistance aux attaques exhaustives. Un calcul et / ou démonstration seront certainement bien vus par votre chargé de mission 
    \og{}Coordination Recherche\fg{} ainsi que par les clients. N'ayez pas peur tous va bien se passer.
    
    \subsubsection{Attaques par collisions}
    Cette partie n'est pas forcement necessaire. Se référer au résultats de la réunion du 19 Novembre.
    
    \subsubsection{Failles connues}
    Tout est dans le titre.
    
    \subsubsection{Précautions et préconisation}
    Tout est dans le titre.
    
\section{Utilisations}
On présente ici les possibles utilisation du système.
  \subsection{Cas concrets d'utilisation}
  Tout est dans le titre.
  
  \subsection{Cas d'utilisation envisagés}
  Cas possibles d'utilisation non encore exploité (On peut toujours r\^ever)
  
\section{Conclusion}
La conclusion
  \subsection{Utilisation dans le cadre du projet}
  Reprendre les éléments de l'analyse telles que le performances du sytème le niveau de sécurité et autre afin de définir si, et dans quels contextes, le système devrait
  \^etre utlisé dans le cadre du projet.
  
  \subsection{Perenité du système}
  A voir, pas forcément utile. Se référer aux résultats de la réunion du 19 novembre.
    
\end{document}
