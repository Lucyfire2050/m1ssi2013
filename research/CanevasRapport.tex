\documentclass{../res/univ-projet}

\title{\'Etude du syst\`eme d'OTP machin}
\author{Vous}

\projet{Nom \`a d\'efinir}
\projdesc{\`Etude des syst\`emes de mots de passe jetable}
\filiere{M1SSI}
\logo{../res/logo_univ.png}

\usepackage[T1]{fontenc}
\usepackage[utf8]{inputenc}

\begin{document}
\maketitle

\begin{abstract}
Ce document présente une analyse du système d'OTP machin. Toutes les informations présentées et analysées proviennent des RFCs machin, truc et truc ainsi que de leurs 
correctifs. Le but de cet article est de déterminer dans quelles condition le système blabla est utilisable, sous quelles conditions et avec quelles garanties de sécurité.
Ce documents sera mis en relation avec plusieurs autres afin de réaliser un comparatif entre les systèmes d'OTP majeurs.
\end{abstract}
\newpage
\tableofcontents
\newpage

\section{Prérequis}
Dans cette section présenter les technos et concepts sous-jacent(e)s au systèmes étudiés.

\section{Généralités}
Dans cette section présenter les principe et fonctionnements généraux des systèmes étudiés.

\section{Approfondissement}
  Dans cette section tout doit \^etre étudié en détail.
  \subsection{Generation et partage d'un secret}
  Tout est dans le titre
  
  \subsection{Génération d'un mot de passe jetable}
  Tous est dans le titre
  
  \subsection{Soumission et protocole de vérification}
  Tout est dans le titre. Comment le mot de passe est-il soumis et quelles sont les étapes suivient par le serveur pour déterminer la validité de celui-ci.
  
  \subsection{Synchronisation}
  Token et verificateurs peuvent ils perdre leur synchronisation. Comment ce problème est-il évité? Comment la synchronisation est elle rétablie?
  
  \subsection{Reinitialisation}
  Ne concerne a priori pas tous les systèmes. Expliquer les procédures a utiliser pour reinitialiser le secret et autres.
  
\section{Analyse générale et sécurité}
Partie cruciale du rapport. Merci de me la faire au petits oignons.

  \subsection{Avantages et intér\^ets}
  Tout est dans le titre. Merci d'\^etre exhaustif.
  
  \subsection{Inconvenients et limites}
  Tous est dans le titre. Merci d`\^etre exhaustif.
  
  \subsection{Considérations de sécurité}
  Le clou du spectacle.
    \subsubsection{Attaques exhaustives}
    Analyse de la résistance aux attaques exhaustives. Un calcul et / ou démonstration seront certainement bien vus par votre chargé de mission 
    \og{}Coordination Recherche\fg{} ainsi que par les clients. N'ayez pas peur tous va bien se passer.
    
    \subsubsection{Attaques par collisions}
    Cette partie n'est pas forcement necessaire. Se référer au résultats de la réunion du 19 Novembre.
    
    \subsubsection{Failles connues}
    Tout est dans le titre.
    
    \subsubsection{Précautions et préconisation}
    Tout est dans le titre.
    
\section{Utilisations}
On présente ici les possibles utilisation du système.
  \subsection{Cas concrets d'utilisation}
  Tout est dans le titre.
  
  \subsection{Cas d'utilisation envisagés}
  Cas possibles d'utilisation non encore exploité (On peut toujours r\^ever)
  
\section{Conclusion}
La conclusion
  \subsection{Utilisation dans le cadre du projet}
  Reprendre les éléments de l'analyse telles que le performances du sytème le niveau de sécurité et autre afin de définir si, et dans quels contextes, le système devrait
  \^etre utlisé dans le cadre du projet.
  
  \subsection{Perenité du système}
  A voir, pas forcément utile. Se référer aux résultats de la réunion du 19 novembre.
    
\end{document}
