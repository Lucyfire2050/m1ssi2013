\documentclass{beamer}
\usetheme{Boadilla}
\usepackage{color} 
\usepackage{graphicx}
\usepackage[utf8]{inputenc}
\usepackage[T1]{fontenc}
\usepackage[francais]{babel}
\usepackage{dsfont}
\usepackage{graphicx}
            

\title{Étude du système OTP}

\author{D.\bsc{Picard}, A.\bsc{Smondack}, C.\bsc{Hardouin}, Y.\bsc{Adegoloye}, B.\bsc{Zigh}, G.\bsc{Ferry}, M.\bsc{Michotte}} 
\institute{Université de Rouen}
\date{22 Janvier 2014} 

\AtBeginSection[] 
{ 
\begin{frame}  
\frametitle{Plan} 
\tableofcontents[currentsection,hideothersubsections]
\end{frame}
} 


\begin{document}

\begin{frame} 
\titlepage
\end{frame}


\section{Introduction}
\begin{frame}
\frametitle{Qu'est-ce qu'un OTP ?}
\begin{block}{Définition}
    Un OTP est un mot de passe jetable, c'est à dire qu'il satisfait les deux 
  critères suivants:
  \begin{itemize}
    \item Il n'est pas prédictible.
    \item Il n'est valide que pour une unique session.
  \end{itemize}
\end{block}

\begin{block}{Utilité}
  \begin{itemize}
    \item Permettre une authentification forte.
    \item Éviter les attaques par rejeu.
  \end{itemize}
\end{block}
\end{frame}

\section{Présentation des Protocoles}
\subsection{OTP}
\begin{frame}
\frametitle{OTP}
\begin{block}{Principe}
\begin{itemize}
 \item Le client calcule n hachés successifs à partir d'une clé secrète.
 \item Le système est initialisé avec le $n^{ième}$ haché de la séquence.
 \item L'utilisateur s'authentifie avec le haché précédent, et ainsi de suite.
\end{itemize}
\end{block}

\begin{block}{Problème}
Une ambiguïté a été relevée dans la RFC.
\end{block}
\end{frame}

\subsection{HOTP-TOTP}
\begin{frame}
\frametitle{HOTP-TOTP}
 \begin{block}{Principe}
 \begin{itemize}
 \item Utilisation d'HMAC pour générer les OTP.
 \item Fonctionnement synchrone avec compteur de synchronisation.
 \item TOTP utilise la date (timestamp) au lieu d'un compteur.
 \end{itemize}
 \end{block}
\end{frame}

\subsection{OTPW}
\begin{frame}
\frametitle{OTPW}
  \begin{block}{Principe}
   \begin{itemize}
    \item Génération d'OTP basé sur l'état de la machine à un instant t.
    \item Pour se connecter il faut: une liste de suffixes et un mot de passe.
    \item Authentification par défi/réponse.
   \end{itemize}
  \end{block}
\end{frame}


\section{Comparatifs}
\subsection{Comparatif sécurité}
\begin{frame}
  \frametitle{Comparatif}
  \begin{tabular}{|l|c|c|c|c|}
 \hline
 & OTP & HOTP & OTPW & TOTP\\
 \hline
 Année & 1998 & 2005 & 2006 & 2011 \\
 \hline
 RFC & 2289 & 4226 & NON & 6238 \\
 \hline
 Hachage\footnote{Recommandé par les RFC} & MD5(SHA-1) & SHA-1 & RipeMD-160 & SHA-1(SHA2)\\
 \hline
 Exhaustif & $1/2^{64}$ & $9*10^{-6} $ & $1/2^{102}$ ou $1/2^{48}$ & $9*10^{-6}$\\
 \hline
 Collisions & $1/2^{128}$ & NON & $1/2^{72}$ & NON\\
 \hline
 Failles \footnote{Dans le cadre d'une utilisation respectant les recommandations.} & Aucune & Aucune& Aucune &Aucune\\
 \hline
  \end{tabular}
\end{frame}
\subsection{Comparatif système}
\begin{frame}
\frametitle{Comparatif système}
\begin{tabular}{|l|c|c|c|c|}
 \hline
 & OTP & HOTP & OTPW & TOTP\\
 \hline
 Synchrone & OUI/NON & OUI & NON & OUI \\
 \hline
 Réinitialisation & OUI & NON & OUI & NON \\
 \hline
 Performances & Mauvaises & Moyennes & Bonnes & Moyennes\\
 \hline
  \end{tabular}
\end{frame}

\section{Utilisations}
\subsection{Utilisations courantes}
\begin{frame}
\frametitle{Utilisations courantes}
\begin{itemize}
 \item OATH / LinOTP / Google Authenticator
 \item EAP-POTP
\end{itemize}
\end{frame}


\subsection{EAP}
  \begin{frame}
  \frametitle{EAP}
        \begin{block}{Extensible Authentication Protocol }
\begin{itemize}
\item est un framework d'authentification universelle
\item fréquemment utilisé dans les réseaux sans fil et des connexions point-à-point.
\item est défini par la RFC 3748.
\end{itemize}
    \end{block}
        \begin{block}{Fonctionnement}
EAP n'est pas un mécanisme d'authentification spécifique. Il ne définit que les formats de message.
Extensibilité => plus
\end{block}
  \end{frame}
  
\subsection{EAP-POTP}
\begin{frame}
\frametitle{EAP-POTP: Authentification}
\begin{block}{Fonctionnalités}
\begin{itemize}
\item Authentification unilatérale ou mutuelle
\item Reprise de session
\item Mode protégé

\end{itemize}
\end{block}
\end{frame}

\begin{frame}
\frametitle{EAP-POTP: Reprise de session}
\begin{block}{Résumé}
La reprise de session permet de reprendre une session courante et donc de réduire le nombre d'opérations (par exemple, on ne recalcule pas un nouvel OTP).
\end{block}

\end{frame}

\begin{frame}
\frametitle{EAP-POTP: Mode Protégé}
\begin{itemize}
\item "MAC" (algorithme MAC négocié) est l'algorithme utilisé en mode protégé pour l'authentification et la garantie d'intégrité des données.
\item Lorsque le mode protégé est activé:\\
PBKDF2 = password-based key derivation functions\\
SHA-256 = algorithme de hachage\\
PBKDF2(otp,salt,pepper,authData,iteration\_count,Key\_Length) = K\_MAC.K\_ENC.MSK.EMSK
\end{itemize}

\end{frame}

\subsection{Considérations de sécurité}
\begin{frame}
\frametitle{Considérations de sécurité}
\begin{block}{Bilan}
La plupart des attaques sont évitées avec la variante protégée.
\end{block}
\end{frame}

\section{Conclusion}
\begin{frame}
\frametitle{Conclusion}
\begin{block}{Choix des implémentations}
\begin{itemize}
\item HOTP
\item TOTP
\item OTPW
\end{itemize}
\end{block}
\end{frame}



\end{document}
