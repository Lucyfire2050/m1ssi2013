\documentclass{beamer}
\usetheme{Boadilla}
\usepackage{color} 
\usepackage{graphicx}
\usepackage[utf8]{inputenc}
\usepackage[T1]{fontenc}
\usepackage[francais]{babel}
\usepackage{dsfont}
\usepackage{graphicx}
            

\title{\'Etude du syst\`eme d'OTP}

\author{D.\bsc{Picard}, A.\bsc{Smondack}, C.\bsc{Hardouin}, Y.\bsc{Adegoloye}, B.\bsc{Zigh}, G.\bsc{Ferry}, M.\bsc{Michotte}} 
\institute{Université de Rouen}
\date{22 Janvier 2014} 

\AtBeginSection[] 
{ 
\begin{frame}  
\frametitle{Plan} 
\tableofcontents[currentsection,hideothersubsections]
\end{frame}
} 


\begin{document}

\begin{frame} 
\titlepage
\end{frame}


\section{Introduction}
\begin{frame}
\frametitle{Qu'est-ce qu'un OTP ?}
\begin{block}{Définition}
    Un OTP est un mot de passe jetable, c'est à dire qu'il satisfait les deux 
  critères suivants:
  \begin{itemize}
    \item Il n'est pas prédictible
    \item Il n'est valide que pour une unique session.
  \end{itemize}
\end{block}

\begin{block}{Utilité}
  \begin{itemize}
    \item Permettre une authentification forte.
    \item Éviter les attaques par rejeu.
  \end{itemize}
\end{block}
\end{frame}

\section{Présentation des Protocoles}
\subsection{OTP}
\begin{frame}
\frametitle{OTP}
\begin{block}{Principe}
\begin{itemize}
 \item Le client calcule n hachés successifs à partir d'une clé secrète.
 \item Le système est initialisé avec le $n^{ième}$ haché de la séquence.
 \item L'utilisateur s'authentifie avec le haché précédent, et ainsi de suite.
\end{itemize}
\end{block}

\begin{block}{Problème}
Une ambiguïté a été relevée dans la RFC.
\end{block}
\end{frame}

\subsection{HOTP-TOTP}
\begin{frame}
\frametitle{HOTP-TOTP}
 A MEUBLER
\end{frame}

\subsection{OTPW}
\begin{frame}
\frametitle{OTPW}
  \begin{block}{Principe}
   \begin{itemize}
    \item Génération d'OTP basé sur l'état de la machine à un instant t.
    \item Pour se connecter il faut: une liste de suffixes et un mot de passe.
    \item Authentification par défi/réponse.
   \end{itemize}
  \end{block}
\end{frame}


\section{Comparatif}
\begin{frame}
  \frametitle{Comparatif}
  \begin{tabular}{|l|c|c|c|c|}
 \hline
 & OTP & HOTP & TOTP & OTPW\\
 \hline
 RFC (année) & & & &\\
 \hline
 Hachage recommandé & & & &\\
 \hline
 Attaque exhaustive & vOTP & & & \\
 \hline
 Attaque collision & & & &\\
 \hline
 Failles connues \footnote{Dans le cadre d'une utilisation respectant les recommandations.} & & & &\\
 \hline
  \end{tabular}
\end{frame}

\section{Utilisation}
\begin{frame}
\frametitle{Utilisation}
\begin{itemize}
 \item OATH / LinOTP / Google Authenticator
 \item EAP-POTP
\end{itemize}
\end{frame}


\subsection{EAP}
  \begin{frame}
  \frametitle{EAP}
    \begin{block}{A quoi ça sert?}
    zdazd
    \end{block}
    
  \end{frame}


\end{document}
