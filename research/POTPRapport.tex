\documentclass{../res/univ-projet}

\title{\'Etude du syst\`eme d'EAP POTP}
\author{Yves et Claire}

\projet{Nom \`a d\'efinir}
\projdesc{\`Etude des syst\`emes de mots de passe jetable}
\filiere{M1SSI}
\logo{../res/logo_univ.png}

\usepackage[T1]{fontenc}
\usepackage[utf8]{inputenc}

\begin{document}
\maketitle

\begin{abstract}
Ce document présente une analyse du système d'EAP POTP. Toutes les informations présentées et analysées proviennent des RFCs 4793 ainsi que de leurs
correctifs. Le but de cet article est de déterminer dans quelles conditions le système EAP POTP est utilisable, sous quelles conditions et avec quelles garanties de sécurité.
Ce documents sera mis en relation avec plusieurs autres afin de réaliser un comparatif entre les systèmes d'OTP majeurs.
\end{abstract}
\newpage
\tableofcontents
\newpage

\section{Prérequis}
Dans cette section présenter les technos et concepts sous-jacent(e)s au systèmes étudiés.
On a besoin d'EAP et d'un OTP (généré par exemple par un périphérique relié à la machine).
HMAC et SHA256.

\section{Généralités}
Dans cette section présenter les principes et fonctionnements généraux des systèmes étudiés.
Fournit une authentification unilatérale ou mutuelle.
Protocol de la couche liaison de données comme PPP.

\section{Approfondissement}
  Dans cette section tout doit \^etre étudié en détail.
  \subsection{Generation et partage d'un secret}
  Tout est dans le titre
  
  \subsection{Génération d'un mot de passe jetable}
  Tous est dans le titre
  
  Par un périphérique connecté à la machine ou un logiciel séparé.
  
  \subsection{Soumission et protocole de vérification}
  Tout est dans le titre. Comment le mot de passe est-il soumis et quelles sont les étapes suivient par le serveur pour déterminer la validité de celui-ci.
  
  \subsection{Synchronisation}
  Token et verificateurs peuvent ils perdre leur synchronisation. Comment ce problème est-il évité? Comment la synchronisation est elle rétablie?
  
  \subsection{Reinitialisation}
  Ne concerne a priori pas tous les systèmes. Expliquer les procédures a utiliser pour reinitialiser le secret et autres.
  
\section{Analyse générale et sécurité}
Partie cruciale du rapport. Merci de me la faire au petits oignons.

  \subsection{Avantages et intér\^ets}
  Tout est dans le titre. Merci d'\^etre exhaustif.
  Inclut un grand nombre d'algorithmes OTP.
  Permet aussi bien une authentification mutuelle que unilatérale.
  Le mode protégé protège contre les attaques par rétrogradation de versions (version downgrade attacks).
  L'utilisation du poivre permet une meilleure sécurité pour un coût initial plus élevé mais égal sur le long terme.
  
  \subsection{Inconvenients et limites}
  Tous est dans le titre. Merci d`\^etre exhaustif.
  L'utilisation d'un tunnel sécurisé offre une meilleure sécurité au système.
  Beaucoup d'attaques reste possible même si on connait les moyens pour les contrer.
  
  \subsection{Considérations de sécurité}
  Le clou du spectacle.
    \subsubsection{Attaques exhaustives}
    Analyse de la résistance aux attaques exhaustives. Un calcul et / ou démonstration seront certainement bien vus par votre chargé de mission
    \og{}Coordination Recherche\fg{} ainsi que par les clients. N'ayez pas peur tous va bien se passer.
    La sesion doit durer moins longtemps que le temps nécéssaire à une attaque exhaustive.
    
    \subsubsection{Attaques par collisions}
    Cette partie n'est pas forcement necessaire. Se référer au résultats de la réunion du 19 Novembre.
    
    \subsubsection{Failles connues}
    Tout est dans le titre.
    Un ancien OTP permet de trouver certaines informations sur le PIN de l'utilisateur et représente donc toujours un risque (il faut donc s'assurer que le PIN de l'utilisateur change régulièrement).
    Les attaques par déni de services sont possible, il faut donc ne pas tenir compte des OTP TLV avec un trop grand nombre d'itérations.
    Attaque de course : il faut empécher la création simultanée de plusieurs sessions.
    
    Pour la version avancée:
    Protection par la variante de la méthode EAP : temps limité d'une session afin d'éviter que l'attaquant (man-in-the-middle) ai le temps de trouvé l'OTP valide.
    
    Pour la version de base:
    Une attaque par man-in-the-middle est possible (il faut être en authentification mutuelle pour s'en protéger).
    Détournement de session et attaques par rejeu (pour s'en protéger il faut être connécté de façon sécurisé à un serveur authentifié, en utilisant PEAPv2 par exemple).
    Lors de la reprise de sessions (en canal sécurisé) il reste possible de trouver la combinaison identifiant de session/OTP (pour s'en protéger il faut garder une trace du nombre de tentatives infructueuses de reprises).
    
    \subsubsection{Précautions et préconisations}
    Dans le cas d'une reprise de session, celle-ci doit durer moins longtemps que le temps nécéssaire pour trouver l'OPT utiliser.
    La version de base comporte de nombreuses failles, il vaut mieux utiliser la variante protégée (protected variante).
    Le poivre doit être stocké de façon sur.
    
\section{Utilisations}
On présente ici les possibles utilisation du système.
  \subsection{Cas concrets d'utilisations}
  Tout est dans le titre.
  La réussite de l'authentification grâce aux mots de passes jetables de deux systèmes via le wifi.
  Utilisation pour l'identification d'un téléphone mobile via le wifi par exemple?
  
  \subsection{Cas d'utilisations envisagés}
  Cas possibles d'utilisations non encore exploités (On peut toujours r\^ever)
  
\section{Conclusion}
La conclusion
  \subsection{Utilisation dans le cadre du projet}
  Reprendre les éléments de l'analyse telles que le performances du sytème le niveau de sécurité et autre afin de définir si, et dans quels contextes, le système devrait
  \^etre utlisé dans le cadre du projet.
  La réussite de l'authentification grâce aux mots de passes jetables (mutuelle???) de deux systèmes via le wifi.
  
  \subsection{Perenité du système}
  A voir, pas forcément utile. Se référer aux résultats de la réunion du 19 novembre.
  La perennité est au plus égale à celle de la fonction de hachage utilisée.
    
\end{document}
