\documentclass{../res/univ-projet}

\title{\'Etude du syst\`eme d'OTP \og{}TOTP\fg{}}
\author{Les chats sauvages}

\projet{One Time Project}
\projdesc{\'Etude des syst\`emes de mots de passe jetable}
\filiere{M1SSI}
\logo{../res/logo_univ.png}

\usepackage[T1]{fontenc}
\usepackage[utf8]{inputenc}

\usepackage{algorithmic}
\usepackage{algorithm}

\begin{document}
\maketitle

\begin{abstract}
Ce document présente une analyse du système d'OTP \og{}TOTP\fg{}. Toutes les informations présentées et analysées proviennent de la RFC 6238 ainsi que de leurs 
correctifs. Le but de cet article est de déterminer dans quelles conditions le système blabla est utilisable, sous quelles conditions et avec quelles garanties de sécurité.
Ce documents sera mis en relation avec plusieurs autres afin de réaliser un comparatif entre les systèmes d'OTP majeurs.
\end{abstract}
\newpage
\tableofcontents
\newpage

\section{Prérequis}
Le système d'authentification par mot de passe jetable \og{}TOTP\fg est une extension du système \og{}HOTP\fg{}, par conséquent, il repose sur les m\^emes concepts. Ces 
derniers ayant déjà été donnés dans le document relatif à \og{}HOTP\fg{}, nous ne les redétaillerons pas de nouveau ici.

\section{Généralités}
Le principe général de fonctionnement du système \og{}TOTP\fg{} est strictement identique à celui du système \og{}HOTP\fg{} à une différence pr\^et. Ici, un paramètre 
temporel est utilisé en lieux et places du compteur de synchronisation. Ainsi, la fonction $HMAC_K$ utilisée par \og{}HOTP\fg{} ne sera plus appelée avec pour paramètre la 
valeur du compteur de synchro mais avec la valeur $T = \lfloor{}\frac{curTime - T_0}{X}\rfloor{}$, o\`u $curTime$ représente le temps Unix actuel, $T_0$ le temps Unix \og{}
d'origine\fg{} et $X$ un quantum de temps arbitraire.

\section{Approfondissement}
  \subsection{Generation et partage d'un secret}
    Le problème de génération et de partage de secret pour \og{}TOTP\fg{} est strictement identique à celui de \og{}HOTP\fg{} reportez vous au document relatif à ce 
    dernier pour un détails de méthodes.
    
  \subsection{Génération d'un mot de passe jetable}
    La génération des mots de passe jetables pour le système \og{}TOTP\fg{} est similaire à celle de \og{}HOTP\fg{} hormis sur le point du calcul de $HMAC$. En sommes, la 
    génération peut \^etre décrite par l'algorithme \ref{TOTP:gene}.
    \begin{algorithm}
      \caption{Génération d'un OTP par TOTP}
      \label{TOTP:gene}
   
      \begin{algorithmic}
	\REQUIRE $K$ une clef secrète
	\STATE $T \leftarrow \lfloor{}\frac{curTime - T_0}{X}\rfloor{}$
	\STATE $HVal \leftarrow HMAC(K, C)$
	\STATE $HVal \leftarrow tronc(HVal)$
	\STATE $HVal \leftarrow binToDec(HVal)$
	\newline
	\RETURN $HVal \bmod 10^L$ // L : longueur souhaitée de l'OTP
      \end{algorithmic}
    \end{algorithm}
    
    Les fonctions $tronc$ et $binToDec$ utilisées dans l'algorithme \ref{TOTP:gene} sont strictement identiques à celle utilisée dans l'algorithme de génération de 
    \og{}HOTP\fg{}. Référez vous au document traitant de \og{}HOTP\fg{} pour plus de détails.
    
  \subsection{Soumission et protocole de vérification}
    La vérification de mot de passe jetable suis le m\^eme schéma général que pour le système \og{}HOTP\fg{}. On trouve cependant quelque détails différents. Ces derniers 
    sont résumés dans l'algorithme \ref{TOTP:verif}.
    \begin{algorithm}
      \caption{Vérification d'un mot de passe jetable.}
      \label{TOTP:verif}
      
      \begin{algorithmic}
	\STATE $attentNB \leftarrow 0$
	\STATE $T \leftarrow \lfloor{}\frac{curTime - T_0 + \Delta_t}{X}\rfloor{}$
	\STATE $K \leftarrow$ clef secrète de l'utilisateur
	\STATE $OTP_1 \leftarrow TOTP_K(C)$
	\WHILE{$attentNB < MAX\_ATTENT$}
	  \STATE $OTP_2 \leftarrow$ recuperer\_OTP
	  \STATE $attentNB \leftarrow attentNB + 1$
	  \IF{$OTP_1 = OTP_2$}
	    \STATE accepter l'utilisateur
	  \ELSE
	    \IF{resynchro possible}
	      \STATE $resynchro$
	      \STATE accepter l'utilisateur
	    \ENDIF
	  \ENDIF
	\ENDWHILE
	\STATE $verouiller$
	\STATE $prevenir$
      \end{algorithmic}
    \end{algorithm}
    Ainsi, on retrouve bien le schéma générale de verification vu précédemment, utilisant la valeur $T$ décrite plus haut au lieu de celle du compteur de synchronisation.
    Cependant, dans cet algorithme, la valeur de $T$ est légèrement altérée afin d'y faire apparaitre la valeur $\Delta_t$. Cette dèrnière est utilisée pour garantir la 
    synchronisation temporelle entre le client et le sereur. Le fonctionnement et le calcul de cette valeur est décrit dans la section \ref{OTP:syncSec}.
    
  \subsection{Synchronisation}
  \label{OTP:syncSec}
    De m\^eme que pour le système \og{}HOTP\fg{}, il est possible que le serveur et le client perdent leur synchronisation temporelle. Il suffit pour cela que soit 
    l'horloge du client soit celle du serveur soit plus rapide que son homologue. Bien que l'utilisation de plus en plus répandue des serveur de synchronisation 
    temporelle mondiaux tendent à réduire les possibilités de décalage des horloge, ce cas reste à envisager. C'est pourquoi on retrouve les fonction \og{}resynchro 
    possible\fg{} et $resynchro$, chargées d'assurer la cohérence des vérification d'OTP en cas de désynchronisation d'horloge. Les descriptions de ces fonctions sont 
    données dans les algorithme \ref{TOTP:isSynch} et \ref{TOTP:synch}.
    \begin{algorithm}
      \caption{Vérification de la possibilité de resynchronisation}
      \label{TOTP:isSynch}
      
      \begin{algorithmic}
	\REQUIRE $OTP$ la valeur de l'OTP fournie par l'utilisateur
	\REQUIRE $K$ la clef secrète de l'utilisation
	\STATE $T \leftarrow \lfloor{}\frac{curTime - T_0 + \Delta_t}{X}\rfloor{}$
	\STATE $MAX\_BWD \leftarrow b$ //$MAX\_BWD$ contient le nombre maximal de valeurs a verifier vers l'arrière
	\STATE $MAX\_FWD \leftarrow a$ //$MAX\_FWD$ contient le nombre maximal de valeur a verifier vers l'avant
	\FOR{$i$ de $-MAX\_BWD$ à $MAX\_FWD$}
	  \STATE $OTP2 \leftarrow HOTP_K(T + i)$
	  \IF{$OTP = OTP2 AND not\_used(OTP)$}
	    \RETURN vrai
	  \ENDIF
	\ENDFOR
	\RETURN faux
      \end{algorithmic}
    \end{algorithm}
    
    \begin{algorithm}
      \caption{Resynchronisation}
      \label{TOTP:synch}
      
      \begin{algorithmic}
	\REQUIRE $OTP$ la valeur de l'OTP fournie par l'utilisateur
	\REQUIRE $K$ la clef secrète de l'utilisation
	\STATE $T \leftarrow \lfloor{}\frac{curTime - T_0 + \Delta_t}{X}\rfloor{}$
	\STATE $MAX\_BWD \leftarrow b$ //$MAX\_BWD$ contient le nombre maximal de valeurs a verifier vers l'arrière
	\STATE $MAX\_FWD \leftarrow a$ //$MAX\_FWD$ contient le nombre maximal de valeur a verifier vers l'avant
	\FOR{$i$ de $-MAX\_BWD$ à $MAX\_FWD$}
	  \STATE $OTP2 \leftarrow HOTP_K(T + i)$
	  \IF{$OTP = OTP2 AND not\_used(OTP)$}
	    \STATE $\Delta_t \leftarrow \Delta_t + i * X$
	    \STATE accepter l'utilisateur
	  \ENDIF
	\ENDFOR
	\STATE rejeter l'utilisateur
      \end{algorithmic}
    \end{algorithm}
    
    Le fonctionnement de ces algorithmes est simple. Il s'agit de rechercher dans une fenetre temporelle donnée une valeur d'OTP correspondant à celle donnée par 
    l'utilisateur. Si une telle valeur est trouvée, on calcul le décalage temporel entre les deux parties et on l'ajoute à la valeure courante du décalage $\Delta_t$.
    Sinon, l'utilisateur est rejeté.
    Comme pour le système \og{}HOTP\fg{}, ces deux fonctions pourront, en pratique, \^etre fusionnées en une seule réalisant à la fois la vérification et la 
    synchronisation.
  
\section{Analyse générale et sécurité}
Partie cruciale du rapport. Merci de me la faire au petits oignons.

  \subsection{Avantages et intér\^ets}
  Tout est dans le titre. Merci d'\^etre exhaustif.
  
  \subsection{Inconvenients et limites}
  Tous est dans le titre. Merci d`\^etre exhaustif.
  
  \subsection{Considérations de sécurité}
  Le clou du spectacle.
    \subsubsection{Attaques exhaustives}
    Analyse de la résistance aux attaques exhaustives. Un calcul et / ou démonstration seront certainement bien vus par votre chargé de mission 
    \og{}Coordination Recherche\fg{} ainsi que par les clients. N'ayez pas peur tous va bien se passer.
    
    \subsubsection{Attaques par collisions}
    Cette partie n'est pas forcement necessaire. Se référer au résultats de la réunion du 19 Novembre.
    
    \subsubsection{Failles connues}
    Tout est dans le titre.
    
    \subsubsection{Précautions et préconisation}
    Tout est dans le titre.
    
\section{Utilisations}
On présente ici les possibles utilisation du système.
  \subsection{Cas concrets d'utilisation}
  Tout est dans le titre.
  
  \subsection{Cas d'utilisation envisagés}
  Cas possibles d'utilisation non encore exploité (On peut toujours r\^ever)
  
\section{Conclusion}
La conclusion
  \subsection{Utilisation dans le cadre du projet}
  Reprendre les éléments de l'analyse telles que le performances du sytème le niveau de sécurité et autre afin de définir si, et dans quels contextes, le système devrait
  \^etre utlisé dans le cadre du projet.
  
  \subsection{Perenité du système}
  A voir, pas forcément utile. Se référer aux résultats de la réunion du 19 novembre.
    
\end{document}
