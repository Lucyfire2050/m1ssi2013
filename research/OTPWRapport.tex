\documentclass{../res/univ-projet}

\title{\'Etude du syst\`eme OTPW}
\author{Damien \bsc{PICARD}}

\projet{OTP}
\projdesc{\`Etude des syst\`emes de mots de passe jetable}
\filiere{M1SSI}
\logo{../res/logo_univ.png}

\usepackage[T1]{fontenc}
\usepackage[utf8]{inputenc}
\usepackage{algorithm}
\usepackage{algorithmic}


\begin{document}
\maketitle

\begin{abstract}
    Ce document présente une analyse du système d'OTPW. Toutes les
    informations présentées et analysées proviennent 
    \hyperref[de ce site]{http://www.cl.cam.ac.uk/~mgk25/otpw.html}. Le
    but de cet
    article est de déterminer dans quelles condition le
    système OTPW est utilisable, sous quelles conditions et avec quelles garanties
    de sécurité.
    Ce documents sera mis en relation avec plusieurs autres afin de réaliser un
    comparatif entre les systèmes d'OTP majeurs.
\end{abstract}
\newpage
\tableofcontents
\newpage

\section{Prérequis}
\begin{itemize}
    \item Pouvoir stocker une liste de chaine de caractères sur la machine serveur.
        environ 4 kilo octets. 
\end{itemize}

\section{Généralités}
Ce système repose sur un fichiers de chaînes de caractères généré à partir
d'un mot de passe préfix initial. Les mots de passes suffixe sont générés par un
générateur aléatoire basé sur l'algorithme \verb?RIPEMD-160? initialisé en
hashant la sortie de moult commandes du shell. Les mots de passes jetables sont
la concaténation du mot de passe préfixe et du mot de passe suffixe.

La liste des suffixes est stockée sur le serveur dans un fichier .otpw
appartenant uniquement à l'utilisateur.

\section{Approfondissement}
\subsection{Generation et partage d'un secret}
Le secret est une chaîne choisie par l'utilisateur. Il permet ensuite
en le hashant avec de l'aléatoire de générer une liste de suffixe de mots de
passe jetables. Cette liste sera stockée sur le serveur et donnée à
l'utilisateur.
\begin{algorithm}
    \label{gen}
    \begin{algorithmic}
        \STATE $SEED \leftarrow HASH(stdout)$
        \STATE $l1 \leftarrow vide()$
        \STATE $l2 \leftarrow vide()$
        \FOR{ i = 0 to nbSuffixe}
        \STATE $H \leftarrow HASH(prefixe, SEED)[0..72]$
        \STATE $ajouterListe(l1, H[0..72])$
        \STATE $ajouterListe(l2, HASH(prefixe + H[0..72]))$
        \STATE $SEED \leftarrow H[73..160]$
        \ENDFOR
        \STATE $ecrireServeur(l2)$
        \STATE $envoyerClient(l1)$
    \end{algorithmic}
    \caption{Algorithme de génération de secret}
\end{algorithm}

La liste des suffixe est sotcké sous la forme qui suit:
\begin{itemize}
    \item une ligne contenant \verb?OTPW1?
    \item une ligne contenant:
        \begin{itemize}
            \item Le nombre total de suffixe généré
            \item Le nombre de digit nécessaire pour indicer les suffixes
            \item Le nombre de caractères en base 64 pour un hash
            \item Le nombres de caractères pour un suffixe
        \end{itemize}
    \item Autant de ligne que de suffixes générés, les lignes des
        suffixes déja utilisés sont effacées avec des \verb?-?.
        Ces lignes contiennent:
        \begin{itemize}
            \item Un num\'ero d'OTP
            \item Le hash\'e \verb?RIPEMD-160? du pr\'efixe concat\'en\'e
                au suffixe correspondant à ce num\'ero.
        \end{itemize}
\end{itemize}

\subsection{Génération d'un mot de passe jetable}
Les mots de passe sont générés en concatenant le préfixe et un suffixe 
issu de la liste générée précédemment.

\subsection{Soumission et protocole de vérification}
Les mots de passe sont transmis via le réseau au serveur, il est
recommandé de les transmettre via une connexion sécurisé, en 3 étapes:
\begin{enumerate}
    \item Demande de challenge avec login.
    \item Transmission du login + $HASH(prefixe+suffixe)$ au serveur.
    \item Réponse du serveur d'authentification.
\end{enumerate}

La vérification coté serveur fonctionne selon le principe suivant:
\begin{algorithm}
    \begin{algorithmic}
        \STATE $otp \leftarrow recevoirOTP()$
        \STATE $test \leftarrow rechercher(fichier, otp)$
        \IF{$test$}
        \STATE{$effacer(fichier, otp)$}
        \ENDIF
        \RETURN $test$
    \end{algorithmic}
    \caption{Vérification d'OTP}
\end{algorithm}

\subsection{Synchronisation}
La synchronisation n'est pas importante, puisque le système fonctionne
selon un modèle asynchrone, le serveur pose une question et l'utilisateur
réponds.

\subsection{Reinitialisation}
Pour réinitialiser le système il faut repasser par l'algorithme de 
génération du secret en \ref{gen}.

\section{Analyse générale et sécurité}

\subsection{Avantages et intér\^ets}
L'avantage de ce système est de devoir posséder:
\begin{itemize}
    \item Le préfixe
    \item La liste des suffixe OTP ou d'une machine dans un état
        identique a celui du serveur lors de la génération de la liste des
        suffixe.
\end{itemize} 
pour pouvoir usurper l'identité d'une personne.
Il est impossible de le faire si l'un ou l'autres de ces éléments
manquent à l'appel.

\subsection{Inconvenients et limites}
Le premier inconvénient est qu'il est nécessaire de stocker la liste
des suffixes sur le serveur, et pour l'utilisateur comme token.

Un autre inconvénient est de devoir mémoiriser un préfixe, chose que
l'on ne trouvera pas dans d'autres systèmes d'OTP. 

\subsection{Considérations de sécurité}
\subsubsection{Attaques exhaustives}
Nous considérerons ici que la distribution des OTP est équiprobable.
Les chances de trouver la bonne réponse au challenge proposé sont donc:
        \[\frac{1}{2^{longueur(otp)}}\]
sachant que $longueur(otp)$ vaut 160 (bits) car c'est le r\'esultat d'un hash
\verb?RIPEMD-160?.
en limitant à $n$ le nombre de tentative possible pour un challenge
on en vient donc à \[\frac{n}{2^{160}}\] chances
de casser l'OTP.

Il n'y à cependant aucune recommandation quant au nombre d'essai 
autoris\'es.

\subsubsection{Failles connues}
Attaque \verb?race-for-the-last-key? qui est contourn\'ee par
la demande de trois suffixes en cas de demande de login simultan\'ee.

\subsubsection{Précautions et préconisation}
\begin{itemize}
    \item Mots de passes suffisamment longs.
    \item Réduire le nombre de tentatives pour \'eviter les attaques 
exhaustives
    
\end{itemize}

\section{Utilisations}
\subsection{Cas concrets d'utilisation}
Ce système peut être utilisi\'e avec \verb?PAM? et donc permettre
de l'utiliser pour a peu près n'importe quel service se servant de
\verb?PAM? sous linux.

\subsection{Cas d'utilisation envisagés}
Permettre une authentification sur un site web, par exemple.

\section{Conclusion}
\subsection{Utilisation dans le cadre du projet}
    Ce système pr\'esente l'avantage de ne pas avoir de recalcul pour le token
et que de plus le token seul ne permet pas de s'authentifier. De plus il est
pr\'evu pour que même si l'attaquant prends possession du token, il ne puisse
toujours pas s'identifier.

    De plus il est hautement improbable que deux utilisateur, pour un même pr\'efixe
aient la même liste de suffixes. Car la liste des suffixes est g\'en\'er\'ee selon l'\'etat
de la machine à un instant $t$.

\end{document}
