\documentclass{../res/univ-projet}

\title{\'Etude du syst\`eme OTPW}
\author{Damien \bsc{PICARD}}

\projet{OTP}
\projdesc{\'Etude des syst\`emes de mots de passe jetables}
\filiere{M1SSI}
\logo{../res/logo_univ.png}

\usepackage[T1]{fontenc}
\usepackage[utf8]{inputenc}
\usepackage{algorithm}
\usepackage{algorithmic}


\begin{document}
\maketitle

\begin{abstract}
        Ce document présente une analyse du système OTPW. Toutes les
    informations présentées et analysées proviennent 
    \hyperref[de ce site]{http://www.cl.cam.ac.uk/~mgk25/otpw.html}. Le
    but de cet article est de déterminer dans quelles conditions le
    système OTPW est utilisable, sous quelles conditions et avec quelles 
    garanties de sécurité.
    Ce document sera mis en relation avec plusieurs autres afin de réaliser un
    comparatif entre les systèmes d'OTP majeurs.

    Les algorithmes donnés sont extrapolés des descriptions textuelles.
\end{abstract}
\newpage
\tableofcontents
\newpage

\section{Pré-requis}
        Ce protocole requiert pour pouvoir fonctionner de pouvoir stocker
    sur le serveur un fichier contenant la liste de vérification des OTP (environ 
    4ko pour 300 OTP) dans le dossier de l'utilisateur sous le nom \verb?.otpw?.
    Ainsi que la liste des OTP pour l'utilisateur. Habituellement
    c'est une feuille de papier qui est utilisée pour remplir ce rôle.

        Le générateur d'OTP repose sur la fonction de hachage RIPEMD-160.

\section{Généralités}
        Ce protocole fonctionne grâce à deux listes numérotées, une permettant 
    de former les OTP. Et une permettant de vérifier les OTP. Afin d'initialiser
    ces listes l'utilisateur doit entrer un mot de passe que l'on appellera plus
    tard le préfixe. Le serveur va générer deux listes une qu'il va donner à
    l'utilisateur que l'on appellera liste de suffixes. Et une qu'il va garder
    et qui permettra de vérifier les OTP.

        Ces OTP sont construits en concaténant le préfixe avec l'un des suffixes.
    Le serveur pour vérifier la validité d'un OTP va le hacher avec la fonction
    RIPEMD-160 et comparer ce haché et avec celui de sa liste.

        Quand un utilisateur veut s'authentifier le client demandera un challenge 
    au serveur. Ce challenge est un numéro permettant d'identifier un suffixe sur
    la liste. L'utilisateur devra rentrer son login ainsi que l'OTP généré avec
    ce suffixe. Le serveur si l'OTP correspond bien a celui ayant le même numéro
    dans sa liste authentifiera l'utilisateur. Si l'utilisateur est authentifié
    alors il rayera de sa liste le haché de l'OTP.

\section{Approfondissement}
\subsection{Generation et partage d'un secret}
        La génération de secret ne se fait qu'en aillant accès au serveur. une fois
    connecté sur le serveur l'utilisateur doit exécuter la commande \verb?otpw-gen?.
    Cette commande demandera à l'utilisateur de rentrer son mot de passe préfixe.

        Une fois le mot de passe rentré le serveur initialisera les deux listes avec 
    l'algorithme suivant.
    %TODO: Envisager une version de l'algorithme permettant de varier la taille des suffixes.
    \begin{algorithm}
        \begin{algorithmic}
            \REQUIRE{$nbOTP$: le nombre d'OTP a générer}
            \REQUIRE{$prefix$: le préfixe utilisé pour générer les OTP}
            \STATE{$SEED$ $\leftarrow$ RIPEMD-160(initialiserGen())}
            \FOR{$i = 0 \to nbOTP$}
                \STATE{$hash$ $\leftarrow$ RIPEMD-160($prefix + seed$)}
                \STATE{$suffix \leftarrow hash[0..71]$}
                \STATE{$OTPHash$ $\leftarrow$ RIPEMD-160($prefix + suffix$)}
                \STATE{donner suffix à l'utilisateur}
                \STATE{stocker OTPHash}
                \STATE{$SEED \leftarrow$ RIPEMD-160($SEED + hash[72..160]$)}
            \ENDFOR
        \end{algorithmic}
        \caption{Algorithme d'initialisation.}
        \label{gen}
    \end{algorithm}
    
        La liste retournée à l'utilisateur contient tout les suffixes encodés dans une
    base 64 modifiée. Cette base 64 est dîte modifiée car elle remplace les caractères
     \verb?01l? par \verb?:=%? pour éviter les confusions \verb?0O? et \verb?1Il?

    La fonction $initialiserGen()$ est une fonction qui permet d'initialiser
    la variable $SEED$ avec une valeur aléatoire. Pour ce faire elle hache la sortie
    de plusieurs commande shell, des logs. Le but est d'obtenir une valeur aléatoire
    unique. Cela rends complètement aléatoire la liste des suffixes même si on rentre
    deux fois le même préfixe les listes seront différentes.

    La liste des hachés stockée sur le serveur n'est pas stockée en l'état. Elle respecte
    le format suivant:
    \begin{itemize}
        \item une ligne contenant \verb?OTPW1?
        \item une ligne indiquant:
            \begin{itemize}
                \item Le nombre total de suffixe généré
                \item Le nombre de digit nécessaire pour indicer les suffixes
                \item Le nombre de caractères en base 64 pour un haché
                \item Le nombres de caractères pour un suffixe
            \end{itemize}
        \item Autant de ligne que de suffixes générés, les lignes des
            suffixes déjà utilisés sont effacées avec des \verb?-?.
            Ces lignes contiennent:
            \begin{itemize}
                \item Un numéro d'OTP
                \item Le haché \verb?RIPEMD-160? du préfixe concaténé
                    au suffixe correspondant à ce numéro noté dans une variante de base-64.
            \end{itemize}
    \end{itemize}
    Sur le site de M.Kuhn on peut voir qu'il ne stock que les 72 premiers bits des hachés.\\
    Je pense que c'est une aberration, avec les machines actuelles qui font office de serveur
    on peut se permettre de stocker la totalité des hachés, si nous ne prenons que 72 bits
    cela revient à réduire le nombre d'étape pour obtenir une collision.
    
    De plus on voit apparaître le nombre de caractères pour un suffixe. C'est le seul endroit
    sur le site où il est fait mention de ce paramètre. Peut-on le modifier ?\\
    L'utilisation d'un codage en base 64 est il réellement plus efficace que de stocker
    des valeurs hexadécimale pour des hachés ?




\subsection{Génération d'un mot de passe jetable}
    Pour générer un OTP il suffit de concaténer le préfixe avec un des suffixes.

\subsection{Soumission et protocole de vérification}
        Le protocole pour s'identifier est un protocole type défi/réponse. Le client
    va commencer par demander un défi au serveur. Le serveur tirera un nombre aléatoire
    et vérifiera que cet OTP est disponible dans sa liste (i.e. qu'il n'est pas rayé.).
    Si il est disponible alors il enverra au client ce nombre. Pour s'authentifier l'utilisateur
    devra rentrer son login ainsi que la concaténation de son mot de passe préfixe et du suffixe
    demandé.

    \begin{algorithm}
        \begin{algorithmic}
            \REQUIRE{$OTP$: l'OTP a vérifier.}
            \REQUIRE{$num$: le challenge envoyé au client.}
            \STATE{$expected$ $\leftarrow$ getOTP($num$)}
            \STATE{$hash$ $\leftarrow$ RIPEMD-160($OTP$)}
            \IF{match($hash$, $expected$)}
                \STATE{authentifier l'utilisateur}
                \STATE{rayer l'OTP $num$ de la liste}
            \ELSE
                \STATE{refuser l'authentification}
            \ENDIF
        \end{algorithmic}
        \caption{Algorithme d'authentification}
    \end{algorithm}


\subsection{Ré-initialisation}
        Pour réinitialiser le système il faut repasser par l'algorithme \ref{gen}.

\section{Analyse générale et sécurité}
\subsection{Avantages et intérêts}
        Premièrement il faut deux choses pour s'authentifier, connaître le préfixe et avoir
    la liste des suffixes. Sans ces deux morceaux impossible de reconstituer l'OTP
    demandé lors du login. De plus le ``token'' ne requiert jamais de recalculer un
    OTP. Une fois donnée la liste est figée, il suffit de barrer les OTP utilisés au
    fur et à mesure.

\subsection{Inconvénients et limites}
        Le plus gros problème avec ce système est qu'il est périssable, une fois tout les
    suffixes utilisés, impossible de se reconnecter avec un OTP. Et pour pouvoir générer
    une nouvelle liste il faut avoir un accès au serveur.

\subsection{Considérations de sécurité}
\subsubsection{Attaques exhaustives} % Cas le plus théorique avec caractères codés sur 8-bits.
        Une attaque exhaustive sur ce genre d'OTP est extrèmement compliquée.
    En effet la longueur de l'OTP en lui même est inconnue. Il s'agit d'une
    concaténation d'une chaîne de caractères de taille variable et d'une chaîne
    de caractère de longueur prédéterminée. Il n'existe a priori pas de limite
    de taille pour le préfixe. On peut supposer raisonnable de limiter les
    OTP à des chaînes de strictement moins de 17 caractère.
    Cette ensemble compte donc $\sum_{i=0}^{16}(2^8)^i$ éléments.
    On peut compter  $2^{8\times 17} - 1$ soit $2^136 -1$ autrement dit
    $87112285931760246646623899502532662132735$ possibilités.

        Le réglage par défaut de l'implémentation d'OTPW est de 3 essais par tentative de login
    si la troisième échoue l'utilisateur devra recommencer le processus d'authentification
    et le challenge sera différent. ce qui fait que par attaque exhaustive on arrive à
    une probabilité pour 100\% de réussite à
    \[\frac{3}{2^{136}}\]

        Tout ceci en admettant que les OTP soient constitués de strictement moins
    de 17 caractères. De facto la meilleur attaque revient à trouver une collision.

\subsubsection{Attaques par collisions}
        Étant donné que les OTP son re-hachés par le serveur le système est potentiellement sensible
    aux collisions. La fonction de hachage utilisée étant RIPEMD-160 à ce jour aucune méthode rapide
    n'a été identifiée pour trouver des collisions. Donc les probabilités de trouver des collisions
    suivent le paradoxe des anniversaire soit 50\% de chance en $2^80$.

\subsubsection{Failles connues} 
    % FIXME: Race for the last key
    %        Principe de la connexion parallèle qui arrive a prendre la place d'une machine
    %        dont la connexion est établie ?!!
        Une des attaques envisagée par l'implémentation de base est la race for the last key attack.
    Cette attaque tire parti du fait qu'une machine peut envoyer beaucoup plus vite qu'un utilisateur.
    Du coup un attaquant pourrait sur une connexion parallèle attendre la fin de la saisie et envoyer
    le code de fin de chaîne pour valider l'entrée et se connecter à la place de l'utilisateur.
    Pour palier à cela si deux tentatives de login ont lieu en même temps la seconde se verra demandé
    non plus un mais 3 suffixes.

    % Réflexion personnelle, pas forcément pertinent.
        L'implémentation de M.Kuhn présente un autre ``défaut'', les suffixes sont toujours de taille fixe.
    8 caractères en base 64. Ce qui veut dire que si on voit passer un OTP en clair on peut donc déterminer le
    préfixe en supprimant les 8 derniers caractères. Donc l'attaque exhaustive passe de $(2^8)^i$ où $i$ est
    la taille maximale de chaine que l'on veut tester à $(2^6)^8$ soit $2^48$ cela reste énorme surtout
    face au trois essais proposés mais constitue un amoindrissement considérable de la sécurité à mon sens
    la où l'on pouvait aisément rendre les tailles de suffixes non prédictibles.

\subsubsection{Précautions et préconisation}
        Les OTP étant transmis en clair avec quelques OTP on peut déterminer assez
    aisément le préfixe et la longueur des suffixes. Cela réduit énormément la
    difficulté d'une attaque exhaustive (puisqu'une partie de chaque OTP est déja identifiée).
    Il faut donc s'assurer que les suffixes soient assez long, que le nombre de tentatives 
    soient assez faible et de ne pas (en cas d'échec d'authentification) envoyer le même challenge.


        La solution préconisée est d'utiliser en plus un canal sécurisé, pour avoir une 
    identification point à point du serveur et du client ce qui est le cas par défaut avec
    SSH. Avec ce genre de tunnel on évite aussi de donner des informations sur la longueur du préfixe
    et des suffixes et donc le préfixe. % Et c'est aussi la seule façon de me convaincre de l'utiliser.

        Il est aussi recommandé de faire attention aux permissions du fichier \verb?~/.otpw? car
    il est potentiellement lisible par n'importe qui. Typiquement un \verb?chmod 600? fera l'affaire
    dans le cas de SSH ou de PAM.

\section{Utilisations}
\subsection{Cas concrets d'utilisation}
        Il existe déjà une implémentation qui fourni un module pour s'authentifier avec SSH, et un module
    PAM pour s'authentifier pour a peu près n'importe quoi sous Linux.

\subsection{Cas d'utilisation envisagés}
        PAM couvre déjà un large choix de services. Certains préfèrent tout de même utiliser une implémentation
    ne passant pas par PAM pour par exemple un login sur un site web. Auquel cas il faut réimplémenter un serveur.

\section{Conclusion}
\subsection{Utilisation dans le cadre du projet}
        Dans le cadre du projet on peut envisager un serveur permettant une authentification type OTPW, cependant
    assurer une compatibilité totale avec l'implémentation de Markus Kuhn semble compliquée au vu du peu de documentation
    disponible et qu'il n'y ait pas de ``vraie'' normes sur ce protocole (pas de RFC + encodage en base 64 modifié).
    Et il est difficile de considérer une implémentation comme nouvelle si on se contente de récupérer le code source
    de celle de M.Kuhn.

\subsection{Conclusion personnelle}
        Je pense que ce protocole contient de très bonnes idées comme le fait de devoir posséder le préfixe et
    le suffixe pour pouvoir former un OTP. Cependant un certain nombre de points m'irrite, ne pas pouvoir choisir la
    taille de ses suffixes lors de la génération d'une liste, ne stocker qu'une partie des hachés, l'encodage en base 64 
    trafiqué bien que effectivement les \verb?l1I? ainsi que les \verb?0O? soient parfois difficile a distingués. 

        Pour moi le choix de ne stocker qu'une partie des hachés constitue un affaiblissement de la sécurité puisqu'on facilite
    les collisions. Et ne pas pouvoir choisir la taille de ses suffixe facilite beaucoup trop les attaques exhaustives
    si l'attaquant à récupéré un OTP. En effet on passe d'une quinzaine de caractères et une taille variable à 8 caractères.
    C'est un amoindrissement de taille !

        Une modification plutôt interessante serait d'utiliser non plus la fonction RIPEMD-160 pour générer des hachés mais
    une fonction qui génère des hachés de 256 voir 384 ou 512\footnote{\verb?SHA-2? par exemple} bits ceci afin d'avoir une
    plage de sélection plus large pour les suffixe et de pouvoir faire varier leur taille d'une liste à l'autre.
    De plus comme les listes ne sont générées qu'une fois par le serveur le problème de la puissance de calcul se pose moins.

        Le fait que ce protocole date de 2003 et n'ai jamais été soumis à l'IETF pour une RFC joue sûrement dans la présence
    de ces ``problèmes''.

\end{document}
