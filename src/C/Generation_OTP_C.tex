\documentclass{../../res/univ-projet}
\usepackage{graphicx}

\title{Algorithmes de calcul des OTP en C}
\author{}

\projet{One Time Project}
\projdesc{\'Etude des syst\`emes de mots de passe jetable}
\filiere{M1SSI}
\logo{../../res/logo_univ.png}

\usepackage[T1]{fontenc}
\usepackage[utf8]{inputenc}

\begin{document}
\maketitle

\newpage
\tableofcontents
\newpage 

\section{introduction}
Ce document présente les différents algorithmes implantés permettant la 
génération d'OTP dans le langage C. Nous verrons dans un premier temps 
les outils qui ont été nécessaire pour l'implantation de nos fonctions permettant la 
génération d'OTP basé sur les protocoles HOTP et TOTP. Nous verrons ensuite
les fonctions de génération d'OTP pour les protocoles HOTP et TOTP.
\section{Les outils}
\subsection{Création du secret}
Les protocoles TOTP et HOTP reposent sur le partage d'un secret entre le client 
et le serveur ainsi que sur un élément de synchronisation. Nous allons donc commencer
par voir la création de ce secret.\\
Pour cela nous avons défini un type basé sur la structure suivante:
\begin{verbatim}
typedef struct {
    int length;
    char * buffer;
} secret_struct;
typedef secret_struct * secret;
\end{verbatim}
\begin{description}
\item [length]: nombre d'octet du secret
\item [buffer]: pointeur vers les octets pour représenter le secret
\end{description}
La création d'un secret se fait à l aide de la fonction:
\begin{verbatim}
secret createSecret(int length);
\end{verbatim}
Celle-ci renvoyant l'adresse du nouveau secret.\\
Par opposition, il à été défini une fonction permettant de libérer les
ressources associées à un secret soit :
\begin{verbatim}
int destroySecret(secret key);
\end{verbatim}



\subsection{Fonction HMAC}

\subsection{Opération de réduction de taille d'OTP}
\section{HOTP}
\section{TOTP}
\section{conclusion}
\end{document}
